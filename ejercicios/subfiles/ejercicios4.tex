\documentclass[../main_ej.tex]{subfiles}

\begin{document}
\section{Preguntas clase 4}

\subsection*{Pregunta 1}
Demuestre que la acción 
\begin{equation}
  S[x(t)]=-mc^2\int dt \sqrt{1-\frac{1}{c^2}\left( \frac{dx}{dt}\right)^2}
\end{equation}
reproduce la acción de la partícula libre no-relativista, módulo una constante aditiva. \\
\\

\textbf{Solución:}
Para volver a la acción de la partícula libre no-relativista, se define a la velocidad como $v=\frac{dx}{dt}$ y además se considera a $\frac{v^2}{c^2}$, o sea, estamos condiserando a la velocidad de la luz como muy grande. Así, tomando la expansión el Taylor alrededor del origen, para $v$ obtenemos lo siguiente:
\begin{align*} 
  \sqrt{1-\frac{v^2}{c^2}} & \approx 1 - \frac{v^2}{2c^2} - \mathcal{O}\left( \frac{v^4}{c^4}
  \right)
\end{align*}
Considerando que $\mathcal{O}\left( \frac{v^4}{c^4}\right)$ es muy pequeño, entonces 
\begin{align*}
  S[x(t)] & \approx -mc^2\int_{t_1}^{t_2} dt\left(1-\frac{v^2}{2c^2}  \right) \\
  & \approx -mc^2\int_{t_1}^{t_2} dt + \frac{m}{2}\int_{t_1}^{t_2}dt\; v^2 \\
  & \approx -mc^2(t_2-t_1) + \frac{m}{2}\int_{t_1}^{t_2}dt \; v^2
\end{align*}
Así se obtiene que, la acción para la partícula no-relativista, derivada de la acción $S[x(t)]$ se divide en dos contribuciones, la energía cinética clásica de la partícula y una constante aditiva, ahora, el segundo término de la acción obtenida
\begin{equation}
  \frac{m}{2}\int_{t_1}^{t_2}dt \; v^2 = \int_{t_1}^{t_2}dt\frac{m}{2}v^2
\end{equation}
Corresponde a la dicha acción de la partícula libre no relativista
\begin{equation}
  S[x(t)]=\int_{t_1}^{t_2}dt\frac{m}{2}v^2
\end{equation}
en lo cual su término de dentro, corresponde al Lagrangiano para la partícula libre no-relativista
\begin{equation}
  L_{no-rel} = \frac{m}{2}v^2
\end{equation}
\\
%\subsubsection*{}
\subsection{Pregunta 2}
Muestre que la acción de una particula relativista es invariante bajo boost. \\
\\
\textbf{Solución:} 
\\

Considerando a la acción de la particula relativista \eqref{accion-rel} y a las transformaciones de un Boost a lo largo de x \eqref{Boost-t} y \eqref{Boost-x}. Vamos a construir una acción $\tilde{S}[\tilde{x}(\tilde{t})]$ y demostrar que finalmente $\tilde{S}[\tilde{x}(\tilde{t})] = S[x(t)]$. 

\begin{equation}
  S[x(t)] = -mc^2\int dt \sqrt{1 - \frac{1}{c^2}\left( \frac{dx}{dt} \right)^2} \label{accion-rel}
\end{equation}

\begin{equation}
  \tilde{t} = \frac{t - \frac{v}{c^2}x}{\sqrt{1 - \frac{v^2}{c^2}}} \label{Boost-t}
\end{equation}

\begin{equation}
  \tilde{x} = \frac{x - vt}{\sqrt{1 - \frac{v^2}{c^2}}} \label{Boost-x}
\end{equation}

Vamos a construir una acción $\tilde{S}[\tilde{x}(\tilde{t})]$ y demostrar que finalmente $\tilde{S}[\tilde{x}(\tilde{t})] = S[x(t)]$. \\

Por lo tanto, a partir de \eqref{Boost-t} y \eqref{Boost-x}: 

\begin{align}
  d\tilde{t} &= \frac{\partial \tilde{t}}{\partial t} dt + \frac{\partial \tilde{t}}{\partial x}dx \notag \\
  &= \frac{dt}{\sqrt{1 - \frac{v^2}{c^2}}} - \frac{\frac{v}{c^2}dx}{\sqrt{1 - \frac{v^{2}}{c^2}}} \label{dtilde}
\end{align}\\

\begin{align}
  d\tilde{x} &= \frac{\partial \tilde{x}}{\partial t}dt + \frac{\partial \tilde{x}}{\partial x}dx \notag\\
  &= -\frac{v dt}{\sqrt{1 - \frac{v^2}{c^2}}}dt + \frac{dx}{\sqrt{1- \frac{v^2}{c^2}}} \label{dequis} 
\end{align}\\

Dividiendo \eqref{dequis} sobre \eqref{dtilde}: 

\begin{align}
  \frac{d\tilde{x}}{d\tilde{t}} &= \frac{dx -vdt}{dt - \frac{v}{c^2}dx} \notag \\
  &= \frac{\frac{dx}{dt}-v}{1 - \frac{v}{c^2}\frac{dx}{dt}} \label{Boost-vel}
\end{align}

Así, considerando nuestra acción  $\tilde{S}[\tilde{x}(\tilde{t})]$ y reemplazando \eqref{dtilde} y \eqref{Boost-vel}.

\begin{align*}
  \tilde{S}[\tilde{x}(\tilde{t})] &= -mc^2 \int d\tilde{t} \sqrt{a - \frac{1}{c^2}\left(\frac{d\tilde{x}}{d\tilde{t}}\right)^2} \\
  \tilde{S}[\tilde{x}(\tilde{t})] &= -mc^2 \int \left[ \frac{dt}{\sqrt{1 - \frac{v^2}{c^2}}} - \frac{\frac{v}{c^2}dx}{\sqrt{1- \frac{v^2}{c^2}}} \right] \cdot \sqrt{1 - \frac{1}{c^2}\left(\frac{ \frac{dx}{dt}- v}{1 - \frac{v}{c^2}\frac{dx}{dt}}\right)^2} \\
  \tilde{S}[\tilde{x}(\tilde{t})] &= -mc^2 \int \frac{dt}{\sqrt{1 - \frac{v^2}{c^2}}} \cancel{\left( 1 - \frac{v}{c^2} \frac{dx}{dt}\right)} \cdot \frac{\sqrt{c^2 \left[ \left(1 -\frac{v}{c^2}\frac{dx}{dt}\right)^2 - \frac{1}{c^2}\left(\frac{dx}{dt}-v\right)^2 \right]}}{c \cancel{\left( 1 - \frac{v}{c^2}\frac{dx}{dt} \right)}} \\
  \tilde{S}[\tilde{x}(\tilde{t})] &= -mc^2 \int \frac{dt}{\sqrt{1 - \frac{v^2}{c^2}}} \cdot \sqrt{1 - \cancel{\frac{2v}{c^2}\frac{dx}{dt}} + \frac{v^4}{c^4} \left(\frac{dx}{dt}\right) - \frac{1}{c^2}\left(\frac{dx}{dt}\right)^2 + \cancel{\frac{2v}{c^2}\frac{dx}{dt}} -\frac{v^2}{c^2}} \\
  \tilde{S}[\tilde{x}(\tilde{t})] &= -mc^2 \int \frac{dt}{\cancel{\sqrt{1 - \frac{v^2}{c^2}}}} \cdot \sqrt{\cancel{\left( 1 - \frac{v^2}{c^2} \right)} \cdot \left( 1 - \frac{1}{c^2}\left( \frac{dx}{dt} \right)^2 \right)} \\
  \tilde{S}[\tilde{x}(\tilde{t})] &= -mc^2\int dt \sqrt{1 - \frac{1}{c^2}\left( \frac{dx}{dt} \right)^2} \\
  \tilde{S}[\tilde{x}(\tilde{t})] &= S[x(t)]\\
\end{align*}

Así la acción será invariante bajo un boost a lo largo del eje $x$. Para demostrar esto para los tres ejes espaciales (en 3D), solo hay que seguir el analogo a este desarrollo para un boost a lo largo de $y$ y $z$.

\subsection{Pregunta 3}
Muestre que la acción anterior es invariante bajo traslaciones temporales, y muestre que el teorema de Noether implica que tal invariancia es la responsable de la conservación de la energía relativa, dada por
\begin{equation}
  E=\frac{mc^2}{\sqrt{1-\frac{v^2}{c^2}}}
\end{equation}  

\textbf{Solución:}
\\  

\subsection{Pregunta 4}
Muestre que la acción anterior es invariante bajo traslaciones espaciales, y muestre que el teorema de Noether implica que tal invariancia es la responsable de la conservación del momento lineal relativista, dado por
\begin{equation}
  p=\frac{mv}{\sqrt{1-\frac{v^2}{c^2}}}
\end{equation}
\\
\textbf{Solución:} 
\\

Considerando una traslación espacial unidimensional a lo largo del eje $x$: 

\begin{align}
    \tilde{x} &= x + \epsilon \\
    \tilde{y} &= y \\
    \tilde{z} &= z \\
    \tilde{t} &= t \\
\end{align}

Considerando los diferenciales de tiempo $dt =d\tilde{t}$ y espaciales:

\begin{align}
    d\tilde{x} &= \frac{\partial \tilde{x}}{\partial x}dx \\
    &= dx
\end{align}

Deducimos que la velocidad: 

\begin{equation}
    \frac{d\tilde{x}}{d\tilde{t}} = \frac{dx}{dt}
\end{equation}

Así, veamos la variación de la acción: 

\begin{align}
    \delta S &= \tilde{S}[x(t) + \epsilon] - S[x(t)] \\
    &= -mc^2 \int dt \sqrt{1 - \frac{1}{c^2}\left( \frac{d}{dt}(x(t)+\epsilon) \right)^2} + mc^2 \int dt \sqrt{1 - \frac{1}{c^2}\frac{dx}{dt}}  \\
    &= -mc^2 \int dt \left[ mc^2 \int dt \sqrt{1 - \frac{1}{c^2}\frac{dx}{dt}} - mc^2 \int dt \sqrt{1 - \frac{1}{c^2}\frac{dx}{dt}} \right]\\
    &= 0\\
\end{align}

Por lo tanto, para una acción bajo transformaciones espaciales es invariante $\tilde{S}[\tilde{x}(\tilde{t})] = S[x(t)]$ . Y su cantidad de contorno $B=0$. Por otro lado, si quisieramos obtener la cantidad conservada para este tipo de transformaciones deberemos aplicar: 

\begin{equation}
    \partial_{\dot{x}}L \delta q - B = C^{te} \label{custi0n}
\end{equation}

Si consideramos $\dot{x}= dx/dt$ y aplicamos \eqref{custi0n}: 

\begin{align}
    \partial_{\dot{x}}\left( -mc^2\sqrt{1 - \frac{\dot{x}}{c^2}} \right) &= cte\\
    \frac{-mc^2}{\sqrt{1 - \frac{\dot{x}}{c^2}}} \cdot \frac{-2 \dot{x}}{2} &= cte \\
    \frac{m\dot{x}}{\sqrt{1 - \frac{x^2}{c^2}}} &= cte
\end{align}

La cual será la definición del momentum lineal relativista, considerando $v = \dot{x}$: 

\begin{equation}
    P=\frac{mv}{\sqrt{1-\frac{v^2}{c^2}}}
\end{equation}

\subsection{Pregunta 5}
A partir de las expresiones anteriores, muestre la relación de dispersión relativista
\begin{equation}
  E=\sqrt{p^2c^2+m^2c^4}
\end{equation}
\textbf{Solución:}
\\
\subsection{Pregunta 6}
Imagine que un estudiante que ya pasó por mecánica clásica, le pregunta ¿qué siginfica la expresión $E=mc^2$ y de donde viene? ¿ Qué respondería ?  \\
\\
\textbf{Solución:}
\\

Significaría la relación de dispersión de una partícula que no se esta moviendo o tiene un momentum $|p|=0$. O mejor traducido, la energía asociada a una partícula relativista que esta en reposo para algún observador inercial. Como un sistema de referencia lagrangiano pero eso es de fluidos. 

%%%%%%%%%%%%%%%%%%%%%%%%%


 
\end{document}