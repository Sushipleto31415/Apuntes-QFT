\documentclass[../main_ej.tex]{subfiles}

\begin{document}
\section{Preguntas clase 6}
Demuestre que las transformaciones lineale homogéneas que dejan invariante el intervalo están caracterizadas por una matriz $\Lambda^\mi_{\;\nu}$ tal que
\begin{equation}
  \eta_{\mu\nu}\Lambda^\mu_{\;\alpha}\Lamda^{\nu}_{\;\beta} = \eta_{\alpha \beta}
\end{equation}
 donde $\eta_{\mu\nu}=diag(1,-1,-1,-1)$. \\
 Demuestre que las matrices $\Lambda$ que satisfacen la ecuación, forman un grupo (existe una identidad, el producto de dos de ellas da una de ellas, la inversa de $\Lambda$ también deja invariante el intervalo).\\
 Encuentre la matriz $\Lambda$ para un boost a lo largo de eje $x$, un boost a lo largo del eje $z$, una rotación alrededor del eje $y$. Para cada una de estas matrices, expanda a primer orden en el parámetro y escriba $\Lambda$ como $I+\omega$ donde $I$ es la matriz identidad y $\omega$ es la matriz que depende del parámetro de la transformación a primer orden. \\
 Demuestre que en efecto la energía y el momento lineal relativista $E/c$ y $\vec{p}$, respectivamente, bajo boost transforman como las componentes de un cuadrivector, lo que lleva a la definición de cuadri-monentum.
 \begin{equation}
   p^\mu=(p^0,\vec{p}) = \left(\frac{E}{c} , \vec{p} \right)
  \end{equation}
\textbf{Solución:}
\end{document}
