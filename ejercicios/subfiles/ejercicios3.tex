\documentclass[../main.tex]{subfiles}

\begin{document}
\section{Preguntas clase 3}
\subsection{Pregunta 1}
Deduzca la relación de dispersión de la partícula libre no-relativista. \\
\textbf{Solución:}
\\
\subsection{Pregunta 2}
Enuncie y explique los principios de la Relatividad Especial. \\
\\
\textbf{Solución:}
\subsection{Pregunta 3}
Siga la discusión que aparece en Landau y Lifshitz V2, acerca de cómo los principios de la Relatividad Especial implican la invariancia del intervalo. \\
\\
\textbf{Solución:}
\subsection{Pregunta 4}
Escriba las siguietes transformaciones de manera explícita: Traslación temporal, traslació espacial en $x$, traslación espacial en $y$, traslación espacial en $z$, rotación en el plano $(x,y)$, rotación en el plano $(z,x)$, boost a lo largo del eje $x$, boost a lo largo del eje $y$, boost a lo largo del eje $z$. Dé una interpretación clara de cada una de las transformaciones y muestre que el intervalo es invariante.\\
\\
\textbf{Solución:}
  





 
\end{document}