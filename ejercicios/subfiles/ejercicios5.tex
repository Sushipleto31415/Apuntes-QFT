\documentclass[../main_ej.tex]{subfiles}

\begin{document}
\section{Preguntas clase 5}

\subsection{Pregunta 1}
Usando la relación de Plank  $E=\hbar \omega$ y la relación de Broglie $p=\frac{h}{\lambda}$, defina operadores que actuando sobre una onda plana $\psi=Ae^{-i\omega t+ikx}$, implementen las relaciones anteriores de autovalores. \\
Usando el mapeo de operadores que acaba de deducir, muestre la relación de dispersión no-relativista para una partícula libre, lleva a la ecuación Schrödinger. \\
Considere la ecuación de Schrödinger para una partícula en tres dimensiones, y muestre que
\begin{equation*}
  \frac{\partial \rho}{\partial t} + \vec{\nabla} \cdot \vec{j} = 0.
\end{equation*}
donde $\rho=\psi^*\psi$. Encuentre la expresión para $\vec{j}$. Escriba la forma integral de tal ecuación. \\
Usando la relación de dispersión relativista, $E^2=p^2c^2 + m^2c^4$, repita el procedimiento anterior e interprete su resultado, Considerando la ecuación de Klein-Gordon, muestre que las ondas planas $  \psi=Ae^{-i\frac{E}{h}t + i\frac{E}{h}x}$ lleva a dos familias de ecuaciones, con energía $E$ tanto positiva como negativa. Dibuje tal espectro y discuta si tal espectro es  o no sensato para un sistema físico.\\
Considere la ecuación de primer orden 
\begin{align*}
  \alpha \partial_t \psi +\beta_1 \partial_{x^1}\psi + \beta_2 \partial_{x^2}\psi + \beta_3\partial_{x_3}\psi + m\psi  & = 0 \\
  \alpha \partial_t \psi + \beta_i \partial_i\psi + m\psi & = 0
\end{align*}
Actue sobre esta ecuación con el operador $(\alpha \partial_t+\beta_j\partial_j + m)$ y muestre que se recupera la ecuación de Klein-Gordon, siempre y cuando
\begin{align*}
  \alpha^2  & = 1 \\
  \alpha \beta_i + \beta_i \alpha & = 0 \\
  \beta_i\beta_j = -\delta_{ij}
\end{align*}
¿ Es posible realizar estas últimas ecuaciones asumiendo que los cuatro objetos $\alpha$ y $\beta_i$ son números? \\
\\
\textbf{Solución:} Para resolver la primera parte del problema, necesitamos operadores que actuando sobre la función de onda plana, dada por
\begin{equation*}
  \psi(x,t) = Ae^{-i\omega t + ikx}
\end{equation*}
Nos deje las siguientes relaciones, primero, para la relacion de Plank, el operador que actúa sobre la función de onda plana será
\begin{equation*}
  \hat{E} = i\hbar \partial_t
\end{equation*}
Comprobemoslo:
\begin{align*}
  \hat{E}\{\psi(x,t)\}  & = i\hbar \partial_t \left( Ae^{i\omega t + ikx} \right) \\
  & = i\hbar\cdot (-i\omega) A e^{i\omega t + ikx} \\
  & = \hbar\omega \psi(x,t)
\end{align*}
Con lo cual se concluye que $\hat{E}\{\psi\} = \hbar \omega \psi$. Ahora para el caso de la relación de Broglie, el operador estará dado por los siguiente
\begin{equation*}
  \hat{p} = -i\hbar \partial_x
\end{equation*}
Comprobemoslo:
\begin{align*}
  \hat{p}\{\psi\} & = -i\hbar \partial_x \left( Ae^{-i\omega t+ikx} \right) \\
  & = -i\hbar \cdot \left( ik \right) \psi \\
  & = \hbar \frac{2\pi}{\lambda} \\
  & = \frac{h}{\lambda}
\end{align*}
Con lo cual se concluye que $\hat{p}\{\psi\}=\frac{h}{\lambda}\psi$.  \\
La relación de dispersión con potencial clásica corresponde a la siguiente expresión
\begin{equation}
  E = \frac{p^2}{2m} + V(x)
\end{equation}
Con lo cual, primero obtendremos el operador momenta al cuadrado, el cual será, para $\hat{p}=-i\hbar\nabla$ en lo cual, como en este caso generalizaremos a una función de onda en 3d $\partial_x$ pasa a ser el operador gradiente, con lo cual
\begin{equation*}
  \hat{p}^2 = \hbar \nabla^2
\end{equation*}
Con lo cual, por ahora, tendremos que la relación de dispersión puede ser escrita de la siguiente manera
\begin{equation*}
  E = \frac{-\hbar}{2m}\nabla^2 + V(x)
\end{equation*}
Ahora nos queda aplicar el operador Energía, el cual está dado por $\hat{E}=i\hbar\partial_t$ lo que en la relación de dipersión es tal que
\begin{equation*}
  i\hbar \partial_t = -\frac{\hbar}{2m}\nabla^2 + V(x)
\end{equation*}
El cual es el operador de la ecuación de Schrödinger, ahora veamos como actúa sobre una función de onda $\Psi(\vec{x},t)$:
\begin{equation}
  -\frac{\hbar}{2m}\nabla^2\Psi + V(x)\Psi = i\hbar \partial_t\Psi 
\end{equation}
La cual es la archi-conocida ecuación de Schrödinger para una partícula cuántica bajo un potencial $V(x)$. \\
Para la siguiente parte del ejercicio nos dan una ecuación de continuidad 
\begin{equation*}
  \partial_t\rho + \nabla \cdot \vec{j} = 0
\end{equation*}
En lo cual $\rho=\psi^*\psi$ corresponde a la densidad de probabilidad de la función de onda $\psi$, notese que $\psi^*$ corresponde al conjugado de la función de onda, ahora, para obtener una expresión para $\vec{j}$ tendremos que, primero, buscar una forma de relacionar la ecuación de Schrödinger, con lo cual, aprovechandonos que la densidad de probabilidad depende de ambas funciones de onda, usando regla de la cadena, la derivamos parcialmente en el tiempo $t$ como sigue 
\begin{equation*}
  \partial_t\rho = \psi\partial_l\psi^* + \psi^*\partial_l\psi  
\end{equation*}
Ahora, en esta expresión tenemos una derivada parcial de la función de onda y onda conjugada con respecto al tiempo, ahora, esto también lo podemos obtener a partir de la ecuación de Schrödinger, tal que
\begin{align*}
  \partial_t \psi & = \frac{i\hbar}{2m}\nabla^2 \psi -\frac{i}{\hbar}V(x)\psi \\ 
  \partial_t \psi^* & = -\frac{i\hbar}{2m}\nabla^2 \psi^* +\frac{i}{\hbar}V(x)\psi^* 
\end{align*}
Ahora reemplazamos esto en la derivada partial de $\rho$.
\begin{align*}
\partial_t\rho & = \psi^*\left(\frac{i\hbar}{2m}\nabla^2 \psi -\frac{i}{\hbar}V(x)\psi \right) + \psi \left(  -\frac{i\hbar}{2m}\nabla^2 \psi^* +\frac{i}{\hbar}V(x)\psi^* \right) \\
  & = \frac{i\hbar}{2m} \left( \psi^* \nabla^2\psi -\psi \nabla^2\psi^* \right) \\
  & = \frac{i\hbar}{2m}\nabla \cdot \left( \psi^*\nabla \psi - \psi \nabla \psi^* \right)
\end{align*}
Con lo cual, es posible expresar $\vec{j}$ como 
\begin{equation*}
  \nabla \cdot \vec{j} = \frac{i\hbar}{2m} \nabla \cdot \left( \psi^*\nabla \psi - \psi \nabla \psi^* \right)
\end{equation*}
Por tanto, la expresión para el vector $\vec{j}$ haciendo uso de la ecuación de continuidad y de Schrödinger es tal que:
\begin{equation}
  \vec{j} = \frac{i\hbar}{2m}\left( \psi^*\nabla \psi - \psi\nabla \psi^* \right)
\end{equation}
Ahora tenemos en cuenta la relación de dispersión relativista, 
\begin{equation}
  E^2=p^2c^2+m^2c^4
\end{equation}
Para lo cual, podemos escribir lo siguiente
\begin{equation*}
  \hat{E}^2 = -\hbar^2\partial^2_t \quad , \quad \hat{p}^2 = -\hbar^2\nabla^2
\end{equation*}
Ahora, reemplazamos esto en la relación de dispersión relativista para encontrar lo siguiente
\begin{align*}
  -\hbar^2\partial^2_t & = -\hbar^2c^2\nabla^2 + m^2c^4 \\
  \frac{1}{c^2}\partial^2_t-\nabla^2 + \frac{m^2c^2}{\hbar^2} & = 0
\end{align*}
A este operador se le llama, operador de Klein-Gordon y su actuar sobre una función de onda da origen a la ecuación de Klein-Gordon, como sigue
\begin{equation}
  \left(\frac{1}{c^2}\partial^2_t-\nabla^2+\frac{m^2c^2}{\hbar^2}\right)\psi = 0
\end{equation}
o tabíen puede ser escrito en función del d`Alambertiano $\square$,
\begin{equation}
  \left(\square + \frac{m^2c^2}{\hbar^2}\right)\psi = 0
\end{equation}
Ahora consideramos una ecuación de ondas planas
\begin{equation*}
  \psi(\vec{x},t)=Ae^{-i\frac{E}{\hbar}t+i\frac{p}{\hbar}x}
\end{equation*}
Sobre la cual haremos actual el operador de Klein-Gordon, tal que
\begin{align*}
  \left(\frac{1}{c^2}\partial^2_t-\nabla^2+\frac{m^2c^2}{\hbar^2}\right)\psi  & = 0 \\
  \frac{1}{c^2}\left(i\frac{E}{\hbar}\right)^2\psi - \left(i\frac{p}{\hbar}\right)^2\psi + \frac{m^2c^2}{\hbar^2} \psi &  =  \\
  \frac{-E^2}{\hbar^2c^2} + \frac{p^2}{\hbar^2} + \frac{m^2c^2}{\hbar^2} & =  \\
  -E^2 + p^2c^2 + m^2c^4 & = \\
  E^2 & = p^2c^2 +  m^2c^4 \\
  E & = \pm \sqrt{p^2c^2 + m^2c^4}
\end{align*}
Lo cual nos da lugar a dos familias de soluciones, las cuales son las siguientes
\begin{itemize}
  \item $E=\sqrt{p^2c^2+m^2c^4}$ la cual es la relación de dispersión para partículas
  \item $E=-\sqrt{p^2c^2+m^2c^4}$ la cual es la relación de dispersión para antipartículas 
\end{itemize}
Ahora nos dan la siguiente ecuación de primer orden
\begin{equation}
  a\partial_t\psi + \beta_i\partial_i\psi + m \psi = 0
\end{equation}
Sobre el cual nos piden hacer actuar el operador $(\alpha \partial_t - \beta_j \partial_j + m)$, así lo hacemos actuar como sigue
\begin{align*}
  \left(\alpha\partial_t + \beta_j\partial_j  - m\right)\left( \alpha \partial_t\psi + \beta_i \partial_i \psi + m\psi  \right) & = 0 \\
  \alpha^2 \partial^2_t\psi + \alpha \beta_i \partial_t\partial_i \psi +m \alpha \partial_t \psi +  \beta_j\alpha \partial_j \partial_t \psi + \beta_j \beta_i \partial_j\partial_i \psi + m\beta_j\partial_j \psi - m\alpha \partial_t\psi - m\beta_i \partial_i \psi - m^2\psi & = 0 
\end{align*}
Ahora que se expandió la acción del operador sobre la ecuación, usaremos las condiciones impuestas, con las cuales la ecuación debiera reducirse a Klein-Gordon, de primeras impondremos que , $\alpha^2=1$, $\beta_i\beta_j=-\delta_{ij}$ y $\alpha \beta_i + \beta_i \alpha$ 
\begin{align*}
  \alpha^2\partial^2_t \psi +\alpha \beta_i\partial_t \partial_i \psi + \cancel{\alpha m \partial_t \psi} + \beta_j\alpha \partial_j \partial_t \psi - \delta_{ij} \partial_j \partial_i\psi + m\beta_j \partial_j \psi +\cancel{-m\alpha \partial_t \psi} - m\beta_i\partial_i \psi - m^2 \psi & = 0 \\
  \alpha^2\partial^2_t \psi + \left( \alpha \beta_i + \beta_i \alpha  \right)\partial_i \partial_t \psi - \partial^2_i\psi - m^2\psi &  =   \\
  \partial^2_t \psi - \partial^2_i - m^2\psi & = 
\end{align*}
Ahora, notamos que la ecuación, tras la acción del operador dado, se reduce a la ecuación de Klein-Gordon, la cual está dada por:
\begin{equation}
  \boxed{\left(\partial_t^2 - \nabla^2 + m^2\right)\psi = 0}
\end{equation}
\end{document}
