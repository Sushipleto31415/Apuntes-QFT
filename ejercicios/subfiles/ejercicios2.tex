\documentclass[../main_ej.tex]{subfiles}

\begin{document}
\section{Ejercicios clase 2}
\subsection{Pregunta 1}
Manipulando la segunda ley de Newton para una partícula en presencia de una energía potencial $U(x(t))$, muestre que la energía es conservada. \\
\\
\textbf{Solución:}
\subsection{Pregunta 2}
Deduzla la transformación infinitesimal que representa una traslación en el tiempo actuando sobre un grado de libertad $q(t)$. \\
\\
\textbf{Solución:}
\subsection{Pregunta 3}
Deduzla la transformación infinitesimal que representa una traslación espacial del grado de libertad $\vec{r}(t)$.
\\
\\
\textbf{Solución:}
\subsection{Pregunta 4}
Deduzca la transformación infinitesimal que representa una rotación en el plano $(x,y)$, actuando sobre la posición de una partícula $\vec{r}(t)=x\hat{x}+y\hat{y}+z\hat{z}$.
\\
\\
\textbf{Solución:}
\subsection{Pregunta 5}
Demuestre, explicando cada paso, que una transformación infinitesimal que deja quasi-invariante la acción de un conjunto de grados de libertad $q_A$ con $A=1,\dots ,N$, permite construir una cantidad conservada $Q$. Construya tal cantidad conservada para los siguientes casos
\begin{align*}
  S[q(t)] & = \int dt \left[\frac{m}{2}{\dot{q}}^2 - U(q) \right] \quad \text{invariancia bajo traslaciones temporales} \\ 
  S[q(t)] & = \int dt \left[ \frac{m}{2}\frac{dx^i}{dt}\frac{dx^i}{dt}\right] \quad \text{invariancia bajo traslaciones espaciales} \\
  S[q(t)] & = \int dt \left[  \frac{m}{2}\frac{d\vec{r}}{dt}\cdot \frac{d\vec{r}}{dt}-U(|\vec{r}|)\right] \quad \text{invariancia bajo rotaciones en el plano }(x,y)
\end{align*}
Extienda el último caso a al invariancia bajo rotaciones generales, cuya acción finita está dada por $\vec{r}_{\text{transformado}}=O\vec{r}$ con $O$ una matriz ortogonal de determinante $1$. \\
\\
\textbf{Solución:}
Para resolver este problema usaremos el  teorema de Noether, el cual nos dice que una transformación infinitesimal deja la acción quasi-invariante, entones podemos construir una cantidad conservada $Q$ como veremos a continuación para los siguientes casos. \\
\\
\textbf{Traslación temporal:} Se nos pide encontrar la cantidad conservada ante la invariancia sobre traslaciones temporales de la siguiente acción, dada por
\begin{equation*}
  S[q(t)] = \int_{t_1}^{t_2} dt \left[ \frac{m}{2}\dot{q}^2 - U(q) \right]
\end{equation*}
tenemos que la traslación temporal infinitesimal está dada por $\delta q = \epsilon \dot{q}$ en lo cual $\epsilon$ es un término arbitrario y muy pequeño, ahora, para encontrar que la acción es invariante bajo dicha transforación es necesario variar la acción, esto será como sigue
\begin{align*}
  \delta S[q(t)]  & = S[q(t)+\delta q(t)] - S[q(t)] \\
  & = \int_{t_1}^{t_2}dt \; L[q(t)+\delta q(t), \frac{d}{dt}(q(t)+\delta q(t))] - L[q(t),\dot{q}(t)]
\end{align*}
Para lo cual, luego de tomar una serie de Taylor en dos dimensiones, nótese en el enunciado, que en la acción dada no es dependiente explícitamente del tiempo, con lo cual el término extra dependiente del tiempo en la serie de Taylor multivariable se neglecta, así se llega a que la variación de la acción, cuando el Lagrangiano es independiente explícitamente del tiempo, está dada por
\begin{equation}
  \delta S[q(t)] = \int_{t_1}^{t_2} dt \left( \partial_q L \; \delta q + \partial_{\dot{q}L \;\delta \dot{q}}  \right)
\end{equation}
Con lo cual solo queda calcular los términos involucrados en la variación del Lagrangiano, tal que
\begin{equation}
  \delta S[q(t)] = \int_{t_1}^{t_2} dt \left[ -\partial_qU\; \epsilon\dot{q} + m\dot{q}\epsilon \right] = \int_{t_1}^{t_2} dt \frac{dB}{dt}
\end{equation}
Con lo cual es necesario expresar la variación del Lagrangiano como la derivada total de un cantidad, tal que 
\begin{equation}
  \frac{d}{dt}(-U(q)\epsilon + m q \epsilon) = \frac{dB}{dt}
\end{equation}
Con lo cual hemos encontrado la cantidad $B$ la cual es
\begin{equation}
 B = -U(q)\epsilon + mq\epsilon
\end{equation}
Así, por teorema de Noether, podemos construir una canidad conservada $Q$, cuya expresión es
\begin{align*}
  \partial_{\dot{q}}L \; \delta q - B = Q 
\end{align*}
\subsection{Pregunta 6}
\textbf{Partícula conforme:}
Calcule la cantidad conservada asociada a la transformación de simetría $\delta x(t)=\frac{\epsilon}{2}x(t)-\epsilon t\frac{dx(t)}{dt}$, para la acción de la partícula conforme
\begin{equation}
  I[x(t)] = \int_{t_1}^{t_2} dt \left( \frac{m}{2}\left( \frac{dx}{dt} \right)^2 - \frac{\alpha}{x^2} \right).
\end{equation}
Considerando la cantidad conservada asociada esta simetría, además de la conservación de la energía, encuentre la trayectoria de la partícula $x(t)$ de forma alebraica. \\
\\
\textbf{Solución:}
Para encontrar la cantidad conservada es necesario variar la acción con respecto a al transformaciñon de simetría infinitesimal $\delta x(t) = \frac{\epsilon}{2}x(t) - \epsilon t \frac{dx(t)}{dt}$, para lo cual, tenemos el siguiente Lagrangiano 
\begin{equation}
  L = \frac{m}{2}\left( \dot{x} \right)^2 - \frac{\alpha}{x^2}
\end{equation}
Ahora, para ver si la acción es invariante o quasi-invariante, variamos la acción y por tanto, el Lagrangiano, la variación de la accion es la siguiente
\begin{align*}
  \delta I[x(t)] & = I[x(t)+\delta x(t)]-I[x(t)] \\
  & = \int_{t_1}^{t_2} dt \left(L[x(t) + \delta x(x)] - L[x(t)]\right)
\end{align*}
Lo cual, como ya sabemos, luego de una serie de Taylor se reduce a
\begin{equation*}
  \delta I[x(t)] = \int_{t_1}^{t_2} dt \left( \partial_{x}L \; \delta x + \partial_{\dot{x}}L\; \delta \dot{x} \right)
\end{equation*}
Ahora, sabemos la expresión para el Lagrangiano y para la transformación de simetría, con lo cual solo queda calcular las derivadas parciales que aparecen el la variación y el álgebra subsiguiente
\begin{align*}
  \delta I [x(t)] & = \int_{t_1}^{t_2} dt \left[ \frac{2\alpha}{x^3} \left( \frac{\epsilon}{2}x -\epsilon t \dot{x}  \right) -  m\dot{x}\left( \frac{\epsilon}{2}\dot{x} + \epsilon t\ddot{x} \right) \right] \\
  & = \int_{t_1}^{t_2}  dt \epsilon\left[ \frac{\alpha}{x^2} - \frac{2\alpha t \dot{x}}{x^3} - \frac{m\dot{x}^2}{2} - m t \dot{x}\ddot{x}  \right]
\end{align*}
Ahora, para usar el teorema de Noether, que nos dice que, una cantidad conservada $B$ será tal que
\begin{equation}
  \delta I[(t)]=\int_{t_1}^{t_2} dt \delta L= \int_{t_1}^{t_2} \frac{dB}{dt} 
\end{equation}
Para ello, es necesario dejar a la expresión anterior como una derivada total, con lo cual notamos que el primer término de la integral corresponde a
\begin{equation*}
  \frac{\alpha}{x^2} - \frac{2\alpha t \dot{x}}{x^3} = \frac{d}{dt}\left( \frac{\alpha t}{x^2} \right)
\end{equation*}
Y además el segundo término es tal que
\begin{equation*}
  -\frac{m\dot{x}^2}{2} - mt\dot{x}\ddot{x} = -\frac{d}{dt}\left(\frac{mt\dot{x}^2}{2}\right)
\end{equation*}
 Con lo cual es posible reescribir la variación del Lagrangiano tal que
\begin{align*}
  \delta I[x(t)]=  \int_{t_1}^{t_2} dt \epsilon \frac{d}{dt}\left( \frac{\alpha t}{x^2}-\frac{mt\dot{x}^2}{2} \right) = \int_{t_1}^{t_2}\frac{dB}{dt}
\end{align*}
Con lo cual tenemos que, la cantidad  conservada $B$ es igual a 
\begin{equation}
  \boxed{ B = \frac{\alpha t}{x^2} - \frac{mt\dot{x}^2}{2}}
\end{equation}
Por tanto, la acción será quasi-invariante con un término de borde $B$. Ahora, para interpretar esta cantidad conservada $B$ y encontrar mediante ella las ecuaciones de movimiento, tendremos en cuenta lo siguiente, $B$ es muy parecido a la energía del sistema, de hecho, la energía del sistema en este caso estará dada por
\begin{equation*}
  E=\frac{m\dot{x}^2}{2}+\frac{\alpha}{x^2}
\end{equation*}
Con lo cual manipularemos alegebraicamente la expresión de la cantidad conservada $B$ para meterlo en la energía $E$ 
\begin{align*}
  B & = \frac{\alpha t}{x^2} - \frac{mt\dot{x}^2}{2} \quad, \quad /\cdot \frac{1}{t} \\  \frac{B}{t} =  \frac{\alpha}{x^2} - \frac{m\dot{x}^2}{2} \\
  \frac{m\dot{x}^2}{2} = \frac{\alpha}{x^2} - \frac{B}{t}
\end{align*}
Así, tenemos una expresión para la energía cinética en términos de la energía potencial y la constante del movimiento $B$, así reemplazamos esto en $E$
\begin{align*}
  E & = \frac{\alpha}{x^2} - \frac{B}{t} + \frac{\alpha}{x^2} \\
  E + \frac{B}{t} & = \frac{2\alpha}{x^2} \\
  \frac{2\alpha}{E + \frac{B}{t}} & = x^2 \\
  \sqrt{\frac{2\alpha}{E + \frac{B}{t}}} = x 
\end{align*}
Con lo cual hemos obtenido las ecuaciones de movimiento a partir de una cantidad conservada $B$ y la energía $E$ y está dada por
\begin{equation}
 \boxed{  x(t)=\sqrt{\frac{2\alpha}{E+\frac{B}{t}}} \quad , \quad \forall t>0 }
\end{equation}
\subsection{Pregunta 7}
\textbf{Lagrangiano para la partícula cargada en el campo electromagnetico.} En este ejercicio utilice notación de índices, y la convención de Einstein. Considere una partícula cargada eléctricamente, de carga $q$, en presencia de un campo electromagnetico externo descrito por los potenciales $\phi(t,\vec{x})$ y $\vec{A}(t,\vec{x})$. Recuerde que el campo eléctrico y el campo magnético se obtienen a partir de estos potenciales mediante las siguientes expresiones
\begin{align*}
  \vec{E} & =-\nabla \phi - \partial_t\vec{A} \rightarrow E_i = -\frac{\partial \phi}{\partial x^i} - \frac{\partial A_i}{\partial t} \\
  \vec{B} & = \nabla \times \vec{A} \rightarrow B_i = \epsilon_{ijk}\frac{\partial A_k}{\partial x^j}
\end{align*}
La partícula está descrita por el siguiente Lagrangiano
\begin{equation*}
  L=\frac{m}{2}|\vec{v}|^2 - q\phi + q\vec{A}\cdot \vec{v} = \frac{m}{2}\dot{x}^i\dot{x}^i - q\phi(t,x) + qA_i(t,x) \frac{dx_i}{dt}.
\end{equation*}
Muestre que el Lagrangiano lleva a la expresión corrrecta para la fuerza de Lorentz, es decir
\begin{equation}
  m\vec{a}=q(\vec{E}+\vec{v}\times\vec{B})
\end{equation}
Asumiendo que los potenciales no dependen del tiempo, muestre que la acción es invariante bajo traslaciones temporales
\begin{equation}
  \delta x^i=\epsilon \dot{x}^i 
\end{equation}
y calcule la energía como cantidad conservada en el sistema. \\
Finalmente, para potenciales generales que dependen tanto de $t$, como de $\vec{x}$, muestre que el Hamiltoniano toma la forma
\begin{equation}
  H=\frac{1}{2m}\left( \vec{p}-q\vec{A}(t,\vec{x}) \right)^2 + q\phi(t,\vec{x})
\end{equation}
Discuta la diferencia entre $H$, y la energía calculada en el paso anterior. 
\\
\\
\textbf{Solución:}
Para mostrar que dicho Lagrangiano lleva a la fuerza de Lorentz, usaremos las ecuaciones de Euler-Lagrange para $x^i \quad , \quad i=1,\dots,N$ coordenadas, las cuales está dadas por
\begin{equation}
  \partial_{x^i} L  - \frac{d}{dt}\partial{\dot{x}^i}L = 0
\end{equation}
Con lo cual, calculemos dichos términos
\begin{align*}
  \partial_{x^i} L & = \partial_{x^i} \left( \frac{m}{2}\dot{x}^i\dot{x}^i - q\phi + qA_i\dot{x}^i  \right) \\
  & = -q\partial_{x^i}\phi + q\partial_{x^i}A_j\dot{x}^j
\end{align*}
y además
\begin{align*}
  \partial_{\dot{x}^i}L & = \partial_{\dot{x}^i} \left( \frac{m}{2}\dot{x}^i\dot{x}^i - q\phi + qA_i\dot{x}^i \right) \\
  & = m\dot{x}^i + qA_i
\end{align*}
Ahora este último término lo diferenciamos en el tiempo tal que 
\begin{align*}
  \frac{d}{dt}\partial_{\dot{x}^i} L & = \frac{d}{dt}\left( m\dot{x}^i + qA_i \right) \\
  & = m\ddot{x} + q\left( \partial_{t}A_i + \partial_{x^j}A_i\dot{x}^j  \right)
\end{align*}
Ahora reemplazamos esto en las ecuaciones de Euler-Lagrange como sigue
\begin{align*}
  m\ddot{x} & = -\partial_{x^i}\phi + q\partial_{x^j}A_i\dot{x}^j - q\partial_{t}A_i - \partial_{x^j}A_i\dot{x^j} \\
  & = q\left( -\partial_{x^i}\phi - \partial_{t}A_i \right) + q\left( \partial_{x^i}A_j - \partial_{x^j}A_i \right)\dot{x^j} \\
  & = qE_i + q\epsilon_{ijk}\dot{x}^jB^k
\end{align*}
Con lo cual, esto puede ser escrito de la siguiente forma
\begin{equation}
  qE_i + q\epsilon_{ijk}\dot{x}^jB^k = q\vec{E} + q(\vec{v}\times\vec{B})
\end{equation}
Con lo cual hemos confirmado que el Lagrangiano para una partícula inmersa en un campo electromagnético, este llevará a las ecuaciones de movimiento que es la fuerza de Lorentz.\\
Para la siguiente parte se asumirá que los potenciales $\phi$ y $A_i$ son independientes del tiempo. \\
Se tiene la acción
\begin{equation}
  S[x^i(t)] = \int_{t_1}^{t_2} dt \; L[\dot{x}^i,x^i,t]
\end{equation}
Luego se tiene la siguiente transformación infinitesimal. La cual corresponde a una traslación temporal
\begin{equation}
  \delta x^i = \epsilon \dot{x}^i 
\end{equation}
Ahora, para mostrar que la acción es invariante, variamos la acción
\begin{equation}
  \delta S[x^i(t)] = S[x^i(t) + \delta x^i(t)] - S[x^i(t)] = \int_{t_1}^{t_2} dt \left(L\left[\frac{d}{dt}(x^i(t)+\delta x^i(t), x^i(t)+\delta x^i(t),t)\right] - L[\dot{x}^i, x^i ,t] \right)
\end{equation}
Ahora, tomando una serie de Taylor, obtenemos que la variación del Lagrangiano es igual a
\begin{equation}
  \delta L = \partial_{x^i}L \;\delta x^i + \partial_{\dot{x}^i} L \; \delta \dot{x}^i
\end{equation}
Con lo cual calculamos esta expresión usando la transformación dada y el Lagrangiano para una partícula sumida en un campo electromagnético.
\begin{align*}
  \delta L  & = \left( -q\partial_{x^i}\phi +q\partial_{x^i}A_i \dot{x}^i \right)\epsilon \dot{x}^i + \left( m\dot{x}^i + qA_i \right)\epsilon\ddot{x}^i \\
  & = -q\epsilon \dot{x}^i\partial_{x^i}\phi + q\epsilon (\dot{x}^i)^2\partial_{x^i}A_i + \epsilon m \dot{x}^i \ddot{x}^i + q \epsilon A_i \ddot{x}^i 
\end{align*}
Ahora bien, el Lagrangiano es independiente del tiempo podemos escribir lo siguiente
\begin{equation*}
  \delta L = \epsilon\left(\partial_{x^i}L \; \dot{x}  + \partial_{\dot{x}^i}L \; \ddot{x}^i + \cancel{\partial_t L }^0\right) = \epsilon\ D_t L
\end{equation*}
Con lo cual, la cantidad B será
\begin{equation*}
  B = \epsilon L = \epsilon \left( \frac{m}{2}\dot{x}^i\dot{x}^i - q\phi + qA_i \dot{x} \right)
\end{equation*}
y por tanto, mediante el teorema de Noether podemos encontrar una cantidad conservada Q la cual está dada por lo que sigue
\begin{equation*}
  \partial_{\dot{x}^i}L \; \delta x^i - B = Q 
\end{equation*}
reemplazando el valor que encontramos para la cantidad $B$  y la traslación temporal, se obtiene 
\begin{equation}
  \partial_{\dot{x}^i} L \; \epsilon \dot{x}^i  - \epsilon L = Q
\end{equation}
Ahora desarollamos esta expresión para encontrar la cantidad Q conservada
\begin{align*}
  \epsilon\left( m{\dot{x}^i}\dot{x}^i +qA_i\dot{x}^i  - \frac{m}{2}\dot{x}^i\dot{x}^i + q\phi - qA_i\dot{x}^i\right)  & = Q \\
  \epsilon \left( \frac{m}{2}\dot{x}^i\dot{x}^i + q\phi \right) & = Q \\
  \epsilon E  & = Q
\end{align*}
Con lo cual, hemos identificado la cantidad conservada $Q$ como la energía del sistema y por tanto, la energía de una partícula sumida en un campo electromagnético cuyos potenciales son idependientes del tiempo es conservada ante traslaciones espaciales. 
 \\
 Ahora para el cálculo del Hamiltoniano tenemos que, la definición del Hamiltoniano involucra una transformación de Legendre, lo que se representa de la siguiente forma
 \begin{equation}
   H(p_i,x^i,t) = p_i\dot{x}^i - L(\dot{x}^i,x^i,t)
\end{equation}
En lo cual, $p_i$ son los momenta generalizados, cuya expresión está dada por
\begin{equation*}
  p_i=\partial_{\dot{x}^i}L
\end{equation*}
con lo cual, solo queda calcular, los momenta están dados por 
\begin{align*}
  \partial_{\dot{x}^i}L  = m\dot{x}^i + qA_i  & = p_i \\
  \dot{x}^i & = \frac{p_i}{m} - \frac{q}{m}A_i
\end{align*}
Con lo cual, reescribimos el Lagrangiano usando los momenta
\begin{equation*}
  L = \left( \frac{p_i}{2m} - \frac{q}{2m}A_i \right)^2 -q\phi + qA_i\left( \frac{p_i}{m}-\frac{q}{m}A_i \right)
\end{equation*}
Con lo cual, solo queda calcular el Hamiltoniano
\begin{align*}
  H & = p_i\left(\frac{p_i}{m} -\frac{q}{m}A_i\right) -\frac{1}{2m}\left( p_i - qA_i \right)^2 +q\phi - qA_i\left( \frac{p_i}{m}-\frac{q}{m}A_i \right) \\
  & =  \frac{p_i^2}{m} - \frac{qp_iA_i}{m} - \frac{p_i^2}{2m} + \frac{qp_iA_i}{m} - \frac{q^2A_i^2}{2m} + q\phi -  \frac{qp_iA_i}{m} + \frac{qA_i^2}{m} \\
  & = \frac{p_i^2}{2m} - \frac{qp_iA_i}{m} + \frac{q^2A_i^2}{2m} + q\phi \\
  & = \frac{(p_i - qA_i)^2}{2m} + q\phi
\end{align*}
Con lo cual el Hamiltoniano para una particula sumida en un campo electromagnético está dado por
\begin{equation}
  H(p_i,x^i,t) = \frac{(p_i-qA_i)^2}{2m} + q\phi
\end{equation}
Ahora, la diferencia entre la energía y el Hamiltoniano recae en que, el Hamiltoniano incluye términos cinéticos, el potencial magnético, y la energía no, lo que permite que la enegía sea una cantidad conservada en traslaciones temporales, o sea, la energía se conserva en el tiempo. 
 
\end{document}