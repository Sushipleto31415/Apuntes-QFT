\documentclass[../main.tex]{subfiles}

\begin{document}
\section{Vigésimo quinta clase}
Nos convencimos que de las ecuaciones de Maxwell podemos obtener una forma covariante de la Ley de Gauss de la forma
\begin{equation*}
  \partial_\mu F^{\mu \nu} = 0 
\end{equation*}
En lo cual, el tensor de Faraday viene de Fábrica con los indices abajo, lo que tiene fuertes implicancias físicas, se sugiere indagar en esto para poder tener mejor noción de su construcción. \\
Ahora, necesitamos encontrar un principio de acción para el campo $A_\mu$, o sea, para el cuadri-potencial electromagnético.
\begin{equation*}
  I[A_\mu] = \int dt \int d^3x \mathfrak{L} = \int dt L
\end{equation*}
Algo muy imporante de notar es el cómo transforma el tensor de Faraday cuando se aplica una transformación entre observadores inerciales, el tensor de Faraday transforma de la siguiente forma
\begin{equation}
  \tilde{F}_{\mu\nu}= \Lambda^\alpha_\mu \Lambda^\beta_\nu F_{\alpha \beta}
 \end{equation}
 También es importante notar el cómo transforman las ecuaciones de Maxwell ante una transformación de Lorentz en su forma diferencial o vectorial. Pero ambos observadores inerciales ven la misma ecuación dinámica 
 \begin{equation*}
   \frac{\tilde{F}^{\mu \nu}}{\partial \tilde{x}^\mu} = 0
 \end{equation*}
 ya que, recordemos que el campo fundamental no es $F_{\mu \nu}$ sino $A_\mu$ la cual transforma como
 \begin{equation*}
   \tilde{A}_\mu(\tilde{x}) = \Lambda^\alpha_\mu A_\alpha(x) 
 \end{equation*}
 y que el principio de acción sea invariante bajo dicha transformación además de devolver la ecuación dinámica buscada. Proponemos una densidad Lagrangeana de la siguiente forma
 \begin{equation*}
   \mathfrak{L} = c_1 F^{\mu \nu}F_{\mu \nu} + \epsilon^{\mu \nu \lambda \rho} F_{\mu \nu} F_{\alpha \rho}
 \end{equation*} 
 Y podemos hacer el ejercicio de escribir dicha densidad Lagrangeana en términos de los campos eléctrico y magnético de forma explícita, lo que sería de la forma
\begin{equation*}
  \mathfrak{L} = c_1 \left( |\vec{E}|^2 - |\vec{B}|^2 \right) + c_2 \left( \vec{E}\cdot \vec{B} \right)
\end{equation*} 
Cuando calculamos las ecuaciones de movimiento nos daremos cuenta que el segundo término ($E\cdot B$), no contribuye, ya que, es un término de borde.
\subsection{Principio de acción de Maxwell}
A lo cual podemos convencernos que el siguiente principio de acción es el buscado
\begin{equation}
  I[A_\mu ] = \int d^4x \left( -\frac{1}{4}F_{\mu \nu} F^{\mu \nu} \right)
 \end{equation}
 Variamos la acción tal que, asumiendo  que la transformación infinitesimal tiene las puntas amarradas $\delta A_\mu(t_1 \wedge t_2,\vec{x}) = 0$ y además $\delta A_\mu \stackrel{\longrightarrow}{|\vec{x}|\to \infty} 0 $ 
 \begin{align*}
   \delta I = 0  &= I[A_\mu + \delta A_\mu] - I[A_\mu] \\
   & =  \int d^4x \left( -\frac{1}{4}\left[ \partial_\mu \left( A_\nu + \delta A_\nu \right) -\partial_\nu \left( A_\mu + \delta A_\mu  \right) \right] \left[ \partial^\mu \left( A^\nu + \delta A^\nu  \right)  - \partial^\nu \left( A^\mu + \delta A^\mu \right)\right] \right) - I \\
   & = \int d^4x - F^{\mu \nu} \partial_\mu \delta A_\nu \\
   & = \int d^4x \partial_\mu \left( -F^{\mu \nu} \delta A_\nu \right) + \partial_\mu \left( F^{\mu \nu} \right)\delta A_\nu \\
   & = B.T \; + \int d^4x \partial_\mu F^{\mu\nu} \delta A_\nu = 0 \; \Rightarrow \; \partial_\nu F^{\mu \nu} = 0
 \end{align*}
El \textbf{Principio de acción de Maxwell}
\begin{equation}
  I[A_\mu]  = \int d^4x \left( -\frac{1}{4}F_{\mu \nu} F^{\mu \nu} \right)
 \end{equation}
 Es invariante bajo: \\
 \textbf{Transformaciones de Lorentz:} Rotaciones, Boost, $\tilde{x}^\mu = \Lambda^\mu_\alpha x^\alpha$.
 \begin{equation*}
   \tilde{{A}}_\mu(\tilde{x}) = \Lambda^\alpha_\nu A_\alpha(x) 
 \end{equation*}
 Cuya carga conservada es del tipo
 \begin{equation*}
   Q = \int d^3x j^t, \quad j^\mu = \frac{\partial \mathfrak{L}}{\partial \partial_\mu A_\nu}\delta A_\nu - B^\mu
 \end{equation*}
\textbf{Transformaciones de Poincaré}, Traslaciones temporales y espaciales, $\tilde{x}^\mu = x^\mu + a^\mu$. 
\begin{align*}
  \tilde{\phi}(\tilde{x}) & = \phi(x) \\
  \tilde{\psi}(\tilde{x}) & = \psi(x) \\
  \tilde{{A}}_\mu(\tilde{x}) & = A_\mu(x)
\end{align*}
A lo cual, calculamos la traslación espacio temporal
\begin{align*}
  \tilde{{A}}_\mu(x) & = A_\mu(x-a)  \\
  & = A_\mu(x) - a^\alpha \partial_\alpha A_\mu \\
  &\Rightarrow \delta A_\mu = -\epsilon^\alpha \partial_\alpha A_\mu 
\end{align*}
y de ello encontramos que la corriente conservada será el tensor densidad energía momento (bajo traslaciones espacio-temporales).
\begin{equation}
  j^\mu = -\epsilon^\nu T_{\;\nu}^\mu
 \end{equation}
 en donde el tensor densidad de energía momento es el siguiente, que es análogo a la energía clásica.
 \begin{equation}
   T_{\mu \nu} = F_{\mu \alpha} F_\nu^{\;\alpha} - \frac{1}{4}\eta_{\mu \nu} F_{\alpha \beta} F^{\alpha \beta}
  \end{equation}
  El cual cumple con que tiene divergencia nula, ya que es una corriente conservada
  \begin{equation}
    \partial_\mu T^\mu_{\;\nu} = 0
   \end{equation}
Y el tensor energía momento, al estar compuesto por tensores de Faraday, transforma como
\begin{equation}
  \tilde{T}_{\mu \nu}(\tilde{x}) = \Lambda^\alpha_{\;\nu} A^\beta_{\;\nu}T_{\alpha \beta}(x)
 \end{equation}
 \subsection{Hacia teorías de Gauge no-abelianas}
 Las cuales describen los Gluones, $w^{\pm}, z^0$. \\
Supongamos que tenemos un grupo $g(x)$, como por ejemplo el grupo $SU(N)$, el cual está caracterizado por su tabla de multiplicación o de Cayley. El grupo actúa sobre un fermión como
\begin{equation}
  \tilde{\psi}(x) = g(x)\psi(x)
 \end{equation}
 Que no es nada más que un grupo actuando sobre una matriz columna, pero la transformación no actúa sobre los  $x^\mu$ sino que actúa directamente sobre los campos, lo que corresponde a una transformación interna, los campos se mezclan entre sí. \\
 Ahora veamos como transforman las derivadas parciales de $\psi$
 \begin{align*}
   \frac{\partial \tilde{\psi}(x)}{\partial x^\mu} & = \partial_\mu \left( g(x)\psi(x) \right) \\
   & = g(x) \partial_\mu \psi(x) + \partial_\mu g(x) \psi(x)
 \end{align*}
 Pero esto no es una transformación homogénea, con lo cual es necesario definir una derivada que me permita que la transformación es homogénea\footnote{Similar a esto es cuando se define la derivada coraviante para un espacio curvo usando los símbolos de Christoffel $\Gamma^l_{\mu \nu}$}.\\
 Así, definimos la siguiente derivada covariante
 \begin{align*}
   D_\mu \psi(x)&  : = \partial_\mu \psi(x) + A_\mu(x)\psi(x) \\
   \tilde{D_\mu\psi(x)}&  : = \partial_\mu\tilde{\psi}(x) + \tilde{{A}}_\mu (x)\tilde{\psi}(x) \\
   \tilde{D_\mu\psi(x)} = g(x)D_\mu\psi(x)
 \end{align*}
 Con lo cual podemos desarrollar lo siguiente
 \begin{align*}
   \partial_\mu \tilde{\psi} + \tilde{{A}}_\mu \tilde{\psi} & = g\partial_\mu \psi + gA_\mu \psi \\
   \partial_\mu g \psi + g \partial_\mu \psi + \tilde{{A}}_\mu g\psi & = g\partial_\mu \psi + gA_\mu \psi \\
   \tilde{{A}}_\mu g & = gA_{\mu} - \partial_\mu g \\
   \tilde{{A}}_\mu g g^{-1} & = gA_\mu g^{-1} - \partial_\mu gg^{-1}
\end{align*}
\subsubsection{Fórmula de Manuel Cartán}
La ley de transfomación de la conexión de Gauge (campo de Gauge) es:
\begin{equation}
  \tilde{{A}}_\mu(x) = gA_\mu(x) g^{-1} - \partial_\mu g g^{-1}
\end{equation}
\begin{align*}
  \partial_\mu g g^{-1} & = \partial_\mu \left( exp(\lambda_a(x)T_a) \right)\; exp(-\lambda_a(x)T_a) \\
  & = exp (\lambda_a(x)T_a) \partial_\mu (\lambda_b(x))T_b exp(-\lambda_c(x)T_c)
\end{align*}
La expresión
\begin{equation}
  \tilde{{A}}_\mu(x) = gA_\mu(x)g^{-1} - \partial_\mu gg^{-1}
 \end{equation}
 En donde $A_\mu(x)$ es una combinación lineal de generadores
 \begin{equation}
   A_\mu(x) = A^a_\mu(x)T_a 
  \end{equation}
  En donde  $A_\mu(x)$  esto puede ser el grupo $SU(3)$ el cual cuenta con 8 generadores y da cuenta de los 8 gluones. \\
  Si el grupo es abeliano $U(1)$ hay un solo $A_\mu(x)$ y puede ser escrito como $g=g^{i\xi(x)}$
  \begin{align*}
    \tilde{{A}}_\mu(x) & = A_\mu - \partial_\mu (e^{i\xi}) e^{i\xi} \\
    & = A_\mu - i\partial_\mu \xi \\
    & = A_\mu + \partial_\mu \alpha,\quad \alpha=-i\xi \\
    & = \tilde{{A}}_\mu (x) 
  \end{align*}
\end{document} 
