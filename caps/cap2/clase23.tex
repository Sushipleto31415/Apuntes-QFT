\documentclass[../main.tex]{subfiles}

\begin{document}
\section{Vigésima tercera clase}
En la base quiral tenemos que
\begin{align*}
  \gamama^0 & = \begin{pmatrix}
    0 & I \\ I & 0 
  \end{pmatrix}
  , \quad \gamma^i = \begin{pmatrix}
    0 & \sigma^i \\ -\sigma^i & 0
  \end{pmatrix}
  \\
  S_{\alpha \beta} &  =\frac{1}{4} \left[ \gamma_\alpha , \gamma_\beta \right] \\
  D[\Lambda] & = e^{\frac{1}{2}\omega^{\alpha\beta}S_{\alpha \beta}} \\
  D[\Lambda] & = \begin{pmatrix}
    \# & 0 \\ 0 & \#
  \end{pmatrix}
\end{align*}
Tal que, los spinores transforman como
\begin{align*}
  \Psi(x) \rightarrow \tilde{\Psi}(x) & = D[\Lambda] \Psi(\Lambda^{-1}), \quad \text{o sea} \\
  & \begin{pmatrix}
    \# & 0 \\ 0 & \#
  \end{pmatrix}
  \begin{pmatrix}
    \Psi_1 \\ \Psi_2 \\ \Psi_3 \\ \Psi_4
  \end{pmatrix}
\end{align*}
\textbf{Símbolo de Weyl:}
\begin{align*}
  \Psi= \begin{pmatrix}
    u_+ \\ u_-
  \end{pmatrix}
  ,\quad u_+ \;\text{y} \; u_-\; \text{son spinores de 2 componentes}
\end{align*}
Así, la densidad Lagrangeana de Dirac está dada por
\begin{align*}
  \mathfrak{L} & = \bar{\Psi} \left( i\gamma^\mu \partial_\mu - m \right)\Psi, \quad \bar{\Psi}:= \Psi^\dagger \gamma^0 \\
  & = iu^\dagger_- \sigma^\mu \partial_\mu u_+ iu^\dagger_+\bar{\sigma}^\mu \partial_\mu u_+ - m \left( u^\dagger_+ u_- + u_-^\dagger u_+ \right) \\
  & \rightarrow \; I=I \left[ u_-, u^\dagger_-, u_+ , u^\dagger_+ \right]  
\end{align*}
En donde tenemos que pensar esto como el campo escalar complejo
\begin{align*}
  \mathfrak{L} & = \partial_\mu \varphi \partial^\mu \varphi* - m^2 \varphi \varphi* \\
  & I[Re[\varphi],Im[\varhpi]] , \quad I[\varphi,\varphi*]
\end{align*}
Volviendo a la densidad Lagrangeana de Dirac, debemos recordar lo siguiente
\begin{align*}
  \Psi^\dagger \Psi & = u^\dagger_+ u_+ + u^\dagger_- u_- = \left( u_+^\dagger , u^\dagger_- \right) \begin{pmatrix}
    u_+ \\ u_-
  \end{pmatrix} \\
  \bar{\Psi}\Psi & = \left( u_+^\dagger , u_-^\dagger \right) \begin{pmatrix}
     & I \\ I & 0 
  \end{pmatrix}
  \begin{pmatrix}
    u_+ \\ u_-
  \end{pmatrix}
\end{align*}
La invariancia de Lorentz requiere que la presencia de un término de masa necesariamente contenga tanto $u_+$ como $u_-$.\\
El término de masa invariante de Lorentz no es complatible con fijar $u_+=0$ o $u_-=0$. \\
Un spinor de Dirac es llamado spinor quiral si en la base de Weyl toma la forma:
\begin{align*}
  \Psi = \begin{pmatrix}
    u_+ \\ 0 \\ 0 
  \end{pmatrix}
  \quad \text{o}, \quad \Psi= \begin{pmatrix}
    0 \\ 0 \\ u_-
  \end{pmatrix} \\
  \text{Si} \; m=0 \rightarrow & \delta_{u_+} I = 0 \Leftrightarrow \delta_{u^\dagger_+} = 0 \\
  & \delta_{u_-} I = 0 \Leftrightarrow \delta_{u_-^\dagger} I = 0
\end{align*}
Recordar que en este contexto las matrices de Pauli no tienen 4 componentes, si no que su primer componente es la matriz 1,
\begin{align*}
  \sigma_{2\times 2}^\mu & = \left( 1, \sigma^i \right) \\
  \bar{\sigma}_{2\times 2}^\mu & = \left( 1,-\sigma^\mu \right)
\end{align*}
Lo que finalmente nos lleva a las ecuaciones de Weyl, las cuales están dadas por
\begin{align*}
  i\bar{\sigma}^mu \partial_\mu u_+ & = 0 \\
  i\sigma^\mu \partial_\mu u_- & = 0
\end{align*}
El spinor de Dirac transfoma en la representación:
\begin{align*}
  \left( 1/2 , 0 \right) \oplus \left( 0,1/2 \right)
\end{align*}
del grupo de Lorentz
\begin{itemize}
  \item $u_+$ transforma en la $(1/2,0)$, $e^{\frac{\vec{\chi}\cdot \vec{\sigma}}{2}}$ 
  \item $u_-$ transforma en la $(0,1/2)$ , $e^{\frac{-\vec{\chi}\cdot \vec{\sigma}}{2}}$
\end{itemize}
\begin{align*}
  \gamma^\mu & = A \gamma^\mu_{Weyl} A^{-1} \\
  \rightarrow S^{Weyl}_{\alpha \beta} & = \frac{1}{4}[\gamma^{Weyl}_\alpha,\gamma^{Weyl}_\beta], \quad S_{\alpha \beta} = \frac{1}{4} A S^{\Weyl}_{\alpha \beta}A^{-1} \\
  D[\Lambda] & = e^{\frac{1}{2}S^{\alpha \beta}S_{\alpha \beta}} \\
  & = A e^{\frac{1}{2}\omega^{\alpha \beta}S_{\alpha\beta}^{Weyl}} A^{-1} =A D[\Lambda]  A^{-1}
\end{align*}
¿Cómo caracterizamos un spinor quiral en una base genérica?, para ello definimos la siguiente matriz
\begin{align*}
  \gamma^5 := -i\gamma^0\gamma^1\gamma^2\gamma^3 \\
  \left( \gamma^5 \right)^2 = 1, \\
\end{align*}
\textbf{Demostración:} recordemos que satisfacen el álgebra de Clifford, $\gamma^\mu \gamma^\nu = - \gamma^\nu \gamma^\mu + 2\eta^{\mu \nu}1_4$ tal que, en la signatura $(+,-,-,-)$
 \begin{align*}
   \left( \gamma^5 \right)^2 & = -\gamma^0 \gamma^1 \gamma^2\gamma^3\gamma^2\gamma^1\gamma^3 \\
   & = \gamma^1 \gamma^2\gamma^3\gamma^3\gamma^1\gamma^2\gamma^3 \\
   & = - \gamma^2\gamma^3\gamma^2\gamma^3 \\
   & = - \gamma^3 \gamma^3 \\
   & = +1
 \end{align*} 
 y además
 \begin{align*}
   \{\gamma^5,\gamma^\mu\} = 0 \\
   \{\gamma^5,\gamma^0\} & = \gamma^5\gamma^0 + \gamma^0 \gamma^5 \\
   & = -i \left( \gamma^0 \gamma^1 \gamma^2 \gamma^3 \gamma^0 + \gamma^0 \gamma^0 \gamma^1 \gamma^2 \gamma^3 \right) \\
  & = -i \left( -\gamma^1 \gamma^2 \gamma^3 + \gamma^1 \gamma^2 \gamma^3 \right) \\
   & = 0
 \end{align*}
 \textbf{Tarea:} hacer lo mismo anti-conmutador, pero con $\gamma^1$. \\
 \textbf{Tarea:} En la base quiral
 \begin{align*}
   \gamma^5 \begin{pmatrix}
     0 \\ 0 \\ u_-
   \end{pmatrix} = 
   \begin{pmatrix}
     0 \\ 0 \\ u_-
   \end{pmatrix}
   , \quad \gamma^5 = 
   \begin{pmatrix}
     I & 0 \\ 0 & -I
   \end{pmatrix}, \\
   \gamma^5 \begin{pmatrix}
     u_+ \\ 0 \\ 0
   \end{pmatrix} ) =+1
   \begin{pmatrix}
     u_+ \\ 0 \\ 0
   \end{pmatrix}
 \end{align*}
 \begin{itemize}
  \item Los spinores quirales son auto-spinores de $\gamma^5$ 
   \item Los spinores de Dirac izquierdos tienen autovalor +1
   \item Los spinores de Dirac derechos tienen autovalo -1
 \end{itemize}
 Proyectore $P_+$ y $P_-$
 \begin{align*}
   P_+ & 0 \frac{1}{2}\begin{pmatrix}
     I + \gamma^5
   \end{pmatrix} =_{In\; Weyl} \begin{pmatrix}
     I_2 & 0 \\ 0 & 0
   \end{pmatrix} \\
   P_- & = \frac{1}{2}\left( I-\gamma^5 \right) =_{In\; Weyl} \begin{pmatrix}
    0 & 0 \\ 0 & 0
   \end{pmatrix}
 \end{align*}
 \textbf{Demostrar:} que
 \begin{align*}
   P_+ P_- & = 0 \\
   P^2_+ = P_+ , \quad P^2_- = P_-
 \end{align*}
 en una base arbitraria.
 \textbf{En la base quiral:}
 \begin{align*}
   \begin{pmatrix}
    \Psi_1 \\ \Psi_2 \\ 0 \\ 0 
   \end{pmatrix} = P_+ \begin{pmatrix}
     \Psi_1 \\ \Psi_2 \\ \Psi_3 \\ \Psi_4
   \end{pmatrix}
   \\
   \begin{pmatrix}
     0 \\ 0 \\ \Psi_3 \\ \Psi_4
   \end{pmatrix} = P_- 
   \begin{pmatrix}
     \Psi_1 \\ \Psi_2 \\ \Psi_3 \\ \Psi_4
   \end{pmatrix}
 \end{align*}
 \textbf{En una base genérica:}
 \begin{align*}
   \Psi_{\pm} = P_{\pm} \Psi
 \end{align*}
 En donde $\Psi_{\pm}$ son spinores quirales en una base genérica. \\
 \textbf{Base de Majorana:}
\begin{align*}
  \gamma^0 = \begin{pmatrix}
    0 \\ \sigma^2 \\ \sigma^2 & 0
  \end{pmatrix}, \quad \gamma^1 = \begin{pmatrix}
    i\sigma^3 & 0 \\ 0 & i\sigma^3
  \end{pmatrix}
  \\
  \gamma^2 = \begin{pmatrix}
    0 & -\sigma^2 \\ \sigma^2 & 0
  \end{pmatrix}, \quad 
  \gamma^3 = \begin{pmatrix}
    -i \sigma^1 & 0 \\ 0 & -i\sigma^1
  \end{pmatrix}
  \\ 
  \simga^1 = \begin{pmatrix}
    0 & 1 \\ 1 & 0
  \end{pmatrix}, \quad \sigma^2 = \begin{pmatrix}
    0 & -i \\ i & 0 
  \end{pmatrix}
  , \quad \sigma^3 = 
  \begin{pmatrix}
    1 & 0 \\ 0 & -1
  \end{pmatrix}
\end{align*} 
En donde $\gamma^\mu$ cumple con el Álgebra de Clifford
\begin{align*}
  \{\gamma^\mu , \gamma^\nu\} = 2\eta^{\mu \nu}1_4, \quad \gamma^\mu_{Maj} = A \gamma^mu_{Weyl}A^{-1} \\
  \left( \gamma^\mu_{Maj} \right)* = - \gamma^\mu_{Maj} \Rightarrow S^{Maj}_{\mu \nu} = \frac{1}{4} \left[ \gamma^{Maj}_\mu , \gamma^{Maj}_{\nu} \right], \quad \text{Son reales} \\
  \left( S^{Maj}_{\mu\nu} \right)* = \frac{1}{4} \left( \left( \gamma^{Maj}_{\mu} \right)* \left( \gamma_\nu^{Maj} \right)* - \left( \gamma^{Maj}_\nu \right)* \left( \gamma^{Maj}_\mu \right)*\right) = S^{Maj}_{\mu \nu}
\end{align*}
En la base de Majorana $S_{\alpha \beta}$ son reales, por tanto
\begin{align*}
  D[\Lambda] & = e^{\frac{1}{2}\omega^{\alpha \beta}S_{\alpha \beta}} \\
  & \rightarrow \text{Son reales}
\end{align*}
En general un spinor de Dirac tiene componentes complejos, pero en la base de Majorana se puede pedir de forma invariante de Lorentz que el spinor sea real
\begin{equation}
  \Psi*(x) = \Psi(x)
 \end{equation}
Los spinores que satisfacen esta condición de realidad son llamados spinores de Majorana \footnote{Tambíen es posible pedir que un spinor sea spinor de Majorana-Weyl, sin embargo, en dimensión 4, es imposible}. \\
\textbf{Matriz de conjugación de Carga:} La matriz de conjugación de carga será una operación que al aplicar a un spinor nos devuelve otro spinor tal que
\begin{align*}
  \Psi \longrightarrow \text{Conj. de cagra} \longrightarrow \Psi^{(c)} \\
\Psi^{(c)}(x) = C\Psi*(x), \;\text{Con C}:=\text{Matriz de conjugación de carga}, \\
  \text{tal que} \rightarrow C^\dagger C = I \quad \text{y}, \\
  C^\dagger \gamma^\mu C = - \left( \gamma^\mu \right)*
\end{align*}
Ahora, la matriz de conjugación de carga en las bases de Majorana y Weyl es
\begin{align*}
  C_{Maj} & = I \\
  C_{Weyl} & = i \gamma^2 = \begin{pmatrix}
    0 & i\sigma^2 \\ -i\sigma^2 & 0
  \end{pmatrix}
\end{align*}
En la base de Majorana un spinor de Majorana satisface que, en una base genérica, un spinor de Majorana será igual a su conjugado de carga.
\begin{equation}
  \Psi^{(c)} = C \Psi*(x) = \Psi(x)
 \end{equation}
Lo que se llama, condición de realidad en una base genérica. \\
¿ Cómo luce un spinor de Majorana en la base de Weyl?
\begin{align*}
  \begin{pmatrix}
    u_+(x) \\ u_-(x)
  \end{pmatrix} = \Psi(x) = C\Psi*(x) = \begin{pmatrix}
    0 & i\sigma^2 \\ -i\sigma^2 & 0
  \end{pmatrix} 
  \begin{pmatrix}
    u*_+(x) \\ u_-*(x)
  \end{pmatrix} = 
  \begin{pmatrix}
    i\sigma^2 u*_(x) \\ -i\sigma^2 u*_+(x)
  \end{pmatrix}
\end{align*}
Con lo cual tenemos la siguiente condición
\begin{align*}
  u_+(x) & = i\sigma^2u_-*(x) \\
  u_-(x) & = -i\sigma^2 u_+*(x)
  \Psi(x) = \begin{pmatrix}
    u_+(x) \\ -i\sigma^2 u*_+(x)
  \end{pmatrix}
\end{align*}
En donde $\Psi(x)$ es un spinor de Majorana en la base de Weyl. \\
En dimensión 4 (3+1) no existen lo spinores de Majorana-Weyl. \\
Si $\Psi(x)$ satisface la ecuación de Dirac, entonces $\Psi^{(c)}$ tambíen satisface la ecuación de Dirac. \\
La ecuación de Dirac en presencia en presencia de un campo $A_\mu=(\phi,\vec{A})$ electromagnético es:
\begin{align}
  \left( i\gamma^\nu \left( \partial_\mu - ieA_\mu \right) - m \right)\Psi(x) & = 0 \\
  \left( i\gamma^\mu \partial_\mu - m +e\gamma^\mu A_\mu \right)\Psi(x) & =
\end{align}
Entonces, la antipartícula, carga opuesta, tambíen satisface la ecuación de conjuado de carga de Dirac, o sea,
\begin{equation}
  \left( i\gamma^\mu \partial_\mu - m -e\gamma^\mu A_\mu   \right)\Psi^{(c)}(x) = 0
 \end{equation}
\end{document}
