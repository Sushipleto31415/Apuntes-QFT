\documentclass[../main.tex]{subfiles}
\begin{document}
\section{Decimocuarta clase}

Se estudiaron las propiedades de la Relatividad Especial lo que nos llevó a estudiar la estructura del grupo de Lorentz. Con ello se está en posición de estudiar Teoría Clásica de Campos y es importante tener en mente hacia dónde se va con ello. El primer punto en Teoría Clásica de Campos es el cuantizar el campo electromagnético. Pero ¿Qué significa el cuantizar el campo electromagnético? ¿Cómo interactúan los fotones con la materia? Antes de ello, es importante el cuantizar un campo algo más simple que el campo electromagnético, que corresponde al campo escalar. \\

Para ello, se piensa en un observador $K$ y un observador $\tilde{K}$ los cuales le dan etiquetas al espacio tiempo de $x^\mu=(t,\vec{x})$ y $\tilde{x}^\mu = (\tilde{t},\tilde{\vec{x}})$ respectivamente, cuya transformación entre sí es conocida y corresponde a las transformaciones de Lorentz. Ahora, los observadores también pueden observar un campo eléctrico y magnético, cuyas etiquetas serán $\vec{E},\vec{B}$ y $\tilde{\vec{E}},\tilde{\vec{B}}$ respectivamente. Pero, ¿cómo encontramos la regla de transformación entre los campos eléctrico y magnético según cada observador? \\

\textbf{Según K:}
\begin{align*}
  \nabla \cdot \vec{E} & = 0 \\
  \nabla \times \vec{E} & = -\partial_t \vec{B} \\
  \nabla \cdot \vec{B} & = 0 \\
  \nabla \times \vec{B} & = \partial_t \vec{E}
\end{align*}

\textbf{Según $\tilde{K}$:}
\begin{align*}
  \tilde{\nabla}\cdot \tilde{\vec{E}} & = 0 \\
  \tilde{\nabla} \times \tilde{\vec{E}} & = -\partial_{\tilde{t}}\tilde{\vec{B}} \\
  \tilde{\nabla} \cdot \tilde{\vec{B}} & = 0 \\
  \tilde{\nabla} \times \tilde{\vec{B}} & = \partial_{\tilde{t}}\tilde{\vec{E}}  % Fixed last equation
\end{align*}

Las ecuaciones de Maxwell vienen de un principio de acción:
\begin{equation}
  I_{\text{Maxwell}} = k\int dt L = k \int dt \int d^3 x  \mathfrak{L} 
\end{equation}
donde la densidad lagrangiana es
\begin{equation}
  \mathfrak{L} = |\vec{E}|^2 - |\vec{B}|^2  % Fixed missing argument
\end{equation}

¿Cuál es la acción invariante de Lorentz y de traslaciones espacio-temporales para un campo escalar? \\
Un campo escalar de Lorentz es una asignación de un número a cada punto del espacio-tiempo de forma que:
\begin{equation*}
  \tilde{\phi} (\tilde{x}) = \phi(x)
\end{equation*}
El observador $K$:
\begin{equation*}
  \phi(x^\mu) = \phi(t,\vec{x}) = \phi_{\vec{x}}(t)
\end{equation*}
una función que depende del tiempo a cada punto del espacio, lo que nos lleva a pensar en los grados de libertad que puede poseer dicha cantidad. El campo escalar tiene tantos grados de libertad como puntos hay en el espacio, o sea, infinitos. \\

La teoría clásica de campos describe un conjunto infinito de grados de libertad. ¿Cuánto vale el lagrangiano que lleva a las ecuaciones de Klein-Gordon?
\begin{equation}
  I_{\text{KG}}[\phi(t,\vec{x})] = \int_{t_1}^{t_2} dt \int_{-\infty}^{\infty} d^3 x  \mathfrak{L}_{\text{KG}}  % Fixed syntax
\end{equation}

Si bien no lo es, cuando se refiere al lagrangiano en teoría de campos, en verdad se está refiriendo a la densidad lagrangiana. Ahora, si tomo una variación infinitesimal en la acción de Klein-Gordon:
\begin{equation*}
  I[\phi+\delta \phi] - I[\phi] := \delta I \stackrel{!}{=} 0 
\end{equation*}
Se busca que de esa condición se obtenga la ecuación de Klein-Gordon ($c=\hbar=1$):
\begin{equation}
  \partial_\mu\partial^\mu \phi = 0
\end{equation}
(Klein-Gordon sin masa, con signatura (+,-,-,-)). Fijamos condiciones de borde:
\begin{align*}
  \delta \phi (t_2,\vec{x}) & = 0 \\
  \delta \phi (t_1,\vec{x}) & = 0
\end{align*}
El lagrangiano de Klein-Gordon es:
\begin{equation}
  \mathfrak{L} = \frac{1}{2}(\partial_t \phi)^2 - \frac{1}{2}(\nabla \phi)^2  % Generalized to 3D
\end{equation}
La acción:
\begin{equation}
  I = \int_{t_1}^{t_2}dt \int d^3x \left[ \frac{1}{2}(\partial_t\phi)^2 - \frac{1}{2}(\nabla\phi)^2 \right]
\end{equation}
Variamos la acción:
\begin{align*}
  \delta I & = I[\phi+\delta\phi] - I[\phi] \\
  & = \int dt \int d^3 x \left[ \frac{1}{2}(\partial_t\phi + \partial_t\delta\phi)^2 - \frac{1}{2}|\nabla\phi + \nabla\delta\phi|^2 \right] - \frac{1}{2}\int dt \int d^3x \left( (\partial_t\phi)^2 - |\nabla\phi|^2 \right) \\
  & = \int dt \int d^3x \left[ \partial_t \phi  \partial_t \delta \phi - \nabla\phi \cdot \nabla \delta \phi \right] + \mathcal{O}(\delta\phi^2) \\
  & = \int dt \int d^3x \left[ \partial_t (\partial_t \phi  \delta \phi) - (\partial_t^2 \phi) \delta \phi - \nabla \cdot (\nabla\phi  \delta \phi) + (\nabla^2 \phi) \delta \phi \right] \\
  & = \left[ \int d^3x  \partial_t \phi  \delta \phi \right]_{t_1}^{t_2} - \int_{t_1}^{t_2} dt \int d^3x  (\partial_t^2 \phi - \nabla^2 \phi) \delta \phi 
\end{align*}
El primer término se anula por condiciones de borde temporales. Considerando condiciones de borde espaciales $\delta\phi \to 0$ cuando $|\vec{x}| \to \infty$:
\begin{align*}
  \delta I = - \int_{t_1}^{t_2} dt \int d^3x  (\partial_t^2 \phi - \nabla^2 \phi) \delta \phi = 0 
\end{align*}
Como $\delta\phi$ es arbitrario:
\begin{equation*}
  \partial_t^2 \phi - \nabla^2 \phi = 0
\end{equation*}
Queda de tarea variar la acción de Klein-Gordon con masa:
\begin{equation}
  \mathfrak{L} = \frac{1}{2}(\partial_t\phi)^2 - \frac{1}{2}(\nabla\phi)^2 -\frac{m^2}{2}\phi^2  % Fixed mass term
\end{equation}

¿Qué ganamos al haber encontrado un principio de acción? Podemos asociar nociones de:
\begin{itemize}
  \item Energía
  \item Momento lineal
  \item Momento angular 
\end{itemize}
vía teorema de Noether. \\

\textbf{Teorema de Noether para campo escalar:}
La acción:
\begin{align*}
  I & = \int dt \int d^3x  \mathfrak{L}(\partial_t \phi, \nabla \phi, \phi) \\
  & = \int d^4x  \mathfrak{L}(\partial_\mu \phi, \phi)
\end{align*}
Supongamos que existe una variación infinitesimal $\delta \phi$ tal que:
\begin{align*}
  \delta I = \int d^4x  \partial_\mu B^\mu \\
  \partial_\mu B^\mu = \partial_t B^t + \nabla \cdot \vec{B}
\end{align*}
La variación de la acción es:
\begin{align*}
  \delta I & = \int d^4x \left[ \frac{\partial\mathfrak{L}}{\partial \phi}\delta \phi + \frac{\partial \mathfrak{L}}{\partial (\partial_\mu \phi)} \partial_\mu (\delta \phi) \right] \\
  & = \int d^4x \left[ \left( \frac{\partial \mathfrak{L}}{\partial\phi} - \partial_\mu \frac{\partial \mathfrak{L}}{\partial(\partial_\mu\phi)} \right) \delta \phi + \partial_\mu \left( \frac{\partial \mathfrak{L}}{\partial(\partial_\mu \phi)} \delta\phi \right) \right]
\end{align*}
Usando las ecuaciones de movimiento y la quasi-invariancia:
\begin{equation}
  \partial_\mu \left( \frac{\partial \mathfrak{L}}{\partial(\partial_\mu \phi)}\delta\phi - B^\mu \right) = 0
\end{equation}
Esto da una corriente conservada:
\begin{equation}
  j^\mu = \frac{\partial \mathfrak{L}}{\partial(\partial_\mu \phi)}\delta\phi - B^\mu, \quad \partial_\mu j^\mu = 0
\end{equation}
Integrando en el espacio:
\begin{align*}
  \int d^3x  \partial_t j^t + \int d^3x  \nabla \cdot \vec{j} & = 0 \\
  \frac{dQ}{dt} + \oint_{\partial V} d\vec{S} \cdot \vec{j} & = 0
\end{align*}
donde $Q = \int d^3x  j^t$. Si el flujo en el infinito es cero, $Q$ se conserva.

\end{document}
