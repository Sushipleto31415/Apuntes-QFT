\documentclass[../main.tex]{subfiles}

\begin{document}
\section{Vigesima segunda clase}
Sabemos que
\begin{align*}
  \Psi(x) \rightarrow \tilde{\Psi}(x) = D[\Lambda] \Psi(\Lambda^{-1}) \\
  \Psi^\dagger (x) \gamma^0 = : \bar{\Psi}(x) \rightarrow \tilde{\bar{\Psi}}(x) = \bar{\Psi}(\Lambda^{-1}x) D^{-1}[\Lambda] \\
  D[\Lambda] = e^{\frac{1}{2}\omega^{\alpha \beta}S_{\alpha \beta}}
\end{align*}
de lo cual se concluyó que
\begin{align*}
  \bar{\Psi}(x)\Psi(x) , \quad \text{es un escalar de Lorentz} \\
  \bar{\Psi}(x) \gamma^\mu \Psi(x), \quad \text{es un vector de Lorentz}
\end{align*}
El primero corresponde al término de masa y el segundo es una corriente vectorial. \\
Con estos ingredientes demostramos que la acción de Dirac es quasi-invariante bajo Lorentz:
\begin{equation*}
  I[\Psi,\bar{\Psi}]= \int d^4x \bar{\Psi}(x) \left(i\gamma^\mu \partial_\mu - m \right)\Psi(x)
\end{equation*}
Todo los observadores inerciales estarán de acuerdo en cuál es la ecuación dinámica para un spinor de Dirac.
\begin{equation*}
  \delta_{\bar{\Psi}} I = 0 \Rightarrow \left( i\gamma^\mu \partial_\mu - m  \right)\Psi(x) = 0, \quad \text{Ecuación de Dirac}
\end{equation*}
Ahora, la ecuación para la variación con $\Psi$
\begin{align*}
  \delta_{\Psi} I & = I[\Psi+ \delta \Psi , \bar{\Psi}] - I [\Psi,\bar{\Psi}] \\
  & = \int d^4x \left( \bar{\Psi} i\gamma^\mu \partial_\mu \delta \Psi - m \bar{\Psi} \delta \Psi \right) \\
  & = \int d^4x \left( \partial_\mu \left( \bar{\Psi} i \gama^\mu \delta\Psi \right) - \partial_\mu \bar{\Psi} i \gamma^\mu \delta \Psi - m \bar{\Psi} \delta \Psi\right) \\
  & = \cancel{B.T.} - \int d^4x \left( i \partial_\mu \bar{\Psi} \gamma^\mu + m \bar{\Psi} \right)\delta \Psi  = 0 \\
\end{align*}
Con lo cual, hemos obtenido
\begin{equation}
  i\partial_\mu \bar{\Psi} \gamma^\mu + m\bar{\Psi} = 0  
 \end{equation}
\textbf{Afirmación:} La ecuación obtenida al calcular es el conjudado de Dirac de la ecuación de Dirac. \\
\textbf{Demostración:}  \\
Recuerdo: 
\begin{equation*}
  I[\varphi, \varphi*] = \int d^4 x \left( \partial_\mu \varphi \partial^\mu \varphi* - m^2\varphi \varphi* \right)
\end{equation*}
En donde $\varphi$ es un campo escalar complejo, el cual cumple lo siguiente
\begin{align*}
  \delta_\varphi I & = 0 \Rightarrow \square \varphi* + m^2 \varphi* = 0 \\
  \delta_{\varphi*} I & 0 \Rightarrow \square \varphi + m^2 \varphi = 0
\end{align*}
Comencemos ocn la ecuación de Dirac, tal que
\begin{align*}
  i \gamma^\mu \partial_\mu \Psi - m \Psi & = 0, \quad /()^\dagger \\
  -i\partial_\mu \Psi^\dagger \gamma^{\mu \dagger} - m \Psi^\dagger & = 0 \\
  i \partial_\mu \left( \Psi^\dagger \gamma^{\mu \dagger} \gamma^0 \right) + m\Psi^\dagger \gamma^0 & = 0, \quad \text{pero}\; \gamma^{\mu \dagger} \gamma^0 = \gamma^0 \gamma^\mu \\
  i\partial_\mu \left( \Psi^\dagger \gamma^0 \right)\gamma^\mu + m\Psi^\dagger \gamma^0 & = 0 \\
  i \partial_\mu \bar{\Psi}\gamma^\mu + m\bar{\Psi} = 0
\end{align*}
En donde hemos obtenido la misma ecuación que antes, con lo cual bajo un espacio de Hilbert, podemos definir
\begin{align*}
  \hat{H} \ket{\vec{k}} & = \sqrt{\vec{k}^2 + m^2} \ket{\vec{k}} \\
  \hat{\vec{p}}\ket{\vec{h}} & = \vec{k} \ket{\vec{h}}
\end{align*}
Notemos que el que haya un vector dentro de un ket es puramente notación para la ecuación de onda. \\
Ahora bien, el término de masa que se encuentra en la ecuación de Dirac corresponde a la masa de los quantos que mide el campo en particular, masa la cual puede ser medida por ejemplo con el efecto de Compton. \\
\begin{align}
  i\gamma^\mu \partial_\mu  - m \Psi = 0 \\
  \lambda_F   = \lambda_I + \frac{h}{mc} \left( 1-\cos{\theta} \right)
\end{align}
Entonces, cuando veamos una acción sin términos cuadrados, podemos concluir que el quanta del campo no tendrá masa, como por ejemplo la acción de Maxwell.
\begin{align*}
  I_{Max} [A_\mu] & = \int d^4x - \frac{1}{4}F_{\mu \nu}F^{\mu \nu} + e \bar{\Psi}\gamma^\mu A_\mu \Psi \\
  & = \frac{-1}{4}\ind d^4 x \left( \partial_\mu A_\nu - \partial_\nu A_\mu \right) \left( \partial^\mu A^\nu - \partial^\nu A^\mu \right) 
 \end{align*}
 Entonces el segundo término será un término de interacción de los campos, pero no de masa, y que finalmente,
 \begin{align*}
   \partial_\mu F^{\mu \nu} & = j^\nu \rightarrow \nabla \cdot \vec{E} = \rho \\
   & = e\bar{\Psi}\gamma^\nu \Psi \rightarrow \nabla \times \vec{B} = \partial_t\vec{E} + \vec{J}
 \end{align*}
 \textbf{Afirmación:} Cada componente del spinor de Dirac satisface la ecuación de Klein-Gordon.
 \begin{align*}
   \left( i\gamma^\mu \partial_\mmu - m \right)\Psi = i\gamma^0 \partial_t\Psi +i\gamma^1 \partial_x\Psi + i\gamma^2\Psi_y + i\gamma^3\partial_z \Psi - m \Psi & = 0\\
   i \begin{pmatrix}
     0 & 0 & 1 & 0 \\ 0 & 0 & 0 & 1 \\ 1 & 0 & 0 & 0 \\ 0 & 1 & 0 & 0
   \end{pmatrix}
   \begin{pmatrix}
     \partial_t \Psi_1 \\ \partial_x \Psi_2 \\ \partial_y \Psi_3 \\ \partial_z \Psi_4
   \end{pmatrix} + \dots - m \begin{pmatrix}
     \Psi_1 \\ \Psi_2 \\ \Psi_3 \\ \Psi_4
   \end{pmatrix} = \begin{pmatrix}
     0 \\ 0 \\ 0 \\ 0
   \end{pmatrix}
 \end{align*}
 Lo cual son cuatro ecuaciones difereciales lineales acopladas
 \begin{align*}
   i\partial_t \Psi_3 + \dots - m \Psi_1 & = 0 \\
   i\partial_t \Psi_4 + \dots - m \Psi_2 & = 0 \\
  i\partial_t\Psi_1 + \dots - m \Psi_3 & = 0 \\
   i\partial_t \Psi_2 + \dots -m \Psi_4 & = 0
 \end{align*}
 \textbf{Demostración:}
 \begin{align*}
   i\gamma^\mu \partial_\mu \Psi - m \Psi & = 0 ,\quad /\left( i\gamma^\mu \partial_\mu + m \right) \\
   \left( i\gamma^\mu \partial_\mu + m \right)\left( i\gamma^\mu \partial_\mu \Psi - m \Psi \right) & = 0 \\
   \gamma^\nu \gamma^\mu \partial_\mu \partial_\mu \Psi + m^2\Psi & = 0 \\
   \left( \frac{1}{2}\left( \gamma^\nu \gamma^\mu + \gamma^\mu \gamma^\nu  \right) + \frac{1}{2} \left( \gamma^\nu \gamma^\mu - \gamma^\mu \gamma^\nu \right) \right)\partial_\nu \partial_\mu \Psi + m^2 \Psi & = 0 \\
   \frac{1}{2} 2 \eta^{\mu \nu} \partial_\nu \partial_\mu \Psi + m^2 \Psi & = 0 \\
   \square \Psi + m^2 \Psi & = 0
 \end{align*}
 A lo cual tendremos que, los $\Psi$ que sean soluciones a la ecuación de Dirac serán soluciones de la ecuación de Klein-Gordon, pero no viceversa. \\
 \textbf{Spinores quiarales:} $\{\gamma^\mu , \gamma^nu\} = 2\eta^{\mu \nu} 1_4$. \\
 En la base quiral:
 \begin{equation*}
   \gamma^0 = \begin{pmatrix}
     0 & I \\ I & 0
   \end{pmatrix}, \quad \gamma^i = \begin{pmatrix}
     0 & \sigma^i \\ -\sigma^i & 0
   \end{pmatrix}
 \end{equation*}
 En la base de Majorana:
 \begin{align*}
   \gamma^0_{\Maj} & = \begin{pmatrix}
     0 & \sigma^2 \\ \sigma^2 & 0 
   \end{pmatrix}
   ,\quad \gamma^1_{Maj} = \begin{pmatrix}
     i\sigma^3 & 0 \\ 0 & i\sigma^3
   \end{pmatrix} \\
   \gamma^2_{Maj} & = \begin{pmatrix}
     0 & -\sigma^2 \\ \sigma^2 & 0
   \end{pmatrix}, \quad \gamma^3_{Maj} = \begin{pmatrix}
     -i\sigma^1 & 0 \\ 0 & -i\sigma^1
   \end{pmatrix}
 \end{align*}
En donde $\sigma $ son las matrices de Pauli
\begin{align*}
  \sigma^1 & = \begin{pmatrix}
    0 & 1 \\ 1 & 0 
  \end{pmatrix}
  \\
  \sigma^2 & = \begin{pmatrix}
    0 & -i \\ i & 0
  \end{pmatrix} \\
  \sigma^3 & = \begin{pmatrix}
    1 & 0 \\ 0 & -1
  \end{pmatrix}
\end{align*} 
Pero existe una matriz $A_{4\times 4}$ invertible que
\begin{equation*}
  \gamma^\mu_{Maj} = A \gamma^\mu_{Quiral} A^{-1}
\end{equation*}
En la base quiral: $S_{\alpha \beta} = \frac{1}{4}[\gamma_\alpha , \gamma_\beta]$ dan origen al a transformación de Lorentz de la forma 
\begin{align*}
  e^{\frac{1}{2}\omega^{\alpha \beta}S_{\alpha \beta}} = D[\Lambda_{ROT}] = \begin{pmatrix}
    e^{i\vec{\varphi}\cdot \vec{\sigma}/2} & 0 \\ 0 & e^{i\vec{\varphi}\cdot \vec{\sigma}/2}
  \end{pmatrix}\\
  D[\Lambda_{Boost}] = \begin{pmatrix}
    e^{\vec{\chi}\cdot\vec{\sigma}/2} & 0 \\ 0 & e^{-\vec{\chi}\cdot /2}
  \end{pmatrix}
\end{align*}
En donde $\vec{\chi}$ sabe acerca de la rapidez y dirección del boost, tal que
\begin{equation*}
  \tilde{\Psi}(x) = D[\Lambda_{Boost}] \Psi(\Lambda^{-1})
\end{equation*}
Existen campos de dos componentes que sienten las transformaciones de Lorentz, tales campos se llaman spinores de Weyl. \\
En la base quiral:
\begin{align*}
  \begin{pmatrix}
    \Psi_1 \\ \Psi_2 \\ \Psi_3 \\ \Psi_4
  \end{pmatrix} = \begin{pmatrix}
    u_+ \\ u_-
  \end{pmatrix}
\end{align*}
En donde $u_+$ es el spinor de Weyl izquierdo y $u_-$ es el spinor de Weyl derecho. \\
¿Cómo transforma bajo boosts y bajo rotacione un spinor de Weyl izquierdo?
\begin{align*}
  \text{Rot:} \quad u_+(x) \rightarrow \tilde{u}_+(x) & = e^{\vec{\varphi}\cdot \vec{\sigma}/2} u_+(\Lamnda^{-1}x) \\
  \text{Boosts:} \quad u_+(x) \rightarrow \tilde{u}_+(x) & = e^{\vec{\chi}\cdot / 2} u_+(\Lambda^{-1}x)
\end{align*}
Pero como sería para Weyl derecho? \\
$u_+(x)$ transforma en la representación $(1/2,0)$? , y lo mismo para $u_-$? \\
¿Cómo transforma la acción de Dirac en términos de spinores izquierdos y derechos?
\begin{equation*}
  \mathfrak{L}  = \bar{\Psi} \left( i\gamma^\mu \partial_\mu - m \right)\Psi = iu^\dagger_- \sigma^\mu \partial_\mu u_+ + iu^\dagger_- i+ i u^\dagger_+ \bar{\sigma}^\mu \partial_\mu u_+ - m \left( u^+_+ u_- + u^\dagger_- u_+  \right) = 0
\end{equation*}
En donde $\sigma^0=1$ y el resto son las matrices de Pauli. \\
Si $m=0$, las ecuaciónes de Euler-Lagrange para $u_-$ y $u_+$ son
\begin{align*}
  i\sigma^\mu \partial_\mu u_- & = 0 \\
  i\sigma^\mu \partial_\mu u_+ & = 0
\end{align*}
Las cuales son llamadas las ecuaciones de Weyl. \\
Estas ecuaciones definen de manera consistente la evolución temporal de Fermiones quirales
\begin{align*}
  \Psi & = \begin{pmatrix}
    u_+(x) \\ 0 \\ 0
  \end{pmatrix} \\
  \Psi & = \begin{pmatrix}
    0 \\ 0 \\ u_-(x) 
  \end{pmatrix}
\end{align*}
Ahora, la física no puede depender de la base, con lo cual es necesario además definir todo esto para una base genérica.
\end{document}
