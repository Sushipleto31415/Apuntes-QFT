\documentclass[../main.tex]{subfiles}

\begin{document}
\section{Propuestos y resueltos}
Aquí van ejercicios varios del capítulo completo.\\
 He aquí, los operadores diferenciales clásicos en coordenadas Oblatas Esferoidales
\begin{align}
     \vec{\nabla} f(\emph{v},\phi,\theta) & =\frac{1}{a\sqrt{\sinh{\emph{v}}^2+\sin{\theta}^2}}\left[ \frac{\partial f}{\partial v}\hat{v}+ \frac{\partial f}{\partial \theta} \hat{\theta} \right]+\frac{1}{a\cosh{\emph{v}}\cos{\theta}}\frac{\partial f}{\partial \phi}\hat{\phi} \\
\vec{\nabla} \cdot \vec{f}(v, \phi, \theta) & = \frac{1}{ a\left( \sinh v^2 + \sin{\theta}^2\right)} \left\{ \frac{1}{\cosh v} \frac{\partial}{\partial v} \left[\sqrt{ (\sinh v^2 + \sin v^2)} \cosh{\emph{v}} f^v \right] \right. \dots \nonumber \\
&\quad+ \left. \frac{1}{\cos \theta} \frac{\partial}{\partial \theta} \left[ \sqrt{(\sinh v^2 + \sin \theta^2)} \cos{\theta} f^\theta \right] \right\}
+ \frac{1}{(a\cosh v \cos \theta)} \frac{\partial f^\phi}{\partial \phi} \\ 
    \vec{\nabla}^2 f(\emph{v},\phi,\theta) & = \frac{1}{a^2(\sinh{\emph{v}^2+\sin{\theta}^2})} \left[ \frac{1}{\cosh{\emph{v}}} \frac{\partial }{\partial \emph{v}}\left(  
     \cosh{\emph{v}}\frac{\partial f}{\partial \emph{v}} \right) + \frac{1}{\cos{\theta}} \frac{\partial}{\partial \theta} \left( 
      \cos{\theta} \frac{\partial f}{\partial \theta}  \right)\right] \\
      & + \frac{1}{a^2(\cosh{\emph{v}}^2\cos{\theta}^2)}\frac{\partial^2f}{\partial\phi^2}
\end{align}
\end{document}