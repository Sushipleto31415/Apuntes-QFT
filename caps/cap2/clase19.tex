\documentclass[../main.tex]{subfiles}
\begin{document}
\section{Decimonovena clase}
Se habló acerca de spinores de Dirac, los cuales relacionan un campo medido por dos obvseradores tal que
\begin{equation*}
  \tilde{\Psi}_a(x) = \left( e^{\frac{1}{2}\omega^{\mu \nu}S_{\mu \nu}} \right)\Psi(\Lambda^{-1}x)
\end{equation*}
En donde
\begin{equation*}
  S_{\mu \nu} = \frac{1}{4}\left[\gamma_{\mu} , \gamma_\nu  \right]
\end{equation*}
En donde el conmutador da $\left[ \gamma_\mu , \gamma_\nu \right] = 2\eta_{\mu \nu}I_4$
\textbf{En la base chiral o base de Weyl:}
Con $\omega_{ij} = -\epsilon_{ijk}\varphi^k$
\begin{equation*}
  \left( e^{\frac{1}{2}\omega^{\mu \nu}S_{\mu \nu}} \right_{Rot})  = \begin{pmatrix}
    e^{\frac{i}{2}\vec{\varphi}\cdot \vec{\sigma}} & O_{2\times 2} \\ O_{2\times 2} & e^{\frac{i}{2}\vec{\varphi}\cdot \vec{\sigma}}
  \end{pmatrix}
\end{equation*}
Cuando elijo $\omega$ tal que esta transformación de Lorentz implementa un boost , con
\begin{equation*}
  \left( e^{\frac{1}{2}\omega^{\mu \nu}S_{\mu \nu}} \right_{Boost}) = \begin{pmatrix}
    e^{\frac{1}{2}\vec{\chi}\cdot \vec{\sigma}} & O_{2\times 2} \\ O_{2\times 2} & e^{-\frac{1}{2}\vec{\chi}\cdot\vec{\sigma}}
  \end{pmatrix}
\end{equation*}
En donde $\omega_{0i}=-\chi_i$. \\
Cuando se realiza una rotación en el plano (x,y) en $2\pi$ se tiene lo siguiente
\begin{equation*}
  \left( e^{\frac{1}{2}\omega^{\mu \nu}S_{\mu \nu}} \right)_{Rot (x,y) 2\pi} = -\begin{pmatrix}
    1 & 0 & 0 & 0 \\ 0 & 1 & 0 & 0 \\ 0 & 0 & 1 & 0 \\ 0 & 0 & 0 & 0 & 1
  \end{pmatrix}
\end{equation*}
Es algo como, el spinor, cuando cambiamos de observador, devuelve menos el spinor
\begin{equation*}
  \tilde{\Psi}_a(\tilde{x}) = -\Psi_a(x) \Psi_a(x)
\end{equation*}
Lo que es algo como una cinta de Moebius, la cual solo tiene una única superficie. \\
Podemos pensar los coeficientes de la exponencial como los coeficientes de un grupo actuando sobre el spinor, tal que $\omega^{\mu \nu}S_{\mu \nu}$ son lo componentes del espacio vectorial con su base respectiva. \\
En general, los spinores $\Psi_a(x)$ son números complejos. Los $\Psi(x)$ son variales de Grassman complejas. Para convencernos de esto, cuantizaremos la teoría de Dirac y luego tomaremos el límite clásico.  \\
Un spinor de Dirac está compuesto por dos spinores de dos componentes que no se mezclan entre ellos bajo una transformación de Lorentz.
\begin{equation*}
  \Psi_a(x) = \begin{pmatrix}
    u(x) \\ v(x)
  \end{pmatrix}
\end{equation*}
Y veremos que los $\Psi_a(x)$ transforman en la representación $\left( 1/2,0 \right)\oplus\left( 0,1/2 \right)$  del álgebra de Lorentz. En donde $u(x)$ es llamado spinor izquierdo y $v(x)$ es llamado spinor derecho. \\
Las rotaciones están implementadas por matrices unitarias. 
\begin{equation}
  \left( e^{\frac{1}{2}\omega^{\mu \nu}S_{\mu \nu}} \right)^\dagger_{Rot} \left( e^{+\frac{1}{2}\omega^{\mu \nu}S_{\mu \nu}} \right)_{Rot} = I
 \end{equation}
 Hagamos el cálculo
 \begin{align*}
   \begin{pmatrix}
     e^{\frac{-i}{2}\vec{\varphi}\cdot\vec{\sigma}^\dagget} & 0 \\ 0 & e^{\frac{-i}{2}\vec{\varphi}\cdot\vec{\sigma}}
   \end{pmatrix}
   \begin{pmatrix}
     e^{\frac{i}{2}\vec{\varphi}\cdot \vec{\sigma}} & 0 \\  0 & e^{\frac{i}{2}\vec{\varphi}\vec{\sigma}} 
   \end{pmatrix} 
   & = 
   \begin{pmatrix}
       e^{-\frac{i}{2}\vec{\varphi}\cdot \vec{\sigma}} & 0 \\  0 & e^{\frac{-i}{2}\vec{\varphi}\cdot{\vec{\sigma}}} 
     \end{pmatrix} 
     \begin{pmatrix}
       e^{\frac{i}{2}\vec{\varphi}\cdot\vec{\sigma}} & 0 \\ 0 &  e^{\frac{i}{2}\vec{\varphi}\cdot \vec{\sigma}}
     \end{pmatrix} \\
  & = \begin{pmatrix}
    I_2 & 0_2 \\ 0_2 & I_2
  \end{pmatrix} \\
   & = I_4
 \end{align*}
Las matrices que implementan lo boosts no son unitarias. \\
¿Tensión con la mecánica cuántica? \\
Pues la unitaridad es fundamental en mecánica cuántica para que no se pierda la movilidad. \\
No hay tensión pues veremos que la representación del grupo de Lorentz que actúa sobre $\mathfrak{H}(\varphi)$ el espacio de Hilbert del sistema cuántico  spinor de Dircac, espacio de infinita dimensión. \\
Los grupos no compactos no tienen representaciones unitarias finito dimensionales. 
\begin{equation}
  g(\varphi_1,\varphi_2,\varphi_3,\chi_1,\chi_2,\chi_3)
 \end{equation}
 En donde $\varphi$ son las rotaciones en donde cada $\varphi$ vive entre $0\leq \varphi <2\pi$ y $\chi$  es la rapidity, con $-\infty < \chi_1 <\infty $ la cual cumple con la relación $v_x=c\tanh \chi_1$, con $-c<v_x<c$ Notemos que esta velocidad corresponde a la velocidad relativa entre dos observadores incerciales, con lo cual no hay observadores que viajan a la velocidad de la luz. \\
El grupo de las rotaciones es $SO(3)$, tal que $SO(3)S^3/Z_2$. Ahora el álgebra 
\begin{align*}
  so(3)~ su(2) \rightarrow SU(2) = S^3
\end{align*}
Lo cual sí correpsonde a una 3-esfera. $U\in SU(2)$ tal que
\begin{align*}
  U = \begin{pmatrix}
    a+ib & c+id \\ -c+id & a-ib
  \end{pmatrix}
\end{align*}
Con,
\begin{equation*}
  a^2+b^2+c^2+d^2 = 1
\end{equation*}
y las matrices $U$ son unitarias, con lo cual $UU^\dagger=1$. Sea $O_{ij}\in SO(3)$, tal que 
\begin{equation*}
  O_{ij} = \frac{1}{2}tr \left( A^\dagger \sigma_i A \sigma_j \right)
\end{equation*}
En donde $A\in SU(2)$. Notemos que si en $U$ yo tomo cada número con su negativo, aparezco en el punto opuesto de la 3-esfera, punto el cual es llamado la antípoda. \\
¿ Cómo evoluciona el campo de Dirac en el tiempo?\\
Con ello la pregunta se refiere a ¿ Cuál es el principio de acción cuasi-invariante bajo: 
\begin{itemize}
  \item Rotaciones (Isotropía del espacio)
  \item Boosts (Equivalencia de los observadores inerciales)
  \item Traslaciones espaciotemporales (Homogeneidad del espacio-tiempo)
\end{itemize}
Para el campo de Dirac? \\
Bajo Lorentz: 
\begin{equation*}
  \tilde{\Psi}_a(x) = \left( e^{\frac{1}{2}\omega^{\mu \nu}S_{\mu \nu}} \right)_{ab} \Psi_b(\Lambda^{-1}x) 
\end{equation*}
Bajo una traslación espacio-temporal $x^\mu\rightarrow \tilde{x}^\mu = x^\mu - a^\mu$
\begin{equation*}
  \tilde{\Psi}_a(\tilde{x}) \Psi_b(x) \rightarrow \tilde{\Psi}_a(x) = \Psi_a(x+a)
\end{equation*}
Notemos que esto es 4 veces la transformación bajo traslación para un campo escalar en la representación trivial del grupo de Lorentz.  \\
El grupo de Poincaré = Lorentz y Traslaciones espacio-temporales. LLámele los generadores $P_{\mu \nu}$ y $P_\mu$ respectivamente, tal que
 \begin{align*}
   \left[ J , J \right] & = \eta J - \eta J + \eta J - \eta J \\
   \left[ J , P \right] & = \eta P - \eta P \\
   \left[ P_\mu , P_\nu \right] & = 0
 \end{align*} 
 \textbf{Campo escalar}: Bajo Lorentz $x^\mu \rightarrow \tilde{x}^\mu = \Lambda^\mu_\nu x^\nu$
 \begin{equation*}
   \tilde{\phi}(\tilde{x}) = e^{\frac{1}{2}\omega^{\mu \nu}J_{\mu \nu}}\phi(x) ,\; \text{Con}\; J_{\mu \nu} = O_{1\times 1} \Rightarrow \tilde{\phi}(\tilde{x}) = \phi(x) \Rightarrow \tilde{\phi}(x) = \phi(\Lambda^{-1})
 \end{equation*}
 Traslaciones $x^\mu \rightarrow \tilde{x}^\mu = x^\mu + a^\mu$
 \begin{equation*}
   \tilde{\phi}(\tilde{x}) = e^{\frac{1}{2}\varepsilon^\mu P_\mu} \phi(x), \; \text{Con}\; P_\mu = 0 \Rightarrow \tilde{\phi}(\tilde{x}) = \phi(x) \Rightarrow \tilde{\phi}(x) = \phi(x-a)
 \end{equation*}
 \textbf{Campo de Dirac}: Bajo Lorenz $x^\mu \rightarrow \tilde{x}^\mu = \Lambda^\mu_\mu x^\nu$
 \begin{equation*}
   \tilde{\Psi}_a(\tilde{x}) = \left( e^{\frac{1}{2}\omega^{\mu \nu}S_{\mu \nu}} \right)_{ab} \Psi_b(x)
 \end{equation*}
 Traslaciones $\x^\mu \rightarrow \tilde{x}^\mu = x^\mu + a^\mu$
 \begin{equation*}
   \tilde{\Psi}_a(\tilde{x}) = \left( e^{\frac{1}{2}\varepsilon^\mu P_\mu} \right)_{ab} \Psi_b(x)
 \end{equation*}
\textbf{Campo vectorial}: $A^\mu(x) = \left( \phi,\vec{A} \right)$. Bajo Lorentz $x^\mu \rightarrow \tilde{x}^\mu = \Lambda^\mu_\mu x^\nu$
\begin{equation*}
  \tilde{{A}}^\mu (\tilde{x}) = \Lambda^\mu_\nu A^\nu(x) = \left( e^{\frac{1}{2}\omega^{\alpha \beta}J_{\alpha \beta}} \right)^\mu_{\; \nu} A^\nu(x)
\end{equation*}
Con,
\begin{equation*}
   \left( J_{\alpha \beta} \right)^\mu_{\; \nu} = \eta_{\alpha \nu} \delta^\mu_\beta - \eta_{\mu \beta}\delta^\mu_\alpha
\end{equation*}
Traslaciones $x^\mu \rightarrow \tilde{x}^\mu = x^\mu + a^\mu$
\begin{equation*}
      \tilde{{A}}^\mu(\tilde{x}) = \left( e^{\frac{1}{2}\varepsilon^\alpha P_\alpha} \right)^\mu_{\;\nu} A^\nu(x) \Rightarrow \tilde{{A}}^\mu(\tilde{x}) = A^\mu(x)
\end{equation*}
\end{document}

