\documentclass ../main.tex]{subfiles}

\begin{document}



\vspace{3cm}
\chapter{Sumario de principios elementales}
\section*{Clase 2 (20/3/25)}
En la última clase repasamos algo de física de mećanica clásica y vimos una introducción algo general de la necesidad de enontrar un marco conceptuál algo más general de la mecánica clásica ya que necesitamos describir un proceso que ocurre en la naturaleza en el cual el estado inicial no es el mismo del final.
Supongan que tienen una partícula en precencia de una energía potencial $U(x)$ a lo cual la 2da ley de Newton nos dice lo siguiente
$$ m \frac{d^2x(t)}{dt^2}=-\frac{\partial U(x)}{\partial x }$$
A lo cual multiplicamos por $\frac{dX}{dt}$
$$ m \frac{dX}{dt}\frac{d^2x(t)}{dt^2}=-\frac{dX}{dt}\frac{\partial U(x)}{\partial x }$$
Ahora, sacamos la derivada temporal hacia fuera
$$ \frac{d}{dt} \left(  \frac{m}{2} \left( \frac{dx}{dt} \right)^2\right)= -\frac{d}{dt} \left( \frac{\partial x}{\partial x}\right)$$ 
$$\frac{d}{dt} \left(  \frac{m}{2} \left( \frac{dx}{dt} \right)^2+ \left( \frac{\partial x}{\partial x}\right) \right)=0 $$
La combinación
\begin{equation}
    \frac{m}{2}\left( \frac{dx}{dt}\right)^2+U(x)=E
\end{equation}
En lo cual $E$ corresponde a la energía del sistema. $E $ es una constante, con lo cual no depende del tiempo
En general diremos que una cantidad $Q=a(x,\dot(x)$ es conservada si
\begin{equation}
    \frac{d}{dt} Q\left( x(t), \frac{dx(t)}{dt}\right)=0
\end{equation}
En el contexto de la mecánica cláscia en el que estamos interesados en encontrar $x(t)$, las cantidades conservadas con extremadamente útiles. Las cantidades conservadas tambíen se llaman integrales de movimiento.  Si un sistema tiene un número suficientemente alto de integrales de movimiento, entonces podemos encontrar las historias de los grados de libertad sin integrar . \\
\begin{equation}
    q_i=q_i(t) \quad i=1,\dots ,N
\end{equation}
Teorema de Noether: Si el funcional de la acción es quasi-invariante bajo una transformacion infinisetimal, entonces existirá una cantidad conservada asociada a la transformación.  \\
Ejemplos de transformaciones infinitesimales, la transformación infinitesimal tendrá una forma bien precisa, dado lo siguiente. \\
\textbf{Transformación: traslación temporal}. Sea una coordenada $q(t)$ que depende del tiempo, a la cual haremos una tralación al futoro en $a$ seg. $q(t-a)$, notemos que $a $ puede ser cualquier número, digamos que $a$ es infinitesimal, con lo cual lo llamaremos $\epsilon$, ahora, tomando la serie de Taylor a $q(t-\epsilon$ tenemos lo siguiente:
\begin{equation}
    q(t-\epsilon)=q(t) - \epsilon \frac{dq}{dt}+ O(\epsilon^2)
\end{equation}
Para lo cual el tèrmino de $O(\epsilon^2)$ puede ser despreciado ya que será muy pequeño, ahora sigamos
\begin{equation}
    q(t-\epsilon)-q(t) =- \epsilon \frac{dq}{dt}=\delta q
\end{equation}
A $\delta q$ lo llamaremos traslación temporal \\
\textbf{Trasformación: Traslación espacial}
Sea un vector posición $r(t)$ el cual es situado con respecto a un eje coordenado cartesiano al cual lo trasladaremos espacialmente en un vector $a$ con lo cual la posición luego de la traslación será $\vec{r}(t)+\vec{a}$, ahora bien, supongamos que el vector $\vec{a}$ es infinitesimal, con lo cual la llamaremos $\vec{\epsilon}$, asì, la traslacion temporal infinitesimal estará dado por
\begin{equation}
    (\vec{r(t)}+\vec{\epsilon} )- \vec{r(t)}=\vec{\epsilon}=\delta \vec{r}
\end{equation}
En lo cual $\delta \vec{r}$ es llamada traslación espacial. \\
\textbf{Transfomación: Rotación espacial.} \\ Sabemos que en una rotación espacial una cantidad conservada sería el momento angular. Ahora, definamos una rotación.
\begin{align}
    x \prime &  = \cos{\theta}x- \sin{\theta}y \\
    y \prime &  = \sin{\theta}x + \cos{\theta}y
\end{align}
Ahora, en el caso que la rotación fuera infinitesimal, llamaremos $\theta=\epsilon$, con lo cual la rotación definida quedaría dada por
\begin{align}
    x \prime = x- \epsilon y \xrightarrow{} \delta x = x \prime - x = -\epsilon y \\
    y \prime  = \epsilon x + y \xrightarrow{} \delta y = y \prime  - y = \epsilon x
\end{align}
con lo cual obtenemos que
\begin{align}
    \delta x = - \epsilon y \\
    \delta y = \epsilon x \\
    \delta z = 0
\end{align}
Así la rotación espacial según el vector posicion $\vec{r}$ sería 
\begin{equation}
    \delta \vec{r} = \vec{r} \times \delta \hat{\phi}
\end{equation}
Acordar que el producto vectorial solo tiene sentido en 3 y 7 dimensiones. \\
Ahora hablemos de la acción 
\begin{equation}
    S [q(t)] = \int dt L(q,\dot{q})
\end{equation}
Ahora, se define la acciòn quasi-invariante como:
\begin{equation}
    \delta S = S[q+ \delta q] - S [q] = \int dt \frac{dB}{dt}
\end{equation}
B es una función que depende del tiempo. \\
Encontraremos que, en el caso que $B=0$ decimos que la acción es invariante, desarollando obtenemos
\begin{align}
    \delta S &  = \int dt \left( \partial_q L \delta q + \partial_{\dot{q}} L \frac{d}{dt} \delta q \right) \\
    & = \int dt \left(   \partial_q L + \frac{d}{dt} \partial_{\dot{q}} L  \right) \delta q + \int dt \frac{d}{dt} \left(  \partial_{\dot{q}} \delta q\right)
\end{align}
Usando la ecuación de movimiento obtenemos:
\begin{align}
    \delta S & = \int dt  \frac{d}{dt}\left(   \partial_{\dot{q}} \delta q \right) = \int dt \frac{dB}{dt} \\
     & =  \int_{t_1}^{t_2} dt \frac{d}{dt} \left(  \partial_{\dot{q}} L \delta q - B\right) = 0
\end{align}
si usted es capaz de encontrar una transformación que deja la accioón quasi-invariante, entonces la siguente cantidad encontrará que es constante
\begin{equation}
    \partial_{\dot{q}}L \delta q - B = C^{te}
\end{equation}
en lo cual la constante no dependerá del tiempo.
Ahora veamos que sucede cuando usamos una traslación temporal. \\
\textbf{Traslación temporal: }
\begin{equation}
    S[q(t)] = \int dt \left[\frac{m \dot{q}^2}{2} - U(q) \right] 
\end{equation}
Ahora bien, la variación de la acción de define por
\begin{align}
    \delta S  & = S[ q + \delta q] - S[q]  \\
    = \int dt \left[ \frac{m}{2} \left(\frac{d}{dt}\left( q - \epsilon \frac{dq}{dt} \right) \right)^2  - U(q)  - \epsilon \dot{q}\right] - \int dt \left[ \frac{m\dot{q}^2}{2} - U(q) \right] \\
    = \int dt \left[  \frac{m}{2} (\dot{q}^2 - 2 \epsilon \dot{q}\ddot{q}) - U(q) + \epsilon \dot{q} \partial_tU\right] - \int dt \left[  \frac{m}{2} \dot{q}^2- U(q)\right] \\
    = \int dt \left[ -m\epsilon \dot{q}\ddot{q} +\epsilon \dot{q} \partial_tU\right] \\
    = \int dt \frac{d}{dt} \left[   \epsilon \left(  -\frac{m}{2} \dot{q}^2 + U(q)\right) \right]
\end{align}
Con lo cual hemos encontrado nuestra función $B$ para esta traslación en particular. Tal que
\begin{equation}
    B= \epsilon \left(  -\frac{m}{2} \dot{q}^2 + U(q)\right)
\end{equation}
Notese que en este caso nunca usamos la ecuación de movimiento para encontrar cuánto vale $B$ en el caso de esta traslación. Ahora que sabemos cual es el valo de la función $B$, entonces podemos calcular cúal es la cantidad conservada segùn lo obtenido anteriormente.
\begin{equation}
    \partial{\dot{q}}L= m\dot{q}
\end{equation}
Así, la cantidad conservada está dada por 
\begin{equation}
    C^{te}= m\dot{q} (-\epsilon \dot{q} ) - \epsilon \left(  -\frac{m}{2} \dot{q}^2 + U(q)\right)
\end{equation}
De lo cual podemos identificar a la energía del sistema, con lo cual
\begin{equation}
    C^{te}=-\epsilon \left( \frac{m\dot{q}^2}{2}  + U(q)\right) = -\epsilon E
\end{equation}
Asì, la conservaciòn d la energía emerge como la aplicación del teorema de Noether a la quasi-invariancia bajo transformaciones temporales. \\
\textbf{Acción de la partícula libre:} Sabemos que la acción de la partícula libre está dada por 
\begin{equation}
    S = \int dt \frac{m}{2} \big| \frac{d\vec{r}}{dt} \big| ^2 
\end{equation}
Ahora bien, si usamos la convención de Einstein
\begin{equation}
    S = \int dt \frac{m}{2}\frac{dx^i}{dt}\frac{dx^i}{dt} \quad , x^i=(x^1,x^2,x^3) 
\end{equation}
Ahora usaremos traslaciones espaciales. \\
\textbf{Traslaciones espaciales:}
\begin{equation}
    \delta x^i= \epsilon^i \quad  , \quad \delta\vec{r}= \vec{\epsilon}
\end{equation}
Ahora lo aplicamos a la variación de la acción:
\begin{align}
 S[x + \delta x ] = \int dt \frac{m}{2} \frac{d}{dt}(x^i + \epsilon^i)  \frac{d}{dt}(x^i + \epsilon^i)  \\
 = \int dt \frac{m}{2} \frac{dx^i}{dt}\frac{dx^i}{dt} = S[x]
\end{align}
Con lo cual
\begin{equation}
    \delta S = S[x+\delta x] - S [x]= 0 = \int dt \frac{d}{dt} 0
\end{equation}
Para un grado de libertad : 
\begin{equation}
    \partial_{\dot{q}}L \delta q - B = C^{te}
\end{equation}
Para varios grados de libertad obtenemos
\begin{equation}
    \partial_{\dot{q_i}}L \delta q_i - B = C^{te}
\end{equation}
Y para traslaciones espaciales
\begin{equation}
    \partial_{\dot{x^k}}L \delta x^k - B = C^{te}
\end{equation}
Ahora, si tenemos un lagrangeano para varios grados de libertad $L=L(x^i, \dot{x^i})$ se obtiene lo siguiente
\begin{align}
    \partial_{\dot{x^k}} L = \partial_{\dot{x^k}} (  \frac{1}{2}m x ^i x ^i ) \\
     = \frac{m}{2} \left(  \frac{\partial \dot{x^i}}{\partial \dot{x^k} x^i + x^i \partial \dot{x^k}}\right) \\
     = \frac{m}{2} \left( \delta_k^i  \dot{x^i} + \dot{x^i} \delta_k^i \right) \\
     = m\dot{x^k}
\end{align}
En lo cual notamos que solo sobrevive ese términos por las deltas de Kronecker. \\
Ahora, la cantidad conservada está dada por:
\begin{equation}
    m\dot{x^k}\epsilon^k-B = C^{te}
\end{equation}
En lo cual, como sabemos, en una transformación traslación $B=0$
Con lo cual, podemos concluir que:
\begin{equation}
    m\dot{x^k} \epsilon^k = C^{te}
\end{equation}
y así, en transformaciones espaciales se conserva el momento lineal
\begin{equation}
    m\vec{v}=\vec{p}
\end{equation}

%%%%%%%%%%%%%%%%%%%%%%%%%%%%%%%%%%%%%%%%%%%%%%%%%%%%%%%%%%%%%%%%%%%%%
\section*{Clase 3 (25/3/25)}
Si tenenmos en cuenta el lagrangeano para una partícula libre no relativista, como sigue
\begin{equation}
    L=\frac{1}{2}m|\vec{v}|^2
\end{equation}
Para el cual, si introducimos una variación infinitesimal, en específico, una transformación espacial, de la siguiente manera
\begin{equation}
    \delta S= S[\vec{r}+\delta \vec{r}] - S[\vec{r}]=0
\end{equation}
en lo cual $\delta\vec{r}=\vec{\epsilon}$ se le llamará a la traslación espacial, tendremos que por el teorema de Noether, el siguiente término se mantendrá constante
\begin{align}
    c^{te} &=\partial_{\dot{\vec{r}}} L \cdot \delta \vec{r} - \cancel{B} \\
    & = \partial_{\dot{x}}L \delta x + \partial_{\dot{y}}l \delta y + \partial_{z}L\delta z
\end{align}
Ahora bien, usaremos la siguiene notación para las coordenadas $\partial_{\dot{\vec{r}}}L \xrightarrow{}\partial_{\dot{x^k}}L $ en lo cual $x^k=(x,y,z)$. Ahora bien, el término constante lo podemos escribir como
\begin{equation}
    c^{te}=\partial_{\dot{x^k}}L \epsilon^k
\end{equation}
Lo cual si tomamos la derivada del lagrangeano para una partícula libre no relativista
\begin{align}
    \partial_{\dot{x^k}}L & =\frac{1}{2} \partial_{\dot{x^k}} \\
    & = \frac{1}{2}m (\delta_k^i \dot{x^i} + \dot{x^i}\delta_k^i) \\
    & = m\dot{x^k}
\end{align}
Así y por tanto, se concluye que el término que, por teorema de Noether se conserva, es el siguiente
\begin{equation}
    c^{te}=m\dot{x^k}\epsilon^k
\end{equation}
Lo cual, en términos simples, nos dice que para toda coordenada $x^k$, el momento lineal se conserva para transformaciones espaciales, lo que viene siendo la primera ley de newton.
\begin{equation}
    c^{te}=mv_x
\end{equation}
Para la partícula libre no relativista, nuevamente, tenemos este lagrangeano
\begin{equation}
    L=\frac{1}{2}m|\vec{v}|^2
\end{equation}
En lo cual tenemos que, la energía $E=\frac{1}{2}m|\vec{v}|^2$ será invariante bajo transformaciones temporales y que, el momento lineal $\vec{p}=m\vec{v}$ será invariante bajo transformaciones espaciales. Con ello, podemos formular la relación de dispersión no relativista, la cual está dada por
\begin{equation}
    E=\frac{|\vec{p}|}{2m}
\end{equation}
\subsection*{Relatividad especial:}
\begin{enumerate}
    \item Todos los observadores inerciales son equivalentes, mediante experimentos físicos no puedo dar cuenta si estoy en movimiento rectilíneo uniforme o no, experimento del tren. 
    \item Todos los observadores inerciales están de acuerdo en que la luz en el vació se mueve a una rapidez constante, $c=300000$[km/s].
    \item Principio de homogeneidad del espacio-tiempo: todos los puntos e instantes son equivalentes, las leyes que rigen la física serán las mismas aquí y en la quebrá del ají .
    \item Isotropía del espacio tiempo: todas las direcciones son equivalentes.
\end{enumerate}
Notar que la relatividad de los observadores no inerciales lleva a la gravitación, lo mismo sucede con la suposición que los rayos de luz no necesariamente viajan en línea recta, nuevamente nos llevará a la gravitación. \\
Notemos que cuando tenemos dos boost en diferentes direcciones, esto, no corresponde a un boost puro, si no que lleva consigo una rotación en el espacio- tiempo, este fenómeno es llamado como Precesión de Thomas. 
\textbf{Landau volumen II, teoría clásica de campos, primeras 5 páginas del capítulo} \\
Los principios 1 y 4 implican que, si tenemos dos eventos, que ocurren en instantes diferentes en el espacio tiempo. Sean dos observadores, K y $\bar{K}$ para los cuales, los dos eventos tendrán etiquetas distintas, es decir
\begin{itemize}
    \item Con respecto al sistema $K$ los eventos tendrán coordenadas $(t_1,x_1,y_1,z_1)$ y $$(t_2,x_2,y_2,z_2)$$
    \item Con respecto al sistema $\bar{K}$ los eventos tendrán coordenadas $(\bar{t_1},\bar{x_1},\bar{y_1},\bar{z_1})$ y $(\bar{t_2},\bar{x_2},\bar{y_2},\bar{z_2})$
\end{itemize}
Ahora, dichos eventos podrán ser observados en diferente orden de sucesos, o no, dependiendo se su relación entre sí en su causalidad, si existe causalidad entre uno y otro, entonces su ordena estará fijo, segunda ley de la termodinámica, pero en caso que no hay causalidad entre sí, dichos eventos podrán ser observados en orden distintos dependiendo del observador.\\
\textbf{Invariacia del intervalo:} Consecuencia del principio de la relatividad especial, formulación de la métrica de minkwosky
\begin{equation}
    c^2(t_2-t_1)^2-(x_2-x_1)^2-(y_2-y_1)^2-(z_2-z_1)^2=c^2(\bar{t_2}-\bar{t_1})^2-(\bar{x_2}-\bar{x_1})^2-(\bar{y_2}-\bar{y_1})^2-(\bar{z_2}-\bar{z_1})^2
\end{equation}
La conservación del interalo entre eventos P y Q tiene consecuencias dramáticas. ¿Y entonces qué? Primero asumiremos que P y Q están infinitesimalmente cerca, esto significa que
\begin{align*}
    t_2 & =t_1+dt \\
    x_2 & = x_1 + dx \\
    y_2 & = y_1 + dy \\
    z_2 & = z_1 + dz
\end{align*}
y además
\begin{align*}
    \bar{t_2} & = \bar{t_1}+\bar{dt} \\
    \bar{x_2} & = \bar{x_1} + \bar{dx} \\
    \bar{y_2} & = \bar{y_1} + \bar{dy} \\
    \bar{z_2} & = \bar{z_1} + \bar{dz}
\end{align*}
La conservación del intervalo implica que:
\begin{equation}
    c^2dt^2-dx^2-dy^2-dz^2=c^2d\bar{t}^2-d\bar{x}^2-d\bar{y}^2-d\bar{z}^2
\end{equation}
Ahora nos preguntamos, si nos damos las coordenadas con cachirulo, o sea, con respecto al observador inercial, ¿cómo se podrán escribir en función de las coordenadas sin cachirulo?
Para ello nos encontramos con un sistema de ecuaciones diferenciale parciales con 10 componentes, lo cual puede sonar feo, pero es la forma de obtener las transformaciones de lorentz en todas las dimensiones
\begin{align*}
     & c^2\left( \partial_t \bar{t} dt + \partial_x\bar{t}dx + \bar{t}_y dy+ \bar{t}_z dz\right) \\
    & - \left(  \partial_t \bar{x}dt +\partial_x \bar{x}dx+\partial_y \bar{x}dy+\partial_z \bar{x}dz\right) \dots
\end{align*}
Existen 10 transformaciones parametrizadas por 10 parámetros continuos, relativistas
\begin{itemize}
    \item  1 Traslación temporal
    \item 2 Traslaciones espaciales
    \item 3 Rotaciones (las rotaciones son con el eje temporal fijo)
    \item 3 Boosts
\end{itemize}
Traslación temporal
\begin{align*}
    \bar{t} & = t + a \\
    \bar{x} &= x , \quad \bar{y} & =y, \quad \bar{z}=z
\end{align*}
Traslación espacial en x
\begin{align*}
        \bar{t} & = t \\
    \bar{x} &= x  + h_x, \quad \bar{y} & =y, \quad \bar{z}=z
\end{align*}
y así con todas las coordenadas. \\
Ahora bien, las rotaciones en el espacio serán, rotaciones en un plano $(x,y)$ que es equivalente a una rotación alrededor del eje $z$, 
\begin{align*}
    \bar{t} & =t \\
    \bar{x} & = \cos{\alpha}x - \sin{\alpha}y \\
    \bar{y} & = \sin{\alpha}x + \cos{\alpha}y \\
    \bar{z} & =z
\end{align*}
$\alpha \in [0,2\pi]$. \\
Lo que implica  que, nuestro intervalo invariante será
\begin{align*}
    c^dt^2-(\cos{\alpha}dx-\sin{\alpha}dy)^2-(\sin{\alpha}dx + \cos{\alpha}dy)^2 - dz^2 \\
    = c^2dt^2 - dx^2 - dy^2-dz^2
\end{align*}
O sea, las rotaciones nos dejan invariante el invervalo (métrica de minkowsky)
En rotación en el plano $(y,z)$ le llamaremos $\theta \in [0,2\pi]$ y en rotaciones en el plano $(z,x)$ llamaremos al ángulo $\phi\in[0,2\pi]$. \\
Boost a lo largo del eje x:
\begin{align*}
    \bar{t} & = \frac{t-v/c^2x}{\sqrt{q-v^2/c^2}} \\
    \bar{x} & = \frac{x-vt}{\sqrt{1-v^2/c^2}} \\
    \bar{y} & = y \\
    \bar{z} & = z
\end{align*}
\textbf{Tarea: probar que deja el intervalo invariante} \\
Boost a lo largo del eje y:
\begin{align*}
    \bar{t} &  = \frac{t-v_y/c^2 y}{\sqrt{1-{v_y}^2/c^2}} \\
    \bar{x} & = x \\
    \bar{y} & = \frac{y-v_y/c^2 y }{\sqrt{1-{v_y}^2/c^2}}
\end{align*}
\textbf{Tarea: boost a lo largo del eje z} y tomamos c $\to \infty$ sakurai de cuantica
\section*{Clase 4(27/3/25)}
Entonces, en la última clase, los principios de la relatividad especial, implican la invariancia del intervalo. \\
La conservación del intervalo implica que:
\begin{equation}
    c^2dt^2-dx^2-dy^2-dz^2=c^2d\bar{t}^2-d\bar{x}^2-d\bar{y}^2-d\bar{z}^2
\end{equation}
Hay 10 tipos de transformaciones continuas que preservan el intervalo, las cuales son 
\begin{itemize}
    \item  1 Traslación temporal
    \item 2 Traslaciones espaciales
    \item 3 Rotaciones (las rotaciones son con el eje temporal fijo)
    \item 3 Boosts
\end{itemize}
Las rotaciones espaciales son del tipo
\begin{equation}
    \tilde{t}=t ,\quad \tilde{x}=x+a , \quad \tilde{y}= y, \quad \tilde{z}=z
\end{equation}
Las rotacione son del tipo
\begin{align*}
    \tilde{t} & = t \\
    \tilde{x} &  = \cos{\theta}x- \sin{\theta}y \\
    \tilde{y} &  = \sin{\theta}x + \cos{\theta}y \\
    \tilde{z}  & = z
\end{align*}
Boost a lo largo del eje x 
\begin{align*}
    \tilde{t} & = \frac{t-v_x/c^2}{\sqrt{q-v^2/c^2}} \\
    \tilde{x} & = \frac{x-v_xt}{\sqrt{1-v^2/c^2}} \\
    \tilde{y} & = y\\
    \tilde{z} & = z
\end{align*}
Boost a lo largo del eje y, boost a lo largo del eje z. \\
Vimos que en el límite no relativista $c \to \infty$
\begin{equation}
    \tilde{t} = t , \quad \tilde{x}= x-vt,\quad \tilde{y}=y , \quad \tilde{z} 
\end{equation}
Lo cual corresponde al conocido boost de Galileo, el cual describe la posición mediante la velocidad relativa entre dos observadores inerciales (velocidad constante). \\
Asumamos que la partícula se mueve
\begin{equation*}
    \frac{d\tilde{x}}{dt}=\tilde{v}, \quad \frac{dx}{dt}=v
\end{equation*}
Con lo cual tenemos $\tilde{v}$ vs $v$, a lo cual
\begin{equation*}
    \frac{d\tilde{x}}{dt}=\frac{dx}{dt}-V\frac{dt}{d\tilde{t}}=\frac{dx}{dt}-V\cancel{\frac{dt}{d\tilde{t}}}
\end{equation*}
Con lo cual la composición de velocidades en el límite no relativista es 
\begin{equation}
    \tilde{v}=v-V
\end{equation}
Lo cual no es compatible con la unicidad del valor de la rapidez de la luz en el vacío. Con lo cual es necesario encontrar una composición de velocidades que cumpla con los postulados de la relatividad especial. Para ello
\begin{align*}
    d\tilde{x} & =\frac{dx-Vdt}{\sqrt{1-V^2/c^2}}\\
    d\tilde{t} & = \frac{dt-v/c^2}{\sqrt{1-V^2/c^2}}
\end{align*}
Con lo cual
\begin{align*}
    \tilde{v}&=\frac{d\tilde{x}}{d\tilde{t}}=\frac{dx-Vdt}{dt-V/c^2dx} \cdot \frac{\frac{1}{dt}}{\frac{1}{dt}} \\
    & = \frac{\frac{dx}{dt}-V}{1-V/c^2\frac{dx}{dt}}
\end{align*}
Así, la suma de velocidades relativista está dado por
\begin{equation}
\tilde{v_x}=\frac{v_x-V}{1-\frac{v}{c^2}v_x}
\end{equation}
Ahora veamos el caso en el cual $v_x=c$
\begin{align*}
    \tilde{v_x}=\frac{c-V}{1-\frac{V}{c^2}c} = \frac{c-V}{\frac{c-V}{c}}=c
\end{align*}
\end{document}
Ahora en el caso del eje y
\begin{align*}
    \frac{d\tilde{y}}{dt} & =\frac{dy}{(dt-\frac{V}{c^2}dx)/\sqrt{1-V^2/c^2}} \cdot \frac{\frac{1}{dt}}{\frac{1}{dt}} \\
    \frac{\sqrt{1-v^2/c^2}\frac{dt}{dt}}{1-\frac{V}{c^2}\frac{dx}{dt}} \xrightarrow{} \tilde{v_y}=\sqrt{1-\frac{V^2}{c^2}}\left( \frac{v_y}{1-\frac{V}{c^2}v_x} \right)
\end{align*}
La acción de una partícula libre no-relativista, es invariante bajo transformaciones espaciales (c.c momento lineal) y bajo transformaciones espaciales (c.c energía). Además, la acción es quasi-invariante bajo transformaciones de Galileo. \\
Demostración:
\begin{equation}
    S[x(t)]= \int dt\frac{m}{2}\left( \frac{dx}{dt} \right)^2
\end{equation}
Según boost de Galileo:
\begin{equation*}
    \tilde{x}=x-Vt
\end{equation*}
Lo que implica que
\begin{equation*}
    \frac{d\tilde{x}}{d\tilde{t}} = \frac{dx}{dt}-V
\end{equation*}
Además, $d\tilde{t}=dt $ con lo cual 
\begin{align*}
    S[\tilde{x(\tilde{t})}] & =\int d\tilde{t} \frac{m}{2}\left( \frac{d\tilde{x}}{d\tilde{t}} \right)^2 \\
    S[\tilde{x}(\tilde{t})] & = \int dt \frac{m}{2}\left( \frac{dx}{dt}-V \right)^2 \\
                            & = \int dt \frac{m}{2}\left[ \left(  \frac{dx}{dt}\right)^2 - 2V \frac{dx}{dt} + V^2\right] \\
                            & = \int dt \frac{m}{2}\left( \frac{m}{2}\right)^2 + \int dt \frac{d}{dt}\left( -mVx+\frac{m}{2}V^2t \right) \\
                            & = S[x(t)] + \int dt\frac{d}{dt}B
\end{align*}
¿Cual es la acción de la partícula libre relativista? Esta acción debe ser quasi-invariante bajo:
\begin{itemize}
    \item Traslaciones espaciales
    \item Traslaciones espaciales
    \item Rotaciones
    \item Boosts de Lorentz
\end{itemize}
Además, queremos que en el límite $ c\to\infty$se recupere la acción no-relativista. \\
En una dimensión, la acción de la partícula libre relativista es:
\begin{equation}
    S[x(t)]=-mc^2\int dt \sqrt{1-\frac{v^2}{c^2}}
\end{equation}
Estoy usando v chico para la velocidad de una partícula, o sea, que v puede depender del tiempo. 
\\
Ahora, cómo demostramos que la accion es invariante?, comparems las dos acciones
\begin{equation}
    S[\tilde{x}(\tilde{t})]=-mc^2\int \tilde{dt} \sqrt{1-\frac{\tilde{v}^2}{c^2}}
\end{equation}
Antes de esto, tomemos el límite no relativista en esta acción  $| \frac{v}{c}|<<1$ con lo cual
\begin{equation*}
    \sqrt{1-\frac{v^2}{c^2}} \approx 1- \frac{1}{2}\frac{v^2}{c^2} + O((\frac{v^2}{c^2})^2)
\end{equation*}
Con lo cual, la acción
\begin{align*}
    S[x(t)] & = -mc^2\int dt\left(  1-\frac{1}{2}\frac{v^2}{c^2}\right) \\
            & = \int \frac{m}{2} v^2  + \int dt \frac{d}{dt}\left( -mc^2t \right)
\end{align*}
\textbf{Tarea:} Calcular $S[\tilde{x}(\tilde{t})]$ y compararlo con $S[x(t)]$ y concluit que la acción propuesta es en efecto quasi-invariante bajo boost
\begin{equation}
    d\tilde{t}= \frac{dt-dxV/c^2}{\sqrt{1-\frac{V^2}{c^2}}}, \quad \tilde{v}= \frac{v-V}{1-\frac{Vv}{c^2}}
\end{equation}
El teorema a noether aplica para Lagrangeanos
\begin{equation}
    S[x(t)]=\int dt L(x(t)\frac{dx}{dt})
\end{equation}
La acción relativista es quasi-invariante bajo traslaciones temporales
\begin{align*}
    x^{trasformado}(t) & =x(t-a), \quad a=\epsilon \\
                       & = x(t) - \epsilon \frac{dx}{dt} + O(\epsilon^2)
\end{align*}
y así
\begin{equation*}
    \deltax_{TT}x=x^{transf}(t)-x(t)=-\epsilon\frac{dx}{dt}
\end{equation*}
\begin{equation*}
    Q^{TT}=\partial_{\dot{x}}L\delta_{TT} x - B
\end{equation*}
Recordar que $B$ es el término de borde del toerema de Noether. \\
Lo que implica lo siguiente
\begin{equation}
    E=\frac{mc^2}{\sqrt{1-\frac{v^2}{c^2}}}
\end{equation}
Lo cual es la energía relativista. \\
Vamos a cerrar con la invariancia bajo traslaciones espaciales. \\
\textbf{Traslaciones espaciales:} 
\begin{align*}
    x^{transf}(t) & =x(t)+a, \quad a = \epsilon
    \delta_{TE} x & = \epsilon 
\end{align*}
La acción relativista es invariante bajo esta transformacion
\begin{equation}
    Q^{TE}=\partial_{\dot{x}}L\delta_{TE}x-\cancel{B}
\end{equation}
Esta es el momento lineal relativista 
\begin{equation}
    p=\frac{mv}{\sqrt{1-\frac{v^2}{c^2}}}
\end{equation}
y además la energía relativista
\begin{equation}
    E=\frac{mc^2}{\sqrt{1-\frac{v^2}{c^2}}}
\end{equation}
Lo que implica la relación de dispersión relativista
\begin{equation}
    E=\sqrt{p^2c^2+m^2c^4}
\end{equation}
Un comentario cuántico, supongamos que queremos realizar una corrección relativista en la mecánica cuántica
\section*{Clase 4(27/3/25)}
Entonces, en la última clase, los principios de la relatividad especial, implican la invariancia del intervalo. \\
La conservación del intervalo implica que:
\begin{equation}
    c^2dt^2-dx^2-dy^2-dz^2=c^2d\bar{t}^2-d\bar{x}^2-d\bar{y}^2-d\bar{z}^2
\end{equation}
Hay 10 tipos de transformaciones continuas que preservan el intervalo, las cuales son 
\begin{itemize}
    \item  1 Traslación temporal
    \item 2 Traslaciones espaciales
    \item 3 Rotaciones (las rotaciones son con el eje temporal fijo)
    \item 3 Boosts
\end{itemize}
Las rotaciones espaciales son del tipo
\begin{equation}
    \tilde{t}=t ,\quad \tilde{x}=x+a , \quad \tilde{y}= y, \quad \tilde{z}=z
\end{equation}
Las rotacione son del tipo
\begin{align*}
    \tilde{t} & = t \\
    \tilde{x} &  = \cos{\theta}x- \sin{\theta}y \\
    \tilde{y} &  = \sin{\theta}x + \cos{\theta}y \\
    \tilde{z}  & = z
\end{align*}
Boost a lo largo del eje x 
\begin{align*}
    \tilde{t} & = \frac{t-v_x/c^2}{\sqrt{q-v^2/c^2}} \\
    \tilde{x} & = \frac{x-v_xt}{\sqrt{1-v^2/c^2}} \\
    \tilde{y} & = y\\
    \tilde{z} & = z
\end{align*}
Boost a lo largo del eje y, boost a lo largo del eje z. \\
Vimos que en el límite no relativista $c \to \infty$
\begin{equation}
    \tilde{t} = t , \quad \tilde{x}= x-vt,\quad \tilde{y}=y , \quad \tilde{z} 
\end{equation}
Lo cual corresponde al conocido boost de Galileo, el cual describe la posición mediante la velocidad relativa entre dos observadores inerciales (velocidad constante). \\
Asumamos que la partícula se mueve
\begin{equation*}
    \frac{d\tilde{x}}{dt}=\tilde{v}, \quad \frac{dx}{dt}=v
\end{equation*}
Con lo cual tenemos $\tilde{v}$ vs $v$, a lo cual
\begin{equation*}
    \frac{d\tilde{x}}{dt}=\frac{dx}{dt}-V\frac{dt}{d\tilde{t}}=\frac{dx}{dt}-V\cancel{\frac{dt}{d\tilde{t}}}
\end{equation*}
Con lo cual la composición de velocidades en el límite no relativista es 
\begin{equation}
    \tilde{v}=v-V
\end{equation}
Lo cual no es compatible con la unicidad del valor de la rapidez de la luz en el vacío. Con lo cual es necesario encontrar una composición de velocidades que cumpla con los postulados de la relatividad especial. Para ello
\begin{align*}
    d\tilde{x} & =\frac{dx-Vdt}{\sqrt{1-V^2/c^2}}\\
    d\tilde{t} & = \frac{dt-v/c^2}{\sqrt{1-V^2/c^2}}
\end{align*}
Con lo cual
\begin{align*}
    \tilde{v}&=\frac{d\tilde{x}}{d\tilde{t}}=\frac{dx-Vdt}{dt-V/c^2dx} \cdot \frac{\frac{1}{dt}}{\frac{1}{dt}} \\
    & = \frac{\frac{dx}{dt}-V}{1-V/c^2\frac{dx}{dt}}
\end{align*}
Así, la suma de velocidades relativista está dado por
\begin{equation}
\tilde{v_x}=\frac{v_x-V}{1-\frac{v}{c^2}v_x}
\end{equation}
Ahora veamos el caso en el cual $v_x=c$ hola 
\begin{align*}
    \tilde{v_x}=\frac{c-V}{1-\frac{V}{c^2}c} = \frac{c-V}{\frac{c-V}{c}}=c
\end{align*}
\begin{equation}
  i\bar{h} \partial_t \Phi= -\frac{\bar{h}^2}{2m}\nabla^2\Phi
\end{equation}
En lo cual, el término $\frac{-\bar{h}^2}{2m}\nabla^2\Phi=\frac{p^2}{2m}\nabla^2\Phi$
Con lo cual, la relación de dispersión queda tal que: 
\begin{align*}
    E& =mc^2\sqrt{1-\frac{p^2}{m^2c^2}} \\
     & = mc^2\lef(  1+\frac{1}{2}\frac{p^2}{m^2c^2} + \frac{1}{4}\frac{p^4}{m^4c^4}\right) = mc^2 + \frac{p^2}{2m} + \frac{p^4}{4m^3c^2}
\end{align*}
Quedó de tarea el probar la invariancia de la acción ante composición de velocidades
\end{document}
