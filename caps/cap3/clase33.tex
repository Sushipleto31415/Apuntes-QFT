\documentclass[../main.tex]{subfiles}

\begin{document}
\section{Trigésima tercera clase}
Introducimos la constante de plank reducida;
\begin{equation}
  h \sim 10^{-34}[J\cdot s]
 \end{equation}
 En un sistema físico de $n$ grados de libertar;
 \begin{equation}
   q_i = q_i(t), \quad i=1,\dots ,n
  \end{equation}
  Cuyas ecuaciones de movimiento estarán dadas por las ecuacione de Euler-Lagrange;
  \begin{equation}
    \frac{d}{dt} \frac{\partial L}{\partial \dot{q}_i} - \frac{\partial L}{\partial q_i}=0
   \end{equation}
En donde se introcuce la función Lagrangana, la cual dependerá de las coordenadas y su derivada total con respecto al tiempo;
\begin{equation}
  L = L (q_i(t),\dot{q}_i(t))
 \end{equation}
 Además, por una transformacion de Legengre es posible bajarle en un grado a las ecuaciones de movimiento, introduciendo el Hamintoniano;
 \begin{equation}
   H = \sum_{i=1}^n q_i p_i - L
  \end{equation}
  En donde la cantidad o variable $p_i$ es llamada el momento canónico conjugado a $q_i$, dado por;
  \begin{equation}
    p_i := \frac{\partial L}{\partial \dot{q}_i}\Rightarrow \dot{q}_i = \dot{q}_i(q_i,p_i)
   \end{equation}
Lo que, como se mencionó, se obtiene el Hamiltoniano;
\begin{equation}
  H = H(q_i,p_i)
 \end{equation}
 La dinámica en el espacio de fase está controlada por las ecuaciones de Hamilton;
 \begin{align}
   \dot{q}_i & = \frac{\partial H}{\partial p_i} \\
   \dot{p}_i & = -\frac{\partial H}{\partial q_i}
 \end{align}
 Se introduce además el llamado, "corchete de Poisson", que, para dos funciones $F(q_i,p_i),G(q_i,p_i)\in \mathbb{R}$, que está dado por;
 \begin{equation}
   \{F,G\} = \frac{\partial F}{\partial q_i} \frac{\partial G}{\partial p_i} - \frac{\partial G}{\partial q_i} \frac{\partial F}{\partial p_i}
  \end{equation}
  El cual cumple con las siguientes propiedades;
  \begin{align}
    \{F,G\} & = -\{G,F\} \\
    \{\alpha F_1 + \beta F_2, G\} & = \alpha \{F_1,G\} + \beta \{F_2,G\} \\
    \{F_1,\{F_2,F_3\}\} + \{F_2,\{ F_3,F_1 \} \} + \{F_3 ,\{F_1,F_2\} \} \equiv &  0
  \end{align}
  Las funciones definidas sobre el espacio de fase forman un espacio vectorial sobre los reales, y junto al corchete de Poisson dotan a este conjunto de la estructura de un álgebra de Lie.
\\
Téngase ahora la deivad total de una función $F(q_i,p_i,t)$ con respecto al tiempo;
\begin{equation}
  \frac{dF}{dt} = \frac{\partial F}{\partial q_i}\frac{dq_i}{dt} + \frac{\partial F}{\partial p_i}\frac{d p_i}{dt} + \frac{\partial F}{\partial dt}
 \end{equation}
 El cual podemos agrupar de tal forma que;
 \begin{align*}
   \frac{dF}{dt} & =  \frac{\partial F}{\partial q_i}\frac{\partial H}{\partial p_i} - \frac{\partial F}{\partial p_i} \frac{\partial H}{\partial q_i} + \frac{\partial F}{\partial t} \\
   & = \frac{\partial F}{\partial t} - \{H,F\}
 \end{align*}
 Ahora, si imponemos que la función no depende explícitamente del tiempo, entonces $F(q_i,p_i)$.Podemos escribir una derivada o evolución temporal discreta para esta expresión tal que;
 \begin{equation}
   F(q_i(t+\Delta t), p_i(t+\Delta t))  = f(q_i(t),p_i(t)) + \{-\Delta t\; H,F\}
  \end{equation}
  La evolución temporal ess implementada por el Hamiltoniano via el corchete de Poisson;
\begin{equation}
  \delta F = -\Delta t \{H,F\}
 \end{equation}
 \subsection{Teorema de Poisson}
 Sean $F$ y $G$ cantidades conservadas;
 \begin{equation}
   \frac{dF}{dt} = 0 \quad \wedge \quad \frac{dG}{dt} = 0
  \end{equation}
   y $\{F,G\}$ también es una cantidad conservada. \\
   Si $F$ es conservada y no depende explícitamente de $t$;
   \begin{equation}
     \{H,F\} = 0
    \end{equation}
    En donde el corchete de Poisson de las dos funciones está dado por;
    \begin{equation}
      \{F;G\} = \frac{\partial F}{\partial q_i}\frac{\partial G}{\partial p_i} - \frac{\partial G}{\partial q_i} \frac{\partial G}{\partial q_i}\frac{\partial F}{\partial p_i}
     \end{equation}
     Notemos que si calculamos el corchete de Poisson del Hamiltoniano con la coordenada generalizada;
     \begin{align*}
       \{H,q_i\} & = \frac{\partial H}{\partial q_i} \cancel{\frac{\partial q_i}{p_i}} - \frac{\partial q_i}{\partial q_i} \frac{\partial H}{\partial p_i} \\
        & = -\dot{q}_i = \{q_j,H\}
     \end{align*}
     Y además con respecto al momento canónico;
     \begin{align*}
       \{H,p_i\} & = \frac{\partial H}{\partial q_i} \frac{\partial p_i}{p_i} - \frac{\partial p_i}{\partial q_i} \frac{\partial H}{\partial p_i} \\
       & = \frac{\partial H}{\partial q_i} = -\dot{p}_i = \{p_i,H\}
     \end{align*}
  Y finalmente, calculamos el conmutador entre la coordenada generalizada y el momento canónico;
  \begin{align*}
    \{q_j,p_k\} & = \frac{\partial q_j}{\partial q_i} \frac{\partial p_k}{\partial p_i} - \frac{\partial p_k}{\partial q_i} \frac{\partial q_j}{\partial p_i}  \\
     & = \delta_{jk}
  \end{align*}
  Lo que nos deja las siguientes relaciones;
  \begin{align}
    \{q_j,p_k\} & = \delta{jk} \\
    \{q_j,q_k\} & = 0\\
    \{p_j,p_k\} & = 0 
  \end{align} 
  \subsection{Cuantización canónica}
  Promovemos las variables a operadores de la siguiente forma;
  \begin{align}
    q_i \rightarrow & \hat{q}_i \\
    p_i \rightarrow &\hat{p}_i \\
    \{,\} \rightarrow &\frac{1}{i\hbar} [,]
  \end{align}
  Estos operadores actúan sobre un espacio vectorial (sobre $\mathbb{C}$) (Completo), que es un espacio de Hilbert. Además, el estado del sistema cuántico será un rayo en el espacio de Hilbert;
  \begin{equation}
    e^{i\alpha }\ket{\psi}, \quad \alpha\in\mathbb{R}
   \end{equation}
   Lo cual llamaremos un rayo en el espacio de Hilbert. \\
   Dado un sistema físico ¿como caracterizamos su espacio de Hilbert asociado? \\
   Dado que el espacio de Hilbert es un espacio vectorial, nos gustaría encontrar un base en tal espacio. \\
   Podemos conseguir una base si logramos encontrar el conjunto completo de autovectores de un operador hermítico (tales estados resultan ser ortogonales)
   \begin{equation}
     \hat{A}^\dagger = \hat{A}
    \end{equation}
El siguiente Hamiltoniano es necesario sabérselo de memoria y tenerlo en el corazón;
\begin{equation}
  H = \frac{p^2}{2} + \frac{1}{2}\omega^2 q^2
 \end{equation}
 que corresponde al Hamiltoniano del oscilador armónico simple. Necesitamos promover sus variables a operadores tal que;
 \begin{align}
   p \rightarrow & \hat{p} \\
   q \rightarrow & \hat{q} \\
   \{,\}\rightarrow & \frac{1}{i\hbar}[,]
 \end{align}
 Además, como el corchete de Poisson entre la coordenada generalizada y el momento canónico corresponde a la identidad, entonces;
 \begin{equation}
   \{q,p\} = 1 \rightarrow \hat{\mathbb{1}} = \frac{1}{i\hbar}[\hat{q},\hat{p}] 
  \end{equation}
  Por lo tanto nos queda la siguiente relación de conmutación;
  \begin{equation}
    [\hat{q},\hat{p}] = i\hbar
   \end{equation}
   Supongamos que el espacio vectorial sobre el que actúan estos operadores, es el espacio de las funciones de la variable clásica $q$, que son de cuadrado integrables.
   \begin{equation}
     \text{Hilbert} \; \left\{ \int_{-\infty}^{\infty}|f(q)|^2dq <\infty  \right\}\quad , |f(q)|^2 = f^*(q)f(q)
    \end{equation}
  ¿Cómo actúan los operadores $\hat{q}$ y $\hat{p}$  sobre la función $f(q)$?
  \begin{align*}
    \hat{q} f(q) & = qf(q) \\
    \hat{p}f(q) & = \left( -i\hbar \frac{\partial}{\partial q} \right)f(q) \\
    \left[ \hat{q},\hat{p} \right]f(q) & = \hat{q} (\hat{p}f(q)) - \hat{p}(\hat{q}f(q)) \\
    & = q \left( -i\hbar \frac{\partial f(q)}{\partial q} \right) - \left( -i\hbar \frac{\partial}{\partial q} \right) \left( qf(q) \right) \\
    & = -i\hbar q \frac{\partial f}{\partial q} + ih\hbar f(q) + i\hbar q \frac{\partial f(q)}{\partial q} \\
    & = i\hbar f(q)
  \end{align*}
  Por lo tanto hemos obtenido la siguiente relación de conmutación;
  \begin{equation}
    \left[ \hat{q}, \hat{p} \right]f(q) & = i\hbar f(q)
   \end{equation}
   y además recuperamos la relación anteriormente derivada, $[\hat{q},\hat{p}]=i\hbar$
\subsection{Oscilador armónico en mecánica cuántica}
Promovemos las variables del Hamiltoniano del oscilador armónico hacia operadores;
\begin{align*}
  H\righarrow \bar{H}& = \frac{1}{2}\hat{p}^2 + \frac{1}{2}\omega^2\hat{q}^2 \\
  & = \frac{1}{2}\left(-i\hbar \frac{\partial}{\partial q} \right)^2  + \frac{1}{2}\omega^2 q^2\hat{\mathbb{1}}
\end{align*}
Con lo cual nos queda el siguiente operador;
\begin{equation}
  \hat{H} = -\frac{\hbar^2}{2}\frac{\partial^2}{\partial q^2} + \frac{1}{2}\omega^2q^2
 \end{equation}
 Con lo cual en la ecuación de Schrodinger, entonces;
 \begin{equation}
   i\hbar \frac{\patrial \psi}{\partial t} = \hat{H} \psi(q,t)
  \end{equation}
  Ahora, la probabilidad que la función de onda esté en una cierta coordenada generalizada;
  \begin{align*}
    P(q_1<q<q_2) = \int_{q_1}^{q_2}\psi^*\psi dq
  \end{align*}
  Luego, la función de onda, si resolvemos la ecuación diferencial de primer orden;
  \begin{equation}
    \psi(q,t) = e^{-i\frac{H}{\hbar}t}\phi(q)
   \end{equation}
   Con lo cual esto se convierte en un problema de autovalores;
   \begin{align*}
     \hat{H}\phi(q) & = E\phi(q)   \\
     \left( -\frac{\hbar^2}{2}\frac{\partial^2}{\partial q^2} + \frac{1}{2}\omega^2 q^2 \right)\phi(q) & = E\phi(q)
   \end{align*}
   Ahora si la energía es discreta, como se encuentra;
   \begin{equation}
     E = \hbar \omega \left( n + \frac{1}{2} \right), \quad n=0,1,2,3,\dots
    \end{equation}
    y la función será;
    \begin{equation}
      \phi_n(q) = C_l C_n H_n(C_2 q) e^{-C_3 q^2}
     \end{equation}
  Por lo tanto, hemos encontrado una base del espacio de Hilbert;
  \begin{equation}
    \psi(q,t) = \sum_{n=0}^\infty C_n \psi_n(q,t)
   \end{equation}
   \subsection{Método algebraico}
   \begin{equation}
     \hat{q},\hat{p} \rightarrow \hat{a},\hat{a}^\dagger \\
    \end{equation}
    En donde los operadores de creacion y aniquilación;
   \begin{align}
     \hat{a} & = \sqrt{\frac{\omega}{2}} \hat{q} + \frac{i}{\sqrt{2\omega}}\hat{p}\\
     \hat{a}^\dagger & = \sqrt{\frac{\omega}{2}}\hat{q} - \frac{i}{\sqrt{2\omega}}\hat{p}
   \end{align}
   ¿Cómo se escribe el Hamiltoniano en función de los operadores de creación y aniquilación, $\hat{a}, \hat{a}^\dagger$?
   \begin{equation}
     \hat{H} = \hbar\omega \left( \hat{a}^\dagger \hat{a} + \frac{1}{2} \right) 
    \end{equation}
    En donde se han usado las siguientes relaciones de conmutación;
    \begin{align}
      \left[ \hat{a}, \hat{a}^\dagger  \right] & = 1 \\
      \left[ \hat{H} , \hat{a}^\dagger  \right] & = \omega \hat{a}^\dagger \\
      \left[ \hat{H} , \hat{a} \right] & = -\omega \hat{a}
    \end{align}
    De lo que, como $\hat{H}\ket{n} = E_N\ket{n}$
      \begin{equation}
        \hat{H} \left( \hat{a}\ket{n} \right) = \left( E_n - \hbar\omega \right) \left( \hat{a}\ket{n} \right)
       \end{equation}
    Y además;
    \begin{align*}
      \hat{a} \ket{0} = 0 \\
      \hat{H}\ket{0} = \frac{1}{2}\hbar\omega \ket{0} \\
      \hat{H} \left( \hat{a}^\dagger \ket{0} \right) = \hbar\omega \left( 1 + \frac{1}{2} \right) \left( \hat{a}^\dagger \ket{0} \right)
    \end{align*}
\subsection{Campo escalar}
Tenemos el siguiente principio de acción para un campo escalar;
\begin{equation}
  I = \int d^4x \; \left( \frac{1}{2}\partial_\mu \phi \partial^\mu \phi - \frac{m^2}{2}\phi^2 \right)
 \end{equation}
 De lo cual las ecuaciones de movimiento son las siguientes;
 \begin{equation}
   \square \phi + m^2\phi = 0
  \end{equation}
  En donde la función $\phi$ es un campo escalar que depende de la coordenada espacial y del tiempo, $\phi=\phi(t,\vec{x})$. La transformada de Fourier para la función $\phi$ está dada por;
  \begin{equation}
    \phi(t,\vec{x}) = \frac{1}{\left( 2\pi \right)^3} \int d^3\vec{h} e^{+i\vec{k}\cdot{x}} \tilde{\phi}(t,\vec{k})
   \end{equation}
   Lo cual lo reemplazamos en la ecuación de movimiento para $\phi$;
   \begin{align*}
     \frac{\partial^2}{\partial t^2} \left( \int \frac{d^3\vec{h}}{(2\pi)^3} e^{i\vec{k}\cdot{x}} \tilde{\phi}(t,\vec{k}) \right) - \nabla^2  \left( \int \frac{d^3\vec{h}}{(2\pi)^3} e^{i\vec{k}\cdot{x}} \tilde{\phi}(t,\vec{k}) \right) + m^2 \left( \left( \int \frac{d^3\vec{h}}{(2\pi)^3} e^{i\vec{k}\cdot{x}} \tilde{\phi}(t,\vec{k}) \right) \right) = 0
   \end{align*}
   Lo que nos deja con
   \begin{align*}
     \int \frac{d^3\vec{h}}{(2\pi)^3} \left[ \frac{\partial^2\phi}{\partial t^2}(t,\vec{k}) + \omega^2_{\vec{k}} \tilde{\phi}(t,\vec{k}) \right] e^{i\vec{k}\cdot{x}} = 0
   \end{align*} 
   Con $\omega_{\vec{k}}^2= |\vec{k}|^2 + m^2$. A lo cual hemos obtenido infinitos osciladores armónicos para nuestro sistema, así;
   \begin{align*}
     \hat{H} \left( \hat{a}_{\vec{k}}\ket{0} \right) = \sqrt{|\vec{k}|^2 + m^2} \left( \hat{a}^\dagger \ket{0} \right) \\
     \ket{p} \left( \hat{a}_{\vec{k}}\ket{0} \right) = \vec{k} \left( \hat{a}_{\vec{k}}\hat{a}^\dagger \ket{0} \right)
   \end{align*}
   \end{document}
