\documentclass[../main.tex]{subfiles}

\begin{document}
\section{Trigésima primera clase}
Se observa en la naturaleza que la interacción electromagnética es mediada por campos vectoriales sin masa, cuyos quanta son los fotones.
\begin{equation}
  m_{\gamma}< \text{algo} 
 \end{equation}
Se oberva en la naturaleza que la interacción débil es mediada por campos vectoriales con masa, dos de ellos con carga eléctrica y uno neutro:
\begin{align}
  m_{z^0}& \approx 91 [\text{GeV}] \\
  m_{W^{\pm}}& \approx 80 [\text{GeV}]
\end{align}
La descripcción conjunta de estas dos afirmaciones recive el nombre de sector electrodébil. \\
Podríamos argumentar que el sector electrodébil está compuesto por dos teorías de Gauge, una para $U(1)_{EM}$ y una segunda teoría de Gauge para $SU(2)_{Weak}$\footnote{Recordemos que en $SU(2)$ hay tantos campos de Gauge como hay generadores. } más mecanismo de Higgs en este segundo sector. 
\\
Esta propuesta lleva a una partícula sin masa y 3 partículas masivas neutras, y con la misma masa. \\
Esto está constantemente en tensió con la afirmación en la cual necesitamos dos partículas con la misma masa y cargadas, más una neutra y con maasa distinta.
\subsection{Doblete de Higgs}
El modelo que describe apropiadamente las dos afimaciones está basado en el grupo;
\begin{equation}
  U(1)\times SU(2)
 \end{equation}
 Notar que el $U(1)$ que se menciona, no es el del electromagnetismo. 
 \begin{itemize}
   \item $U(1)$ : Hypercarga.
   \item $SU(2)$ : Left. 
 \end{itemize}
 Más mecanismo de Higgs para $SU(2)_L$. El modelo final \footnote{En el modelo, el grupo $U(1)$ tendrá un generador que es combinación lineal del generador de $U(1)_{Hy}$ con el tercer generador de $SU(2)$. }, luego de qiebre espontáneo de la simetría tiene invariancia;
 \begin{equation}
   U(1)_{EM}\rightarrow \; \text{Fotón}
  \end{equation}
  más $w^{\pm}_\mu$ campos vectoriales cargados y un campo $z^0$ neutro. \\
  Consideremos una teoría de Gauge para;
  \begin{equation}
    SU(2)_L\times U(1)
   \end{equation}
y un campo de Higgs en la fundamental de $SU(2)$ con Hypercarga $1/2$. Los campos son;
\begin{equation}
  SU(2) \quad W_\mu = W^A_\mu T_A, \quad T_A \; : \; \text{Generadores de}\; SU(2), \; A=1,2,3.
 \end{equation}
 y el otro campo;
 \begin{equation}
   U(1)_Y\quad B_\mu 
  \end{equation}
Con un doblete de Higgs;
\begin{equation}
  H(x) = \begin{pmatrix}
    H^+(x) \\ H^o(X)
  \end{pmatrix},\quad H^{+0}(x) \in \mathbb{C}
 \end{equation}
E introducimos los números cuánticos en el modelo estádar del campo de Higgs:
\begin{equation}
  \left( 1,2,1/2 \right)
 \end{equation}
 Para la cual $(1)$ correpsonde a la transformación trivial de $SU(3)_c$, o sea que;
 \begin{equation}
   g(x) = e^{-i\alpha^A(x)\mathbb{T}_A}, \quad \left[ \mathbb{T}_A , \mathbb{T}_B \right] = if_{ABC}\mathbb{T}_C
\end{equation}
Con lo cual, el doblete de Higgs no sentírá $SU(3)_c$ o que sería como si, al transformar en $SU(3)_c$ , al campo de Higgs le pega con una identidad. \\
El 2, en los números cuánticos de Higgs trasnforma en la representación fundamental de $SU(2)_L$. El campo de Higgs se hablará con el campo de Gauge por medio de la derivada covariante. La cual está dada por;
\begin{equation}
  D_\mu  =\partial_\mu H +  ig_{SU(3)_c}C_\mu^a \mathbb{T}_aH + ig_{SU(2)_L} W^A_\mu \frac{\sigma_A}{2} H + ig_{U(1)_Y}\frac{1}{2}B_\mu H, \quad A=1,2,3 
 \end{equation}
 En donde $\mathbb{T}_a$ son los 8 generadors de $SU(3)_c$ que en la representación trivial, la que siente Higgs, son;
 \begin{equation}
   \mathbb{T}_a = 0
  \end{equation}
  Con lo cual la derivada covariante del campo de Higgs es;
  \begin{equation}
    D_\mu H = \partial_\mu H + igW^A_\mu \frac{\sigma_A}{2}H + ig'\frac{1}{2}B_\mu H 
   \end{equation}
   Queda de tarea demostrar que en efecto, esta derivada es covariante, es decir;
   \begin{itemize}
     \item Bajo 
     \begin{equation*}
       H(x)\rightarrow \tilde{H}(x)= U(x)H(x)$
      \end{equation*}
       y hago;
     \begin{align*}
       W_\mu \rightarrow \tilde{W}_\mu = U(x)W_\mu U^{-1}(x) + \frac{i}{g} \partial_\mu U(x)U^{-1}(x) \\
       B_\mu \rightarrow \tilde{B}_\mu = B_\mu
     \end{align*}
     Entonces;
     \begin{equation}
       D_\mu H \rightarrow \tilde{D_\mu H} = U(x)D_\mu H 
      \end{equation}
      \item Bajo;
      \begin{align*}
        H(x)\rightarrow \tilde{H}(x) = e^{i\alpha_A \frac{g'}{2}} \\
        W_\mu \rightarrow \tilde{W}_\mu = W_\mu \\
        B_\mu \rightarrow \tilde{B}_\mu = B_\mu + \partial_\mu \alpha(x)
      \end{align*}
      Entonces;
      \begin{equation}
        D_\mu H \rightarrow \tilde{D_\mu H} = e^{i\alpha_A\frac{g'}{2}} D_\mu H 
       \end{equation}
   \end{itemize}
 Ahora tomamos la transformación del doblete de Higgs bajo $U(1)_y$ tal que;
 \begin{align*}
   D_\mu \tilde{H} & = \partial_\mu \tilde{H} + ig_{YM} W_\mu \tilde{H} + i\frac{1}{2}g_y\tilde{B}_\mu \tilde{H}  \\
   & = \partial_\mu \left( e^{-\frac{i}{2}\alpha(x)}\tilde{H} \right) + i\frac{1}{2}g_y \left( B_\mu + \partial_\mu \alpha(x) \right) \left( e^{-\frac{i}{2}\alpha(x)}H \right) + ig_{YM}W_\mu e^{-\frac{i}{2}g_y\alpha(x)} H \\
  & = -i\frac{g_y}{2} e^{-\frac{i}{2}} \partial_\mu \alpha(x) + e^{-\frac{i}{2}\alpha(x)}\partial_\mu H + i\frac{1}{2} g_y B_\mu e^{-i\frac{g_y}{2}\alpha(x)}H + i\frac{1}{2}g_y \partial_\mu \alpha(x) e^{-i\frac{g_y}{2}\alpha(x)}H + ig_{YM}W_\mu e^{-i\frac{g_y}{2}\alpha(x)}H \\
  & = e^{-i\frac{g_y}{2}\alpha(x)} \left( \partial_\mu H + ig_{YM}W_\mu H + i\frac{g_y}{2}B_\mu \right) \\
   & = e^{-i\frac{g_y}{2}\alpha(x)} D_\mu H
 \end{align*}
 El hermítico conjugado de la derivada covariante transforma como; 
 \begin{align*}
   \left( D_\mu H \right)^\dagger & = \partial_\mu H^\dagger - ig_{YM} H^\dagger W_\mu - i\frac{g_y}{2}B_\mu H^\dagger \\
   \left( D_\mu H  \right)^\dagger \rightarrow \left( D_\mu H \right)^\dagger '& = \left( \partial_\mu H^\dagger - ig_{YM}H^\dagger W_\mu + i\frac{g_y}{2}B_\mu H^\dagger \right)' \\
   & = D_\mu  H^\dagger e^{i\frac{g_y}{2}\alpha(x)}
 \end{align*}
En consecuencia, podemos escribir la acción de Yang-Mills para un campo de Guauge $SU(2)$ acoplado a un doblete en la representación fundamental de $SU(2)$, y con una hypercarga de $+\frac{1}{2}$, con un potencial de Higgs;
\begin{equation}
  I_{YM} = \int d^4x \; -\frac{1}{4}B_{\mu\nu} B^{\mu\nu} - \frac{1}{4}F_{\mu\nu}^A F^{\mu\nu}_A + \left( D_\mu H \right)^\dagger D_\mu H - \lambda \left( H^\dagger H \right)^2 + \mu^2 H^\dagger H
 \end{equation}
 En donde podemos notar que, como la conexión de Yang-Mills la hemos escalar o por su constante de acoplamiento $g_{YM}$, entonces
 \begin{equation}
   \mathfrak{L} = -\frac{1}{4g_{YM}^2} F^A_{\mu\nu}F_A^{\mu\nu} \rightarrow -\frac{1}{4}F_{\mu\nu}^A F_A^{\mu\nu}
  \end{equation}
  En donde el tensor de Faraday generalizado, o de field strenght;
  \begin{equation}
    F^C_{\mu\nu} = \partial_\mu A^C_\nu - \partial_\nu A^C_\mu - g_{YM} e_{ABC} A^A_\mu A^B_\nu
   \end{equation}
  Similarmente, el tensor asociado al campo $B_\mu$ es;
  \begin{equation}
    B_{\mu\nu} = \partial_\mu B_\nu - \partial_\nu B_\mu
   \end{equation}
\end{document}
