\documentclass[../main.tex]{subfiles}

\begin{document}
\section{Trigésima segunda clase}
Teoría de Gauge para;
\begin{equation}
  SU(2)_L\times U(1)_Y
 \end{equation}
 Con un campo de Higgs que transforma en la fundamental de $SU(2)$; dicho campo de Higgs es un doblete de campos escalares complejos;
 \begin{equation}
   H = \begin{pmatrix}
     H^+(x) \\ H^0(x)
   \end{pmatrix}
  \end{equation}
  Al cual se le asocian 3 números cuánticos;
  \begin{itemize}
    \item \textbf{1} : Transforma en la representación trivial de $SU(3)_C$ , o sea que los gluones no interactúan con el Higgs.
    \item \textbf{2} Transfomra en la representación fundamental de $SU(2)_L$.
    \item $1/2$ : Está asociado con la hypercarga y el grupo $U(1)_Y$
  \end{itemize}
  Bajo esto, a la transformación bajo el grupo de $U(1)_Y$ se le asocia el campo de Gauge $B_\mu$, tal que su término término cinétio está dado por;
  \begin{equation}
    B_{\mu\nu} := \partial_\mu B_\nu - \partial_\nu B_\mu
   \end{equation}
   Además, a la transformación del grupo $SU(2)_L$ se le asocia el campo de Gauge que depende de los generadores de la representación fundamental del grupo;
   \begin{equation}
     W_\mu = W^A_\mu T_A, \quad T_A: \text{generadores de }SU(2)_L
    \end{equation}
El campo de Higgs, se habla con los campos de Gauge via la derivada covariante;
\begin{equation}
  D_\mu H = \partial_\mu H + ig_{YM} W^A_\mu \frac{\sigma_A}{2} H + ig_{y}\frac{1}{2}B_\mu H
 \end{equation}
 A lo cual, la conexión asociada a $U(1)_y$ transforma como;
 \begin{equation}
   B_\mu \rightarrow \tilde{B}_\mu=B_\mu + \partial_\mu \alpha(x)
  \end{equation}
  Y el doblete de Higgs transforma como;
  \begin{equation}
    H(x)\rightarrow \tilde{H}(x) = e^{-i\frac{g_y}{2}\alpha(x)} H(x)
   \end{equation}
   transformación que deja invariante la conexión asociada al grupo de $SU(2)_L$;
   \begin{equation}
     W^A_\mu \rightarrow \tilde{W}^A_{\mu} = W^A_\mu(x)
    \end{equation}
    Tal que, la derivada covariante, transformando bajo el grupo $U(1)_y$;
    \begin{equation}
      \tilde{D_\mu H} := \partial_\mu \tilde{H} + ig_{YM} \tilde{W}^A_\mu \frac{\sigma_A}{2} \tilde{H} + ig_y \frac{1}{2} \tilde{B}_\mu \tilde{H} = e^{-ig_y\frac{1}{2}\alpha(x)}\left( D_\mu H \right)
    \end{equation}
Luego, la transformación bajo la representación fundamental del grupo $SU(2)_L$, la conexión $B_\mu$ no se ve afectada bajo esta transformación;
\begin{equation}
  B_\mu \rightarrow \tilde{B}_\mu(x) = B_\mu(x)
 \end{equation}
 El campo de Higgs en doblete;
 \begin{equation}
   H(x)\rightarrow \tilde{H}(x) = U(x) H(x)
  \end{equation}
 La conexión asociada a $SU(2)$ transforma como;
 \begin{equation}
   \mathbb{W}_\mu \rightarrow = \tilde{\mathbb{W}}_\mu (x) = U(x)\mathbb{W}_\mu (x)U^{1}(x)+\frac{i}{g_{YM}} \partial_\mu U(x) U^{-1}(x).
  \end{equation}
  Tal que la derivada covariante transforma, bajo $SU(2)_L$ en su representación fundamental;
  \begin{equation}
    \tilde{D_\mu H} = U(x) \left( D_\mu H \right)
   \end{equation}
\subsection{Lagrangeano en el sector electrodébil sin fermiones}
El Lagrangeano del sector electrodébil del modelo estándar, bajo la convención $(+,-,-,-)$ es (fin feriomes);
\begin{equation}
  \mathfrak{L} = \left( D_\mu H \right)^\dagger \left( D_\mu H \right) - \frac{1}{2}tr \left( \mathbb{W}_{\mu\nu}\mathbb{W}_{\mu\nu} \right) - \frac{1}{4}B_{\mu\nu} B^{\mu\nu} - V(H)
 \end{equation}
 Con el potencial de Higgs;
 \begin{equation}
   V(H) = -\mu^2 H^\dagger H + \lambda \left( H^\dagger H \right)^2
  \end{equation}
  En donde la derivada covariate está dad por;
  \begin{equation}
    D_\mu H = \partial_\mu H +i \frac{g_{YM}}{2} \mathbb{W} + ig_y \frac{1}{2}B_\mu H
   \end{equation}
   Y si hermítico conjugado;
   \begin{equation}
     \left( D_\mu H \right)^\dagger =\partial_\mu H^\dagger -ig_{Y}W^A_\mu H^\dagger \frac{\sigma_A}{2} - ig_y \frac{1}{2}H^\dagger B_\mu 
   \end{equation}
   Con lo cual esta teoría es invariante local bajo;
   \begin{equation}
     SU(2)_L \times U(1)_y
    \end{equation}
 En $SU(2)_L$;
 \begin{align*}
   \tilde{H}^\dagger \tilde{H} & = \left( U(x)H \right)^\dagger \left( U(x)H \right) \\
   & = H^\dagger U^\dagger U H  =H^\dagger H
\end{align*}
y en $U(1)_y$;
\begin{align*}
  \tilde{H}^\dagger \tilde{H} = \left( e^{-i\alpha(x)\frac{g_y}{2}} H \right)^\dagger \left( e^{-i\alpha(x)\frac{g_y}{2}} H\right) = H^\dagger H
\end{align*}
\subsection{Mecanismo de Higgs}
Para una función;
\begin{equation}
  f(x) = -\mu^2x + \lambda x^2
 \end{equation}
 tal que su derivada es;
 \begin{equation}
   f'(x) = -\mu^2 + 2\lambda x
  \end{equation}
Tal que el $x$ mínimo estará dado por;
\begin{equation}
  x_{min} = \frac{\mu^2}{2\lambda}
 \end{equation}
 En el caso abelian o el potencial es del tipo sombrero mexicano;
 \begin{equation}
   V(\phi,\phi^*) = -\mu^2 \phi^* \phi + \lambda \left( \phi^* \phi \right)^2
  \end{equation}
  Con $\phi=\nu$ y $\phi^* \phi= \nu^2$ 
  El mínimo del potencial de Higgs no-abeliano está en;
  \begin{equation}
    H^\dagger H = \frac{\mu^2}{2\lambda}
   \end{equation}
   Los vacíos posibles de la teoría, que minimizan el potecial de Higgs, serán configuraciones de $H$ constantes, tal que, un $H$ tal es;
   \begin{equation}
     H = \frac{1}{\sqrt{2}} \begin{pmatrix}
       0 \\ \nu 
     \end{pmatrix} , \quad \text{con}\; \nu = \sqrt{\frac{\mu^2}{\lambda}}
    \end{equation}
    Así, término por sus componentes;
    \begin{align*}
      H^\dagger H & = \frac{1}{2} (0\; ,\; \nu) \begin{pmatrix}
        0 \\ \nu
      \end{pmatrix} \\
      & = \frac{1}{2}\frac{\nu^2}{\lambda}
    \end{align*}
    Lo que concuerda con el resultado obtenido anteriormente.\\
    H está definido por 4 números reales y se puede parametrizar en general, de la siguiente forma;
    \begin{equation}
      \begin{pmatrix}
        H^+(x) \\ H^0(x)
      \end{pmatrix} = H = e^{i\xi^A(x)\frac{\sigma_A}{2}}\begin{pmatrix}
        0 \\ \chi(x)
      \end{pmatrix}, \quad \text{4 campos.}
     \end{equation}
Ahora podemos transformar nuestro Higgs, pero bajo la paramtrización;
\begin{align*}
  \tilde{H}& = e^{i\xi^A(x)\frac{\sigma_A}{2}} H(x) \\
 & = \begin{pmatrix}
   0 \\ \chi(x)
 \end{pmatrix}
\end{align*}
Gaugear away, 
\begin{equation}
  H(x) = \begin{pmatrix}
    0 \\ \chi(x)
  \end{pmatrix}
  = \frac{1}{\sqrt{2}}\begin{pmatrix}
    0 \\ \nu + h(x)
  \end{pmatrix}
  = H(x),\quad h(x)\in \mathbb{R}
 \end{equation}
 Luego, construimos la derivada covariante con este Higgs;
 \begin{align*}
   D_\mu H & = \left( \partial_\mu  + ig_{YM} \frac{\sigma^A}{2} W^A_{\mu} + i g_y \frac{1}{2}B-\mu \right)  \frac{1}{\sqrt{2}} \begin{pmatrix}
     0 \\ \nu + h(x)
   \end{pmatrix} \\
   & = \left[ \begin{pmatrix}
    1 & 0 \\ 0 & 1
   \end{pmatrix}\partial_\mu + \frac{i}{2} \begin{pmatrix}
     g_{YM} W^3_\mu + g_y B_\mu & g \left( W^1_\mu -i W^2_\mu \right) \\ g_{YM} \left( W^1_\mu + i W^2_\mu  \right) & - g_{YM} W^3_\mu + g_y B_\mu
   \end{pmatrix} \right] \frac{1}{\sqrt{2}} \begin{pmatrix}
     0 \\ \nu + h(x)
   \end{pmatrix}
 \end{align*}
 Queda de tarea expandir la expresión. \\
A su vez, el término cinético de Higgs es;
\begin{align*}
  \left( D_\mu H \right)^\dagger \left( D_\mu H \right) & = \frac{1}{2}\partial_\mu h(x) \partial^\mu h(x) + \frac{g_{YM}^2}{8} \left[ W^1_\mu W^{1\mu} + W^2_\mu W^{2\mu} \right] \left( 1 + \frac{h^2(x)}{\nu} \right)^2 \\ & + \frac{\nu^2}{8} \left( g_{YM}W^3_\mu - g_yB_mu  \right) \left( g_{YM} W^{3\mu} - B^\mu \right) \left( 1 + \frac{h(x)}{\nu} \right)^2
\end{align*}
Ahora definimos;
\begin{align}
  z^0_\mu & = \cos{\theta}W^3_\mu - \sin{\theta}B_\mu, \quad \theta: \text{Weak mixing angle} \\
  A_\mu & = \sin{\theta}W^3_\mu + \cos{\theta}B_\mu, \quad \tan{\theta} = \frac{g_{y}}{g_{YM}}
\end{align}
El que tenía la pinta de ser el campo de Gauge de electrodinámica, o sea, $B_\mu$ es en verdad el campo $A_\mu$ que es combinación de este, junto con el tercer campo de Gauge $W^3_\mu$.  
Con lo cual, podemos obtener finalmente;
\begin{equation}
  \left( D_\mu H \right)^\dagger \left( D_\mu H  \right) = \frac{1}{2}\partial_\mu h \partial^\mu h + \frac{g_{YM}^2}{8} \left[ W^1_\mu W^{1\mu} + W^2_\mu W^{2\mu} \right] \left( 1 + \frac{2h(x)}{\nu} + \frac{h^2(x)}{\nu^2} \right) + \frac{1}{8} \frac{\nu^2g_{YM}^2}{\cos^2{\theta}} z^0_\mu z^{0\theta}
 \end{equation}
 Con lo cual hemos encontrado dos campos vectoriales masivos iguales y otro más con masa diferente, que era lo que buscabamos;
 \begin{align*}
   m_{W^\pm} & = \frac{g\nu}{2} \\
   m_z & = \frac{g\nu}{2\cos{\theta}}
 \end{align*}
tal que;
\begin{equation}
  m_{z^0} > m_{W^\pm}
 \end{equation}
 Hola
\end{document}
