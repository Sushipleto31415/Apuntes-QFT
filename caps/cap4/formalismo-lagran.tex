\documentclass[../main.tex]{subfiles}

\begin{document}
\section{Formalismo Lagrangeano de la Mecánica Clásica}
Si las fuerzas de un sistema pueden ser obtenidas mediante un potencial escalar \textbf{V} (las fuerzas son conservativas)
\begin{equation}
    F=-\nabla_iV
\end{equation}
Cuando esto sucede, tenemos las \textbf{ecuaciones de Lagrange de primera especie} están dadas por
\begin{equation}
    \frac{d}{dt}\left( \frac{\partial \emph{L}}{\partial \dot{g}^j} \right) - \frac{\partial \emph{L}}{\partial q^j}=0
\end{equation}
Para lo cual, la funcion \textbf{Lagrangeano}, $\textbf{L}(q^j,\dot{q}^j,t)$ debe estar definida de la siguiente manera.
\begin{equation}
    L=T-V
\end{equation}
Para lo cual, \textbf{T} corresponde a la energía cinética y \textbf{V} corresponde al potencial  
Notar que, para un set particular de ecuaciones, no existe un único \textbf{Lagrangeano}. \\
Luego, las \textbf{Ecuaciones de Lagrange de segunda especie} se definen por la ecuación
\begin{equation}
    \frac{d}{dt}\left( \frac{\partial \emph{L}}{\partial \dot{g}^j} \right) - \frac{\partial \emph{L}}{\partial q^j}=Q_j
\end{equation}
Para lo cual la funcion \textbf{Lagrangeano} (\emph{\textbf{L}}) contiene todas las fuerzas conservativas, y para el caso que en el sistema existan fuerzas no conservativas, o sea, que no estén presentes en el potencial \emph{\textbf{V}}, estas estarán contenidas en la función $\emph{\textbf{Q}}_j$(creo que se llama función de disipación). La cual está definida por la siguiente expresión.
\begin{equation}
    Q^j=-\frac{\partial U}{\partial q^j}+\frac{d}{dt}\left( \frac{\partial U }{\partial \dot{q}^j}\right)
\end{equation}
Para alguna funcion $\textbf{U}(q^j,\dot{q}^j)$ la  cual es llamada, \textbf{potencial generalizado}. \\
En cuyo caso, la función \textbf{Lagrangeano} deberá ser definida de la siguiente manera
\begin{equation}
    L=T-U
\end{equation}
\section{Fuerzas centrales}
\begin{equation}
    \emph{v}_a=\frac{1}{2}r^2\dot{\phi}
\end{equation}
\textbf{Ecuacion de Binet}
\begin{equation} \label{eq:binet}
    -\frac{l^2}{mr^2} \left( \frac{d^2}{d \phi^2}\left( \frac{1}{r}\right)+\frac{1}{r} \right)=f(r)
\end{equation}


\end{document}