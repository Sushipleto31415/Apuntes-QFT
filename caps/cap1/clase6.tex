\documentclass ../main.tex]{subfiles}

\begin{document}
\section{Sexta clase}
Volvemos a relatividad especial

\begin{align*}
  \tilde{t} & =\frac{t-V/c^2 x}{\sqrt{1-V^2/c^2}} \\
  \tilde{x} & =  \frac{x-Vt}{\sqrt{1-V^2/c^2}} \\
  V & = c\tanh{\xi}
\end{align*}
Ahora, si definimos $V$ así
\begin{align*}
  1-\frac{V^2}{c^2}  & = 1- \tanh^2{\xi}
  & = \frac{1}{\cosh^2{\xi}}
\end{align*}
Con lo cual recordemos las propiedades que cumplen las funciones trigonométricas hiperbólicas
\begin{align}
  \cosh^2{\xi} - \sinh^2{\xi}  & = 1 \\
  1- \tanh^2{\xi} = \frac{1}{\cosh^2{\xi}}
\end{align}
Por lo tanto, podemos escribir nuestras transformaciones como lo siguiente 
\begin{align*}
  \tilde{t} & = \frac{t-\frac{\tanh{\xi}}{c}x}{\frac{1}{\cosh{\xi}}} \\
  c\tilde{t} & = \cosh{\xi}ct - \sinh{\xi}x \\
  \tilde{x}  & = \cosh{\xi}x - \sinh{\xi}ct \\
  \tilde{z} & = z  \\
  \tilde{y} & = y
\end{align*}
En lo cual, $\sinh{\xi}$ será la rapidity. \\
Ahora, escribamos esto en notación tensorial de la siguiente forma
\begin{equation}
  x^\mu = \{ x^0,x^1,x^2,x^3 \} = \{xt,x,y,z\}
\end{equation}
Lo cual representa las coordenadas de un evento en el espacio tiempo plano (Minkowsky). \\
(Escribió la transformación en matrices, revisar como escribir las matrices) \\
Aplicamos las transformaciones en forma matricial
\begin{equation}
  \tilde{x}^\mu = \{\tilde{x}^0,\tilde{x}^1, \tilde{x}^2, \tilde{x}^3\} = \{c\tilde{t},\tilde{x},\tilde{y},\tilde{z}\} 
\end{equation}
Ahora, la matriz de transformación se representa como un tensor $\Lambda_\nu^\mu$ de transformacion para el tensor de posiciones $x^\mu$ tal que, la regla de transformación para la posición dentro del espacio tiempo plano de Minkowsky, es bajo la siguiente regla
\begin{equation}
  \tilde{x}^\mu=\Lambda^\mu_\nu x^\nu
\end{equation}
en lo cual los índices $\nu$ están contraidos y por tanto, sumados. \\
Pensamos ahora, en una transformación más general y preguntémonos ¿ Qué nos dice la invariancia del intervalo con respecto de $\Lambda^\mu_\nu$?
\begin{align*}
  \tilde{\vec{x}} &  = \Theta \vec{x} \\
  ||\tilde{\vec{x}}||^2 & = ||\vec{x}||^2 \\
  \tilde{\vec{x}}^T\tilde{\vec{x}} & = \vec{x}^T\vec{x} \\
  (\Theta\vec{x})^T(\Theta\vec{x}) & = \vec{x}^T\vec{x} \\
  \vec{x}^T\Theta^T\Theta\vec{x}   & = \vec{x}^T\vec{x} 
\end{align*}
En lo cual, las matrices $\Theta$ se definen como ortogonales, o sea que, $\Theta^T\Theta= I$
\begin{align*}
  ds^2  & = c^2dt^2 - dx^2 - dy^2 - dz^2 \\
  & = c^2d\tilde{t}^2 - d\tilde{x}^2 - d\tilde{x}^2 - d\tilde{z}^2
\end{align*}
Para lo cual
\begin{align*}
  ds^2 = \sum_{\mu=0}^3\sum_{\nu=0}^3\eta_{\mu \nu}dx^\mu dx^\nu
\end{align*}
Donde $\eta_{\mu \nu} = diag(1,-1,-1,-1)$ la cual corresponde a la métrica de Minkowsky, ahora, expandamos la suma de la métrica $ds^2$, primero sumamos la suma, valga la redundancia, en el índice $\nu$.
\begin{align*}
  ds^2  & = \sum_{\nu=0}^3 \left( \eta_{\mu 0}dx^\mu dx^0 + \eta_{\mu 1}dx^\mu dx^1 + \eta_{\mu 2}dx^\mu dx^2 + \eta_{\mu 3}dx^\mu dx^3 \right) \\
  & = \eta_{00}(dx^0)^2 + \eta_{11}(dx^1)^2 + \eta_{22}(dx^2)^2 + \eta_{33}(dx^3)^2 \\
  & = c^2dt^2-dx^2-dy^2-dz^2
\end{align*}
Por tanto
\begin{equation*}
  ds^2 = \eta_{\mu \nu }dx^\mu dx^\nu = \eta_{\mu \nu } d\tilde{x}^\mu d\tilde{x}^\nu 
\end{equation*}
Pero queremos que
\begin{align*}
  \tilde{x}^\mu & = \Lambda_\alpha^\mu x^\alpha \rightarrow d\tilde{x}^\mu = \Lambda^\mu_\alpha dx^\alpha \\
  \tilde{x}^\nu & = \Lambda_\beta^\nu x^\beta \rightarrow d\tilde {x}^\nu   = \Lambda_\beta^\nu dx^\beta 
\end{align*}
Con lo cual, obtenemos que
\begin{align*}
  \eta_{\cancel{\mu}_\alpha \cancel{\nu}_\alpha} dx^{\cancel{\mu}^\alpha} dx^{\cancel{\nu}^\beta}  & = \eta_{\mu \nu } \left( \Lambda_\alpha^\nu dx^\alpha \right)\left( \Lambda_\beta^\nu dx^\beta \right) \\
  \left(\eta_{\mu \nu}\Lambda_\alpha^\nu \Lambda_\beta^\nu-\eta_{\alpha \beta}\right)dx^\alpha dx^\beta & = 0
\end{align*}
Por tanto, tenemos que
\begin{equation}
  \eta_{\mu \nu} \Lambda^\mu_\alpha \Lambda_\beta^\nu = \eta_{\alpha \beta}
\end{equation}
tenemos que $\Lambda$ será una transformación de Lorenz si ocurre lo anterior
\begin{align*}
 \Lambda_\alpha^\mu \eta_{\mu \nu} \Lambda_\beta^\nu & = \eta_{\alpha \beta} \\
  \Lambda^T \eta \Lambda & = \eta 
\end{align*}
versus 
\begin{align*}
  \Theta^T I \Theta = I
\end{align*}
En lo cual se usa la siguiente propiedad
\begin{align*}
  \xi^T C \xi = C 
\end{align*}
Siendo c una matriz antisimétrica con respecto a la diagonal y además diagonal nula. \\
Definición: Diremos que un arreglo denotado por $A^\alpha$ es un vector contravariante de Lorentz si bajo una transoformación de Lorentz: 
\begin{equation}
  \tilde{A}^\alpha = \Lambda_\beta^\alpha A^\beta
\end{equation}
Las coordenadas $x^\mu$ definen un vector contravariante, ahora, un vector contravariante sería el 4-momenta. \\
Ahora, se tiene un observador $\tilde{K}$ el cual se mueve a una velocidad constante $V$ con respecto a un observador $K$, ambos observadores son inerciales. \\
¿ Dados $E$ y $P$, cómo encontramos $\tilde{E}$ y $\tilde{P}$ ? 
\begin{equation*}
  E= \frac{mc^2}{\sqrt{1-\frac{v^2}{c^2}}} \quad , \quad P=\frac{mv}{\sqrt{1-\frac{v^2}{c^2}}}
\end{equation*}
versus sus tildas
\begin{equation*}
  \tilde{E} = \frac{mc^2}{\sqrt{1-\frac{\tilde{v}^2}{c^2}}}\quad , \quad \tilde{P}= \frac{m\tilde{v}}{\sqrt{1-\frac{\tilde{v}^2}{c^2}}}
\end{equation*}
Ahora, aplicamos la transformación para la velocidad $\tilde{v}$ tal que
\begin{align*}
  \tilde{E} & = \frac{mc^2}{\sqrt{1-\frac{1}{c^2}\left( \frac{v-V}{1-\frac{vV}{c^2}} \right)^2}}\\
  & = \frac{c\left( 1-\frac{vV}{c^2} \right)mc^2}{\sqrt{c^2\left( 1-\frac{vV}{c^2} \right)^2 - (v-V)^2}} \\
  & = \frac{c(1-\frac{vV}{c^2})mc^2}{\sqrt{1-\frac{V^2}{c^2}+ \frac{v^2V^2}{c^2}-\frac{v^2}{c^2}}} \\
  & = \frac{mc^2}{\sqrt{1-\frac{V^2}{c^2} + \frac{v^2V^2}{c^2}} - \frac{v^2}{c^2}}  - \frac{mvV}{\sqrt{1-\frac{V^2}{c^2} + \frac{v^2V^2}{c^2} - \frac{v^2}{c^2}}}\\
  & = \frac{\frac{mc^2}{\sqrt{1-\frac{v^2}{c^2}}}}{\sqrt{1-\frac{V^2}{c^2}}} - \frac{V\frac{mv}{\sqrt{1-\frac{V^2}{c^2}}}}{\sqrt{1-\frac{V^2}{c^2}}} \\
  \frac{\tilde{E}}{c}& = \frac{\frac{E}{c}-\frac{V}{c}P}{\sqrt{1-\frac{V^2}{c^2}}}
\end{align*}
Versus
\begin{align*}
  c\tilde{t} =\frac{ct-\frac{V}{c}x}{\sqrt{1-\frac{V^2}{c^2}}}
\end{align*}
Tarea, 
\begin{align*}
  \tilde{p}=\frac{m\tilde{v}}{\sqrt{1-\frac{\tilde{v}^2}{c^2}}} = \frac{p-\frac{V}{c}\left( \frac{E}{c} \right)}{\sqrt{1-\frac{V^2}{c^2}}}
\end{align*}
versus
\begin{equation*}
  \tilde{x} = \frac{x-\frac{V}{c}(c\tilde{t})}{\sqrt{1-\frac{V^2}{c^2}}}
\end{equation*}
Definición: 4-momenta o cuadrivector de momentum , de define como 
\begin{align}
  p^\mu = \left( \frac{E}{c},\vec{p} \right)\quad , \quad \tilde{p}^\mu = \left(\frac{\tilde{E}}{c}, \tilde{\vec{p}}\right)
\end{align}
y además
\begin{equation}
  \tilde{p}^\mu = \Lambda^\mu_\nu p^\nu
\end{equation}
El cuadrimoemento es un vector contravariante de Lorentz.

\end{document}
