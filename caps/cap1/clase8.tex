\documentclass ../main.tex]{subfiles}

\begin{document}
\section{Clase 8 (Inicios del grupo de lorentz)}
Al estudiar las transformaciones de Lorentz, se podrá considerar simplemente un boost a lo largo del eje x. Tal que esta transformación afecta a las etiquetas espacio tiempo de la forma: 

\begin{equation}
  \tilde{x}^\mu = \Lambda_\nu^\mu x^\nu
\end{equation}
<<<<<<< HEAD
Ahora usaremos que $\boxed{c=1}$, un boost a lo largo del eje x sería
\begin{equation}
  \Lambda = \begin{pmatrix}
    \gamma & -\beta \gamma & 0 & 0 \\
    -\beta \gamma & \gamma & 0 & 0 \\
     0 & 0 & 0 & 0 \\
     0 & 0 & 0 & 0 
    \end{pmatrix}
\end{equation}
Lo cual se traduce con
\begin{align*}
  \gamma : = \left(1-\frac{V^2}{c^2}\right)^{-1/2} \\
  \beta : =  \frac{V}{c} \\
\end{align*}
Lo cual, con $c=1$ queda como
\begin{align*}
  \tilde{t} & = \frac{t-Vx}{\sqrt{1-V^2}} \\
  \tilde{x} & = \frac{x-Vt}{\sqrt{1-V^2}} \\
  \tilde{{y}} & = y \\
  \tilde{z} & = z
\end{align*}
La transformación del tipo
\begin{equation*}
  \tilde{x}^\mu = \Lambda_\nu^\mu x^\nu
\end{equation*}
que dejan invariante el intervalo
=======

%Y considerando que $\boxed{c=1}$, la matriz que define esta transformación se verá: 
%\begin{equation}
%  \Lambda = \begin{pmatrix}
%    \gamma & -\beta \gamma & 0 & 0 \\
%    -\beta \gamma & \gamma & 0 & 0 \\
%     0 & 0 & 0 & 0 \\
%     0 & 0 & 0 & 0 
%    \end{pmatrix}
%\end{equation}

%Con los parámetros definidos como:
%\begin{align*}
%  \gamma : = \left(1-\frac{V^2}{c^2}\right)^{-1/2} \\
%  \beta : =  \frac{V}{c} \\
%\end{align*}
%Recordando que definimos a $c=1$ queda como:
%\begin{align*}
%  \tilde{t} & = \frac{t-Vx}{\sqrt{1-V^2}} \\
%  \tilde{x} & = \frac{x-Vt}{\sqrt{1-V^2}} \\
%  \tilde{y} & = y \\
%  \tilde{z} & = z
%\end{align*}

%La transformación del tipo
%\begin{equation*}
%  \tilde{x}^\mu = \Lambda_\nu^\mu x^\nu
%\end{equation*}

Tal que dejan invariante el intervalo
>>>>>>> refs/remotes/origin/main
\begin{equation*}
  \eta_{\mu \nu} dx^\mu dx^\nu = \eta_{\mu \nu}d\tilde{x}^\mu d\tilde{x}^\nu
\end{equation*}

Y cumplirán
\begin{equation*}
  \eta_{\mu \nu} = \Lambda_\mu^\alpha \Lambda^\beta_\nu \eta_{\alpha \beta}
\end{equation*}
¿Cuál es la forma más general que puede tomar $\Lambda_\nu^\mu$ tal que deje invariante el intervalo? \\
La matriz Lambda puede tener 6 familias de transformaciones diferentes, las cuales son:
\begin{itemize}
<<<<<<< HEAD
  \item $\Lambda_\nu^\mu \rightarrow {\Lambda_\mu}_{\text{boost a lo largo del eje x}\nu}(v)$ 
  \item $\Lambda_{\text{boost a lo largo del eje y}}(v)$
  \item $\Lambda_{\text{boost a lo largo del eje z}}(v)$ 
=======
  \item $\Lambda_\nu^\mu\rightarrow \Lambda_{\text{boost a lo largo del eje x}}(v)$ 
  \item $\Lambda_\nu^\mu\rightarrow \Lambda_{\text{boost a lo largo del eje y}}(v)$
  \item $\Lambda_\nu^\mu\rightarrow \Lambda_{\text{boost a lo largo del eje z}}(v)$ 
>>>>>>> refs/remotes/origin/main
  \item $\Lambda_\mu^\nu \rightarrow$ Rotación en el plano (x,y) $(\theta)$
  \item $\Lambda_\nu^\mu\rightarrow$ Rotación en el plano (y,z) $(\theta)$
  \item $\Lambda_\nu^\mu\rightarrow$ Rotación en el plano (z,x) $(\theta)$
\end{itemize}

Tal que cada uno de estos $\Lambda$ esta asociado a diferentes parámetros. Veamos cada uno de ellos: 

\subsection{Inversión temporal:}
Definiendo la transformación a lo largo del eje temporal como: 

\begin{equation*}
  \tilde{t}= -t \quad , \tilde{x}=+x \quad ,\tilde{y}= y \quad , \tilde{y}=y \quad , \tilde{z}=z 
\end{equation*}

Tal que la matriz asociada a la transformación tiene la forma: 

\begin{equation}
  \Lambda_\mu^\nu = \begin{pmatrix}
    -1 & 0 & 0 & 0 \\
    0 & 1 & 0 & 0 \\
    0 & 0 & 1 & 0 \\
    0 & 0 & 0 & 1 
  \end{pmatrix}
\end{equation}

Y dejará al intervalo invariante. 

\begin{equation*}
  d\tilde{t}^2 -d\tilde{x}^2 - d\tilde{y}^2-d\tilde{z}^2 = dt^2-dx^2-dy^2-dz^2
\end{equation*}

\subsection{Transformaciones de paridad:}

Definiendo la transformación de paridad como: 

\begin{equation*}
  \tilde{t}= t \quad , \tilde{x}= -x \quad , \tilde{y}=-y \quad , \tilde{z}=-z
\end{equation*}

Tal que la matriz asociada a la transformación tiene la forma: 
\begin{equation*}
  \Lambda_\nu^\mu = \begin{pmatrix}
    1 & 0 & 0 & 0 \\
    0 & -1 & 0 & 0 \\
    0 & 0 & -1 & 0 \\
    0 & 0 & 0 & -1
  \end{pmatrix} 
\end{equation*}

Es interesante preguntarse "¿Qué sentido físico tienen este tipo de transformaciones?" y "¿Con que otras nociones físicas puedo conectar esta transformación?". \\

Tal y como dice su nombre, veremos que esta transformación tendrá que ver con la paridad a lo largo de los ejes espaciales. Así, será como si dibujasemos un espejo transversalmente a lo largo de los ejes temporales tal y como en [referencia a dibujo oliva]. \\

Vemos que la interacción electromagnética si será invariante bajo transformaciones de inversión temporal y transformación de paridad. En cambio la interacción débil \textbf{no será invariante bajo transformaciones de paridad}, pero si lo será bajo boosts y rotaciones. \\

Esto se ve reflejado cuando analizamos el caso del decaimiento de $^{60}Co$, el cual será mediado por la interacción débil, especificamente por el bosón Z. \\

Ejemplo: (insertar dibujo de Julio) \\

Observamos acá que en el lado izquierdo la dirección del spin $\vec{s}$ coincide con la que se dirigen las partículas ejectadas. En cambio, en el lado derecho, el spin va hacía el otro lado. Esto último no se ha observado en la naturaleza, pues solo se ha visto que la dirección del spin coincida con la que se dirigen las partículas ejectadas. Por tanto, la interacción débil rompe la símetria CPT. 

%Como consecuencia que los neutrinos solo sean izquierdos, se le asocia cierta quiralidad a la interacción débil y que finalmente significa que la interacción débil rompe la paridad. (Solo con la interacción débil). \\


\subsection{Campo escalar bajo transformaciones de Lorentz} 

El campo relativista más simple posible será el campo escalar. Pues teniendo a dos observadores inerciales tales que la velocidad de $\tilde{K}$ con respecto a $K$ es $V$. En un punto p, tal que su posición respecto a $K$ y $\tilde{K}$ sea $x^\mu$ y $\tilde{x}^\mu$ respectivamente. Vemos que $\phi$ será un campo escalar si al evaluarlo con cada observador se cumple:

\begin{equation*}
    \tilde{\phi}(\tilde{x}^\mu) = \phi(x) \label{campo-escalar}
\end{equation*}

Lo que significa que independiente del observado desde el que se mida, el campo escalar debe ser el mismo. Ahora, si consideramos las interacciones entre partículas y los campos que las median notamos que el campo de Higgs será un campo escalar de Lorentz. En cambio el potencial eléctrico, que es un campo escalar usual, no lo será.

Podremos reescribir \ref{campo-escalar} como: 

\begin{align*}
  (\hat{\phi})(x^\mu) & = 1 \phi(x^\mu) \\
  & = e^{\frac{1}{2}\omega^{\alpha \beta}J_{\alpha \beta}}\phi^{x^\mu} \\
\end{align*}

Con $J_{\alpha \beta}=0$. \\
Por otro lado, se tiene que para la posición $\tilde{x}^\mu = \Lambda_\nu^\mu x^\nu$ la derivada con respecto al observador $K$ y  $\tilde{K}$ están dadas por:
\begin{equation*}
<<<<<<< HEAD
  \partial_\mu \phi = \left\{\partial_t\phi,\partial_x\phi,\partial_y\phi,\partial_z\phi\right\}
=======
  \partial_\mu \phi = \left\{\partial_t\phi,\partial_x\phi,\partial_y\phi,\partial_z\phi \right\}
>>>>>>> refs/remotes/origin/main
\end{equation*}

\begin{equation*}
  \partial_{\tilde{\mu}}\tilde{\phi} = \left\{\partial_{\tilde{t}\tilde{\phi}}, \partial_{\tilde{x}}\tilde{\phi} , \partial_{\tilde{y}}\tilde{\phi} , \partial_{\tilde{y}}\tilde{\phi} , \partial_{\tilde{z}}\tilde{\phi}\right\}
\end{equation*}

Comparando a $\partial_\mu\phi$ y $\partial_{\tilde{\mu}}\tilde{\phi}$ considerando $\tilde{t}=\tilde{t}(t,x)$, continuamos desarrollando las expresiones:

\begin{align*}
  \partial_{\tilde{t}}\tilde{\phi}(\tilde{t},\tilde{x}) & = \partial_{\tilde{t}}t \partial_t\tilde{\phi}(\tilde{t}(t,x),\tilde{x})(t,x) + \partial_{\tilde{t}}x\partial_x\tilde{\phi}(\tilde{t}(t,x),\tilde{x}(t,x)) \\
  & = \partial_{\tilde{t}}t \partial_t\phi + \partial_{\tilde{t}}x \partial_x\phi
\end{align*}
Ahora, considerando la dependencia de $\tilde{x}^\mu = \tilde{x}^\mu (x^\alpha)$
\begin{align*}
  \partial_{\tilde{\mu}}\tilde{\phi}(\tilde{x}) & = \partial_{\tilde{\mu}}x^\alpha \partial_\alpha \tilde{\phi}(\tilde{x}) \\
  & = \partial_{\tilde{\mu}}\tilde{\phi}(\tilde{x}^\mu) \\
  &  =\partial_{\tilde{\mu}}x^\alpha \partial_\alpha \phi(x) \label{wea-1}
\end{align*}

Por otro lado si consideramos a $\tilde{x}^\mu = \Lambda_\nu^\mu x^\nu$ y multiplicamos por $\left(\Lambda^{-1}\right)^\gamma_\mu$

\begin{align*}
  \left(\Lambda^{-1}\right)^\gamma_\mu \tilde{x}^\mu & =  \Lambda_\nu^\mu x^\nu \\
  & = \delta_\nu^\gamma x^\nu \\
  & = x^\gamma
\end{align*}
Por lo tanto $x^\gamma= \left(\Lambda^{-1}\right)^\gamma_\mu \tilde{x}^\mu$ con lo cual, ahora si podemos encontrar como transforma el cuadri-gradiente
\begin{align*}
  \partial_{\tilde{\xi}}x^\gamma & = \left(\Lambda^{-1}\right)^\gamma_\mu \partial_{\tilde{\xi}}\tilde{x}^\mu \\
  \left( \Lambda^{-1} \right)^\gamma_\mu \delta_\xi^\mu \\
\end{align*}
Por lo tanto 
\begin{equation}
  \partial_{\tilde{xi}}x^\gamma = \left(\Lambda^{-1}\right)^\gamma_\xi \label{wea-2}
\end{equation}
Ahora si, $\tilde{x}^\mu = \Lambda_\nu^\mu x^\nu$ e introducimos \ref{wea-2} en \ref{wea-1} obtenemos: 
\begin{equation*}
  \partial_{\tilde{\mu}}\tilde{\phi} = \left(\Lambda^{-1}\right)^\alpha_\nu \partial_\alpha \phi
\end{equation*}
Que será como transforma el cuadri-gradiente de un escalar de Lorentz. Este será un conjunto de cuatro números qué transforman entre ellos de manera lineal homogénea, pero no definen un cuadri-vector contravariante ni covariante. Estos se definen como: \\

\textbf{Cuadri-vector contravariante:} 
\begin{equation}
  \tilde{{A}}^\alpha = \Lambda^\alpha_\beta A^\beta 
\end{equation}
\textbf{Cuadri-vector covariante:}
\begin{equation}
<<<<<<< HEAD
\tilde{B}_\alpha = \left(\Lambda^{-1}\right)^\beta_\alpha B_\beta
=======
\tilde{B}_\alpha = \left(\alpha^{-1}\right)^\beta_\alpha B_\beta
>>>>>>> refs/remotes/origin/main
\end{equation}

\section{Nociones de Grupos}
\begin{align*}
  G=\left\{a,b,\dots\right\} \\
  \star : G \times G \rightarrow G \\
\end{align*}
El cual cumplirá con las siguientes propiedades 
\begin{itemize}
  \item $g_1 \star (g_2 \star g_3) = (g_1\star g_2)\star g_3$ 
  \item Existe una identidad $e$ perteneciente a G tal que $e\star g_i = g_i \star e = g_i \forall g_i \in G$
  \item $\forall g \in G$ existe un $g^{-1}\in G$ tal que $g\star g^{-1} = g^{-1}\star g = e$ 
\end{itemize}
Tomemos por ejemplo el grupo $G=U(1)$
Multiplicación $g(\xi)= e^{i\xi}$ números $\xi \in [0,2\pi)$ \\
El otro grupo será el grupo SO(2), con $0\leq\theta\leq 2\pi$ el cual es el grupo de rotaciones y corresponde a una multiplicación matricial
\begin{equation} g(\theta)=
  \begin{pmatrix}
    \cos{\theta} & \sin{\theta} \\
    -\sin{\theta} & \cos{\theta}
  \end{pmatrix}
\end{equation}
Grupos ortogonales: los cuales se denotan por $G=O(N)$ las cuales son matrices ortogonales de $N\times N$ tal que: 
\begin{equation}
  O^TO=I
\end{equation}
\begin{align*}
  g_1\in O(N) \quad y \quad g_2\in O(N) \
  g_3 = g_1g_2 \in O(N) \\
  g_3 = g_1g_2 \in O(N) \\
  g^T_3 g_3 &  = (g_1g_2)^T (g_1g_2)  \\
  & = g_2^Tg_1^Tg_1g_2 \\
  & = g_2^TIg_2 \\
  & = I
\end{align*}
Tarea, mostrar si $g\in O(N) $ tal que $g^{-1}\in O(N)$ 
Será que la 2 esfera será un grupo manifold? no, pero sera que una 3 esfera será un grupo? si \\
Grupos unitarios: \\
Se define el grupo
ver videos de curvas elípticas
\end{document}
