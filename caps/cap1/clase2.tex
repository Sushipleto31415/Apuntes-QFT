\documentclass ../main.tex]{subfiles}

\begin{document}
\subsection{Cantidades conservadas}

%quizas aca introducir como en el estudio de las particulas elementales van a haber casos donde no se conserve la # de particulas, asi que se necesita estudiar en un marco mayor que nos permita ver más alla?? no se grande emily noether. 

Aunque las ecuaciones de Euler-Lagrange describirán la dinámica de un sistema bajo un marco referencial. Necesitamos un marco conceptual que nos muestre como se transforma la dinámica del sistema y si es que algún tipo de simetría de este nos permita conservar ciertas cantidades. \\

Supongamos que tienen una partícula en presencia de una energía potencial $U(x)$ a lo cual, la 2da ley de Newton nos dice lo siguiente: 
$$ m \frac{d^2x(t)}{dt^2}=-\frac{\partial U(x)}{\partial x }$$
A lo cual multiplicamos por $\frac{dX}{dt}$
$$ m \frac{dX}{dt}\frac{d^2x(t)}{dt^2}=-\frac{dX}{dt}\frac{\partial U(x)}{\partial x }$$
Ahora, sacamos la derivada temporal hacia fuera
$$ \frac{d}{dt} \left(  \frac{m}{2} \left( \frac{dx}{dt} \right)^2\right)= -\frac{d}{dt} \left( \frac{\partial x}{\partial x}\right)$$
$$\frac{d}{dt} \left(  \frac{m}{2} \left( \frac{dx}{dt} \right)^2+ \left( \frac{\partial x}{\partial x}\right) \right)=0 $$
La combinación
\begin{equation}
    \frac{m}{2}\left( \frac{dx}{dt}\right)^2+U(x)=E
\end{equation}
En lo cual $E$ corresponde a la energía del sistema. $E $ es una constante, con lo cual no depende del tiempo
En general diremos que una cantidad $Q=a(x,\dot(x)$ es conservada si
\begin{equation}
    \frac{d}{dt} Q\left( x(t), \frac{dx(t)}{dt}\right)=0
\end{equation}
En el contexto de la mecánica cláscia en el que estamos interesados en encontrar $x(t)$, las cantidades conservadas con extremadamente útiles. Las cantidades conservadas tambíen se llaman integrales de movimiento.  Si un sistema tiene un número suficientemente alto de integrales de movimiento, entonces podemos encontrar las historias de los grados de libertad sin integrar . \\
\begin{equation}
    q_i=q_i(t) \quad i=1,\dots ,N
\end{equation}
Teorema de Noether: Si el funcional de la acción es quasi-invariante bajo una transformacion infinisetimal, entonces existirá una cantidad conservada asociada a la transformación.  \\
Ejemplos de transformaciones infinitesimales, la transformación infinitesimal tendrá una forma bien precisa, dado lo siguiente. \\
\textbf{Transformación: traslación temporal}. Sea una coordenada $q(t)$ que depende del tiempo, a la cual haremos una tralación al futoro en $a$ seg. $q(t-a)$, notemos que $a $ puede ser cualquier número, digamos que $a$ es infinitesimal, con lo cual lo llamaremos $\epsilon$, ahora, tomando la serie de Taylor a $q(t-\epsilon$ tenemos lo siguiente:
\begin{equation}
    q(t-\epsilon)=q(t) - \epsilon \frac{dq}{dt}+ O(\epsilon^2)
\end{equation}
Para lo cual el tèrmino de $O(\epsilon^2)$ puede ser despreciado ya que será muy pequeño, ahora sigamos
\begin{equation}
    q(t-\epsilon)-q(t) =- \epsilon \frac{dq}{dt}=\delta q
\end{equation}
A $\delta q$ lo llamaremos traslación temporal \\
\textbf{Trasformación: Traslación espacial}
Sea un vector posición $r(t)$ el cual es situado con respecto a un eje coordenado cartesiano al cual lo trasladaremos espacialmente en un vector $a$ con lo cual la posición luego de la traslación será $\vec{r}(t)+\vec{a}$, ahora bien, supongamos que el vector $\vec{a}$ es infinitesimal, con lo cual la llamaremos $\vec{\epsilon}$, asì, la traslacion temporal infinitesimal estará dado por
\begin{equation}
    (\vec{r(t)}+\vec{\epsilon} )- \vec{r(t)}=\vec{\epsilon}=\delta \vec{r}
\end{equation}
En lo cual $\delta \vec{r}$ es llamada traslación espacial. \\
\textbf{Transfomación: Rotación espacial.} \\ Sabemos que en una rotación espacial una cantidad conservada sería el momento angular. Ahora, definamos una rotación.
\begin{align}
    x \prime &  = \cos{\theta}x- \sin{\theta}y \\
    y \prime &  = \sin{\theta}x + \cos{\theta}y
\end{align}
Ahora, en el caso que la rotación fuera infinitesimal, llamaremos $\theta=\epsilon$, con lo cual la rotación definida quedaría dada por
\begin{align}
    x \prime = x- \epsilon y \xrightarrow{} \delta x = x \prime - x = -\epsilon y \\
    y \prime  = \epsilon x + y \xrightarrow{} \delta y = y \prime  - y = \epsilon x
\end{align}
con lo cual obtenemos que
\begin{align}
    \delta x  & = - \epsilon y \\
    \delta y  & = \epsilon x \\
    \delta z &  = 0
\end{align}
Así la rotación espacial según el vector posicion $\vec{r}$ sería 
\begin{equation}
    \delta \vec{r} = \vec{r} \times \delta \hat{\phi}
\end{equation}
Acordar que el producto vectorial solo tiene sentido en 3 y 7 dimensiones. \\
Ahora hablemos de la acción 
\begin{equation}
    S [q(t)] = \int dt L(q,\dot{q})
\end{equation}
Ahora, se define la acciòn quasi-invariante como:
\begin{equation}
    \delta S = S[q+ \delta q] - S [q] = \int dt \frac{dB}{dt}
\end{equation}
B es una función que depende del tiempo. \\
Encontraremos que, en el caso que $B=0$ decimos que la acción es invariante, desarollando obtenemos
\begin{align}
    \delta S &  = \int dt \left( \partial_q L \delta q + \partial_{\dot{q}} L \frac{d}{dt} \delta q \right) \\
    & = \int dt \left(   \partial_q L - \frac{d}{dt} \partial_{\dot{q}} L  \right) \delta q + \int dt \frac{d}{dt} \left(  \partial_{\dot{q}}L \; \delta q\right)
\end{align}
Usando la ecuación de movimiento obtenemos:
\begin{align}
    \delta S & = \int dt  \frac{d}{dt}\left(   \partial_{\dot{q}}L \;\delta q \right) = \int dt \frac{dB}{dt} \\
     & =  \int_{t_1}^{t_2} dt \frac{d}{dt} \left(  \partial_{\dot{q}} L \delta q - B\right) = 0
\end{align}
si usted es capaz de encontrar una transformación que deja la accioón quasi-invariante, entonces la siguente cantidad encontrará que es constante
\begin{equation}
    \partial_{\dot{q}}L \delta q - B = C^{te}
\end{equation}
en lo cual la constante no dependerá del tiempo.
Ahora veamos que sucede cuando usamos una traslación temporal. \\
\textbf{Traslación temporal: }
\begin{equation}
    S[q(t)] = \int dt \left[\frac{m \dot{q}^2}{2} - U(q) \right] 
\end{equation}
Ahora bien, la variación de la acción de define por
\begin{align}
    \delta S  & = S[ q + \delta q] - S[q]  \\
   & = \int dt \left[ \frac{m}{2} \left(\frac{d}{dt}\left( q - \epsilon \frac{dq}{dt} \right) \right)^2  - U(q)  - \epsilon \dot{q}\right] - \int dt \left[ \frac{m\dot{q}^2}{2} - U(q) \right] \\
   & = \int dt \left[  \frac{m}{2} (\dot{q}^2 - 2 \epsilon \dot{q}\ddot{q}) - U(q) + \epsilon \dot{q} \partial_tU\right] - \int dt \left[  \frac{m}{2} \dot{q}^2- U(q)\right] \\
   & = \int dt \left[ -m\epsilon \dot{q}\ddot{q} +\epsilon \dot{q} \partial_tU\right] \\
   & = \int dt \frac{d}{dt} \left[   \epsilon \left(  -\frac{m}{2} \dot{q}^2 + U(q)\right) \right]
\end{align}
Con lo cual hemos encontrado nuestra función $B$ para esta traslación en particular. Tal que
\begin{equation}
    B= \epsilon \left(  -\frac{m}{2} \dot{q}^2 + U(q)\right)
\end{equation}
Notese que en este caso nunca usamos la ecuación de movimiento para encontrar cuánto vale $B$ en el caso de esta traslación. Ahora que sabemos cual es el valo de la función $B$, entonces podemos calcular cúal es la cantidad conservada segùn lo obtenido anteriormente.
\begin{equation}
    \partial{\dot{q}}L= m\dot{q}
\end{equation}
Así, la cantidad conservada está dada por 
\begin{equation}
    C^{te}= m\dot{q} (-\epsilon \dot{q} ) - \epsilon \left(  -\frac{m}{2} \dot{q}^2 + U(q)\right)
\end{equation}
De lo cual podemos identificar a la energía del sistema, con lo cual
\begin{equation}
    C^{te}=-\epsilon \left( \frac{m\dot{q}^2}{2}  + U(q)\right) = -\epsilon E
\end{equation}
Asì, la conservaciòn d la energía emerge como la aplicación del teorema de Noether a la quasi-invariancia bajo transformaciones temporales. \\
\textbf{Acción de la partícula libre:} Sabemos que la acción de la partícula libre está dada por 
\begin{equation}
    S = \int dt \frac{m}{2} \big| \frac{d\vec{r}}{dt} \big| ^2 
\end{equation}
Ahora bien, si usamos la convención de Einstein
\begin{equation}
    S = \int dt \frac{m}{2}\frac{dx^i}{dt}\frac{dx^i}{dt} \quad , x^i=(x^1,x^2,x^3) 
\end{equation}
Ahora usaremos traslaciones espaciales. \\
\textbf{Traslaciones espaciales:}
\begin{equation}
    \delta x^i= \epsilon^i \quad  , \quad \delta\vec{r}= \vec{\epsilon}
\end{equation}
Ahora lo aplicamos a la variación de la acción:
\begin{align}
 S[x + \delta x ] = \int dt \frac{m}{2} \frac{d}{dt}(x^i + \epsilon^i)  \frac{d}{dt}(x^i + \epsilon^i)  \\
 = \int dt \frac{m}{2} \frac{dx^i}{dt}\frac{dx^i}{dt} = S[x]
\end{align}
Con lo cual
\begin{equation}
    \delta S = S[x+\delta x] - S [x]= 0 = \int dt \frac{d}{dt} 0
\end{equation}
Para un grado de libertad : 
\begin{equation}
    \partial_{\dot{q}}L \delta q - B = C^{te}
\end{equation}
Para varios grados de libertad obtenemos
\begin{equation}
    \partial_{\dot{q_i}}L \delta q_i - B = C^{te}
\end{equation}
Y para traslaciones espaciales
\begin{equation}
    \partial_{\dot{x^k}}L \delta x^k - B = C^{te}
\end{equation}
Ahora, si tenemos un lagrangeano para varios grados de libertad $L=L(x^i, \dot{x^i})$ se obtiene lo siguiente
\begin{align}
    \partial_{\dot{x^k}} L = \partial_{\dot{x^k}} (  \frac{1}{2}m x ^i x ^i ) \\
     = \frac{m}{2} \left(  \frac{\partial \dot{x^i}}{\partial \dot{x^k} x^i + x^i \partial \dot{x^k}}\right) \\
     = \frac{m}{2} \left( \delta_k^i  \dot{x^i} + \dot{x^i} \delta_k^i \right) \\
     = m\dot{x^k}
\end{align}
En lo cual notamos que solo sobrevive ese términos por las deltas de Kronecker. \\
Ahora, la cantidad conservada está dada por:
\begin{equation}
    m\dot{x^k}\epsilon^k-B = C^{te}
\end{equation}
En lo cual, como sabemos, en una transformación traslación $B=0$
Con lo cual, podemos concluir que:
\begin{equation}
    m\dot{x^k} \epsilon^k = C^{te}
\end{equation}

por lo tanto, de igual forma se cumplirá: 
\begin{equation}
    m\dot{x^k} = \tilde{C}^{te}
\end{equation}

y así, en transformaciones espaciales la cantidad conservada será el momento lineal
\begin{equation}
    m\vec{v}=\vec{p}
\end{equation}

\end{document}
%%%%%%%%%%%%%%%%%%%%%%%%%%%%%%%%%%%%%%%%%%%%%%%%%%%%%%%%%%%%%%%%%%%%%

