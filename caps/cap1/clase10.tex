\documentclass ../main.tex]{subfiles}

\begin{document}
\section{Décima clase}
Los grupos de Lie son superficies quee pueden tener componentes desconectados. El elemento identidad corresponderá a un punto en alguna de las piezas. Tal pieza define un subgrupo del grupo completo. En los grupos los elementos conectados a al identidad pueden ser escritos de la forma: \\
Me puedo imaginar que la primera superfice tiene curvas coordenadas, lo cual en una superfice bidimensionar tiene dos familias de curvas coordenadas, entonces a cada punto le doy dos números. Para lo cual podemos dividir el espacio en parámetros $\alpha$ lo cual podrá describir un punto en la superficie y por tanto, un punto en el espacio. Como lo puede ser, por ejemplo, el punto
\begin{equation*}
  g = g(\alpha_A)
\end{equation*}
Pensando en grupo clásicos, los $g$ serán matrices de algún tamaño, tal que el elemento $g$ puede ser escrito de la siguiente manera
\begin{align*}
  g(\alpha_A)  & = e^{i(\lambda_1T_1+\lambda_2T_2 +\lambda_nT_n)} \\
  g(\alpha_A)  & = e^{i(\lambda_1T_1+\lambda_2T_2 + ... +\lambda_nT_n)} \\
  & = e^{i\lambda_AT_A}
\end{align*}
Notemos que, esta forma exponencial solo es válida para lo elementos de la superficie del grupo conectada a la identidad. \\
Para lo cual
\begin{itemize}
  \item $\Lambda_A$, números reales, (los alpha que llamó el profe)
  \item $T_A$ matrices generadoras del álgrebra 
\end{itemize}
Yo puedo tomar un elemento del grupo y multiplicarlo con otro, lo cual por la propiedad de clausura, debe dar otro elemento del grupo
\begin{align*}
  g(\lambda_A) g(\beta_B) = g(\gamma_C= \gamma_C(\lambda_A,\beta_B))
\end{align*}
Lo cual será según la tabla de multiplicación del grupo de Lie, lo cual, a diferencia de los grupos discretos, corresponde a un conjunto de funciones $\gamma$ y el como se relacionan con los parámetros reales $\alpha_A$ y $\beta_B$, a las cuales les llamaremos constantes de estructura. 
\begin{equation*}
  [T_A,T_B] = f_{ABC}T_C
\end{equation*}
Lo cual se denote como, álgebra de LIe asociada al grupo de Lie. \\
¿ Qué forma toma $[T_A,T_B] = f_{ABC}T_C$ para el grupo de Lorentz? \\
Dada una representación del álgebra, es decir un conjunto de $N$ matrices $T_A$ con $A=1,\dots,N$ tal que $[T_A,T_B]=f_{ABC}T_C$, podremos encontrar una representación del grupo, exponiendo tales matrices. Si los $\lambda_A$ so infinitesimales, entonces expando a primer orden la exponencial
\begin{equation}
  g(\lambda_A) = I + i\lambda_AT_A + O{\lambda^2}
\end{equation}
Para hacer teoría de campos, nos basta saber la estructura del álgebra del grupo de Lorentz. \\
Vamos al grupo de Lorentz entonces. \\
Los grupos en física actúan sobre objetos físicos. Voy a definir un objeto físicos transformado, como la acción del elemento del grupo, actuando sobre el objeto físico no transformado
\begin{equation*}
  \tilde{\Phi}^I = g^I_{\; J}(\lambda_A)\Phi^I
\end{equation*}
en donde los índices $I$ y $J$ van desde 1 hasta la dimensión de la representación, por ejemplo en caso que sean matrices de $2\times 2$ entonces toma dos valores, y así. \\
En caso que los parámetros $\lambda_A$ estén conectados en el espacio que tenga la identidad, entonces. Una transformación infinitesimal actuará:
\begin{align*}
  \tilde{\Phi}^I & = \left( I + i\lambda_A T_A \right)^I_{\; K}\Phi^J \\
  & = I^I_{\; J} \Phi^I + i\lambda_A \left(T_A\right)^I_{\; J}\Phi^I \\
  & = \Phi^I + i\lambda_A \left(T_A\right)^I_{\;J}\Phi^J 
\end{align*}
Lo cual, si restamos el transformado con el elemento original, tenemos
\begin{equation*}
  \delta \Phi^I : = \tilde{\Phi}^i - \Phi^I = i\lambda_A\left(T_A\right)^I_{\; J}\Phi^I 
\end{equation*}
Con lo cual, la manera en la cual todo lo elemetos cercanos a la identidad actúan sobre el resto del grupo
\begin{equation}
  \boxed{  \delta \Phi^I  = i\lambda_A\left(T_A\right)^I_{\; J}\Phi^J}
\end{equation}
El grupo SU(2) ya lo definios, 
\begin{equation*}
  U\in SU(2) ,\text{if}\; U^\dagger U = I \; \& det(U) = 1  
\end{equation*}
Lo cual,
\begin{align*}
  [T_A,T_B] = i\epsilon_{ABC}T_C \\
  [T_1,T_2] =iT_3 \\
  [T_2,T_3] = iT_1 \\
  [T_2,T_3] = iT_1 \\
  [T_3,T_1] = iT_2
\end{align*}
Ahora tenemos las representación.\\
\textbf{Representación trivial: } La cual se usar para representar a los objetos sin rotación
\begin{equation*}
  (T_1)= 0,\quad (T_2)= 0,\quad (T_3)=0
\end{equation*}
\textbf{Representación de spin 1/2:} Representación fundamental de $su$(2)
\begin{align*}
  T_1 & = 1/2\begin{pmatrix} 0 & 1 \\ 1 & 0 \end{pmatrix} \\
  T_2 & = 1/2\begin{pmatrix}0 & -i \\ i & 0 \end{pmatrix} \\
  T_3 & = 1/2\begin{pmatrix} 1 & 0 \\ 0 & -1 \end{pmatrix}
\end{align*}
\textbf{Representación de spin 1}
\begin{align*}
  T_1 & = 1/\sqrt{2}\begin{pmatrix} 0 & 1 & 0 \\ 1 & 0 & 1 \\ 0 & 1 & 0 \end{pmatrix} \\
  T_2 & = 1/\sqrt{2}\begin{pmatrix} 0 & -i & 0 \\ i & 0 & i \\ 0 & i & 0 \end{pmatrix} \\
  T_3 & = 1/\sqrt{2}\begin{pmatrix} 1 & 0 & 0 \\ 0 & 0 & 0 \\ 0 & 0 & -1 \end{pmatrix}
\end{align*}
Lo cual
\begin{align*}
  T_1T_2 - T_2 T_2 &  = 1/2 \left[ \begin{pmatrix} 0 & 1 & 0 \\ 1 & 0 & 1 \\ 0 & 1 & 0 \end{pmatrix} \begin{pmatrix} 0 & -i & 0 \\ i & 0 & -i \\ 0 & i & 0 \end{pmatrix}  - \begin{pmatrix}  0 & -i & 0 \\ i & 0 & -i \\ 0 & i & 0 \end{pmatrix}\begin{pmatrix}0 & 1 & 0 \\ 1 & 0 & 1 \\ 0 & 1 & 0 \end{pmatrix} \right] \\
& = 1/2\left[\begin{pmatrix} i & 0 -i \\ 0 & 0 & 0 \\ i & 0 & -i \end{pmatrix} - \begin{pmatrix} -i & 0 & -i \\ 0 & 0 & 0 \\ i & 0 & i \end{pmatrix}\right] \\
  & = 1/2 \begin{pmatrix} 2i & 0 & 0 \\ 0 & 0 & 0 \\ 0 & 0 & -2i \end{pmatrix} \\
    & iT_3
\end{align*}
Lie algebra in particle physcics A. Georgi. \\
Ahora, otra representación de matrices de $3\times 3$
\begin{align*}
  T_1 & = \begin{pmatrix} T_1^{\text{spin 1/2}} & 0 \\ 0 & 0 \end{pmatrix} \\
    T_2 & = \begin{pmatrix} T_2^{\text{spin 1/2}} & 0 \\ 0 & 0 \end{pmatrix} \\
      T_3 & = \begin{pmatrix} T_3^{\text{spin 1/2}} & 0 \\ 0 & 0 \end{pmatrix} 
\end{align*}
Las irreps de $su$(2) están clasificadas y están etiquetadas por un semi entero, $s=0,\frac{1}{2}, 1 , \frac{3}{2},\dots$ y son matrices de $(2s+1)\times (2s+1)$. Para cada valor de $s$ hay una única irrep. \\
\textbf{Representación conjugada:}
\begin{align*}
  (T_AT_B - T_BT_A)^* & = (i\epsilon_{ABC}T_C)^* \\
  T_A^*T_B^*-T_B^*T_A^* & = -i\epsilon_{ABC}T_C^* \\
  (-T_A^*)(-T_B^*) - (T_B^*)(T_A^*) & = \epsilon_{ABC}(-T_C^*)
\end{align*}
Por tanto, las matrices 
\begin{equation*}
  \tilde{T_A} = -T_A^*
\end{equation*}
Y por tanto
\begin{equation*}
  \left[\tilde{T_A},\tilde{T_B} = i\epsilon_{ABC}\tilde{T_C}\right]
\end{equation*}
La representación de spin 1/2 se le llama \textbf{2}, la cual está relacionada con la representación conjugada, de forma que ambas son equivalentes, la cual está dada por\\
\textbf{Representación conjugada (antifundamental)}
\begin{align*}
  \tilde{T_1} & = 1/2\begin{pmatrix} 0 & -1 \\ -1 & 0 \end{pmatrix} \\
  \tilde{T_2} & = 1/2\begin{pmatrix} 0 & -i \\ i & 0 \end{pmatrix}\\
    \tilde{T_3} & = 1/2 \begin{pmatrix} -1 & 0 \\ 0 & 1 \end{pmatrix}
\end{align*}
Representación la cual se denota por $\bar{textbf{2}}$. Ahora, si bien las representaciones son equivalentes via conjugación, estas no describen la misma física.
\begin{equation*}
  \bar{\textbf{2}} \equiv \textbf{2}
\end{equation*}
TAREA, encontrar la siguiente matriz $A$. 
\begin{align*}
  \tilde{T_1} & = A^{-1}T_1A \\
  \tilde{T_2} & = A^{-1}T_2A \\
  \tilde{T_3} & = A^{-1}T_3A
\end{align*}
Volvamos al grupo de Lorentz. 
\begin{align*}
  \tilde{X}^\mu  = \Lambda_{\;\nu}^\mu X^\nu \\
  \eta_{\mu \nu}\Lambda_{\; \alpha}^\mu \Lambda_{\; \beta}^\nu = \eta_{\alpha \beta}
\end{align*}
Para transformaciones de Lorentz infinitesimales 
\begin{equation*}
  \Lambda_{\;\beta}^\alpha = \delta_{\; \beta}^\alpha + \omega_{\; \beta}^{\alpha}
\end{equation*}
Con lo cual, tenemos lo siguiente
\begin{align*}
  \eta_{\mu \nu} \left(\delta_{\; \alpha}^\nu + \omega_{\; \alpha}^\mu\right)\left( \delta_{\;\beta}^\nu + \omega_{\;\beta}^\nu \right) & = \eta_{\alpha \beta} \\
  \eta_{\mu \nu} \delta_{\; \alpha}^\mu \delta_{\; \beta}^\nu + \eta_{\mu \nu} \omega^\mu_{\; \alpha} \delta_{\; \beta}^\alpha + \eta_{\mu \nu}\delta{\;\alpha}^\mu \omega^\nu_{\; \beta} + O(\omega^2) & = \eta_{\alpha \beta} \\
  \eta_{\mu \beta}\omega_{\; \alpha}^\mu + \eta{\alpha \nu}\omega^\nu_{\; \beta} & = 0
  \text{Definición} \quad \omega_{\beta \alpha} + \omega_{\alpha \beta} = 0 \\
  \boxed{\omega_{\alpha \beta} = -\omega_{\beta \alpha}}
\end{align*}
Con lo cual
\begin{align*}
  \tilde{X}^\mu &  =\left(\delta_{\; \nu}^\mu + \omega^\mu_{\; \nu}\right)X^\nu \\
   & = \delta_{\; \nu}^\mu X^\nu + \omega^\mu_{\; \nu} X^\nu \\
   & = X^\mu + \omega^\mu_{\; \nu}X^\nu \\
\end{align*}
Y asi se obtiene que
\begin{equation*}
  \delta X^\mu := \tilde{X}^\mu - X^\mu = \omega^\mu_{\;\nu}X^\nu
\end{equation*}
Desarollamos esta definición
\begin{align*}
  \delta X^\mu & = \omega_{\; \nu}^\mu X^\nu \\
  & = i\frac{1}{2}\omega_{\alpha \beta}\left(T^{\alpha \beta}\right)^\mu_{\; \nu}X^\nu \\
  \left(T^{\alpha \beta}\right)^\mu_{\; \nu} = \#\left(\delta_{\; \nu}^\mu - \delta^\beta_{\; \nu}\eta^{\alpha \mu}\right) \\
  & = i\frac{\#}{2}\omega_{\alpha\beta} \left(\delta_{\nu}^{\alpha}\eta^{\beta \nu} - \delta_{\nu}^\beta\eta^{\alpha \nu}\right)X^\nu \\
  & = \#\omega^\mu_\nu X^\nu = \omega^\mu_\nu X^\nu , \quad \#=-i \\
  & = \# \omega^\mu_\nu X^\nu = \omega^\mu_\nu X^\nu , \quad \#=-i \\
  & = \frac{i}{2}\omega_{\alpha \beta} \left(T^{\alpha \beta}\right)^{\mu}_{\; \nu}X^\nu 
\end{align*}
Donde 
\begin{equation*}
  \left(T^{\alpha\beta}\right)^\nu_{\; \nu} = i \left(\delta^\alpha_\nu \eta^{\beta \mu} - \delta^\beta_{\; \nu}\eta^{\alpha \nu}\right)
\end{equation*}
\end{document}
