\documentclass[twoside]{report}
\usepackage{subfiles}
\usepackage{graphicx} %Requerido para insertar imágenes
\usepackage[paper=letterpaper, left=1in,right=1in,top=1in,bottom=1in ]{geometry}
\usepackage[spanish,es-tabla]{babel}
\usepackage[utf8]{inputenc}
\usepackage{tikz}

\newcommand*\vtick{\textsc{\char13}}

%Personalización de pies de página y encabezados
\usepackage{fancyhdr}
\pagestyle{fancy}
\fancyhf{}
\fancyhead[LE,RO]{\textcolor[RGB]{127,127,127}{Introducción a QFT}}
\fancyfoot[LE,RO]{\thepage}
\fancyfoot[LO,RE]{A. Díaz y F. Mella}


%Columnas más bonitas
\usepackage{tabularx, xcolor}
\definecolor{tcc}{RGB}{217,217,217} % Color de fondo para celdas
\renewcommand\tabularxcolumn[1]{m{#1}}
\renewcommand{\arraystretch}{1.5}



%Símbolos matemáticos más bonitos
\usepackage{amsmath, amssymb, physics}
\numberwithin{equation}{chapter} % Numeración por sección
\usepackage{cancel}


%Estilo de la bibliografía
\usepackage[square,numbers,sort&compress]{natbib}
\bibliographystyle{apsrev4-1} % Estilo físico


 
\makeatletter
\renewcommand{\part}[1]{%
  \chapter*{#1} % Usamos \chapter* para que no se numere el "part"
  \addcontentsline{toc}{part}{#1} % Añadimos el "part" al índice
  \vspace{-1.5cm}  % Reducir el espacio debajo del título
}
\makeatother

\begin{document}

\subfile{portada}

\setcounter{page}{1}
\pagenumbering{roman}


\subfile{caps/intro}

\tableofcontents
\pagenumbering{arabic}
\setcounter{page}{1}


\newpage


%%%%%%%%%%%%%%%%%%%%%%%%%%%%%%%%%%%%%%%%%%%%%%%%%%%%%%%%%%%%%%%%%%%%%%%%%%%%%%
%                          CAPÍTULO 1
%%%%%%%%%%%%%%%%%%%%%%%%%%%%%%%%%%%%%%%%%%%%%%%%%%%%%%%%%%%%%%%%%%%%%%%%%%%%%%
 


\subfile{caps/cap1/clase1} %oyee ns si te tinca renombrar las secciones como del tema q se trata tipo...clase 1-> Invarianza de la acción/ecuaciones de Euler-Lagrange

\subfile{caps/cap1/clase2} % Cantidades conservadas / Teorema de Noether

\subfile{caps/cap1/clase3}% Relatividad especial 

\subfile{caps/cap1/clase4}% y esta sin nombre (la seccion) pq es continuacion directa d la otra 

\subfile{caps/cap1/clase5}

\subfile{caps/cap1/clase6}

\subfile{caps/cap1/clase7}

%%%%%%%%%%%%%%%%%%%%%%%%%%%%%%%%%%%%%%%%%%%%%%%%%%%%%%%%%%%%%%%%%%%%%%%%%%%%%%
%                          CAPÍTULO 2
%%%%%%%%%%%%%%%%%%%%%%%%%%%%%%%%%%%%%%%%%%%%%%%%%%%%%%%%%%%%%%%%%%%%%%%%%%%%%%


%\subfile{caps/cap2/sistema-de-referencia}

%\subfile{caps/cap2/sistema-de-coordenadas}

%\subfile{caps/cap2/bases-coordenadas}

%\subfile{caps/cap2/simbolos-de-christoffel}

%\subfile{caps/cap2/espacio-dual}

%\subfile{caps/cap2/operadores-diferenciales}

%\subfile{caps/cap2/prop-y-res-cap2}

%\newpage

%\bibliography{biblio}


\end{document}
